%\documentclass[11pt,letterpaper]{article}
\documentclass[oneside,11pt]{amsart}

%\usepackage{a4wide}
%\usepackage{epsfig}
%\usepackage{psfig}
\usepackage{graphicx}
\usepackage{natbib,latexsym,url,enumitem,pdfpages}
\usepackage{color}
\usepackage{wrapfig}
\usepackage{caption}

\captionsetup{
    justification=justified,
    margin=0pt,
    font=small}

% Some fancy commenting
\definecolor{todo}{RGB}{200,0,0}
\newcommand{\comment}[2][todo]{{\color{#1}[[{\bf #2}]]}}

% Challenge counter
\newcounter{chalno}
\newcommand{\chal}[1]{\refstepcounter{chalno}\label{#1}}

% User commands
\input{journaldefs}

\DeclareRobustCommand{\gtrsim}{%
\mathrel{\hskip-.5em\begin{array}{c}>\\[-8pt]\sim\end{array}\hskip-.5em}}
\DeclareRobustCommand{\lesssim}{%
\mathrel{\hskip-.5em\begin{array}{c}<\\[-8pt]\sim\end{array}\hskip-.5em}}

\pretolerance=10000
\textwidth=6.4in
\textheight=8.95in
\voffset = 0.in
%\voffset = -0.3in  % For my printer
\topmargin=0.0in
\headheight=0.00in
\hoffset = 0.0in
%\hoffset = 0.33in  %  For my printer
\headsep=0.00in
\oddsidemargin=0in
\evensidemargin=0in
\parindent=2em
\parskip=0.2ex
 
\renewcommand{\baselinestretch}{1.03}

\special{papersize=8.5in,11in}

\newcommand{\markus}{\textcolor{green}}

\setlength{\parskip}{0.6 ex plus 0.4ex minus 0.2ex} \flushbottom
\pagestyle{plain} 

\begin{document}
% \thispagestyle{empty}

\pagenumbering{arabic}

\vspace*{-1.5cm}

\centerline{\textsf {\Large Mid-scale RI-1 (M1:DP): Training for Big Astronomical Data with FOBOS on Keck}}
\centerline{\textsf {\large Project Summary}}
\centerline{\emph{{\small For review by the Division of Astronomical Sciences (AST)}}}

% \noindent\comment{1 pg, doesn't count towards 10-pg limit}

\noindent {\bf Overview:} In this Mid-scale RI-1 (M1:DP) \emph{Design} submission, we propose to dramatically enhance
the power of upcoming panoramic deep-imaging from the Large Synoptic Survey Telescope (LSST), Euclid and the Wide-Field
Infrared Survey Telescope (WFIRST) in order to address key questions in the areas of Dark
Energy, the galaxy ecosystem at $z\sim2$, and the assembly history of the Milky Way and Local Group Galaxies.  We will
design the astrostatistics, instrumentation, and software solutions required over the next decade to provide optimized
spectroscopic training sets and to unlock \emph{physically meaningful} information (e.g., redshifts, galaxy star
formation histories, stellar metallicities) from deep photometry alone.  Applying machine learning to ambitious Data
Science Challenges using simulated data, we will set requirements on future spectroscopic training sets.  These
requirements will guide the preliminary design of FOBOS, a powerful new spectrograph to deploy in 2026 on the 10 m
Keck II Telescope.  FOBOS will provide publicly-available deep, high-multiplex spectroscopy with high target
sampling and flexibility uniquely matched to the ``Big Data'' training problem.


\noindent {\bf Intellectual Merit:} High-multiplex and deep spectroscopic followup of LSST and other panoramic
deep-imaging surveys is a widely recognized necessity.  Reports in 2015 and 2016 by the National Research Council and
National Optical Astronomical Observatory specifically recommend that the NSF support construction of required
spectroscopic facilities because none currently exist or are planned at U.S.\ observatories.  FOBOS satisfies these
spectroscopic needs at relatively low cost by utilizing the existing 10 m Keck II Telescope, a highly-successful
U.S.-led large telescope.  Even then, the astronomical community requires cutting-edge data science techniques to
``train'' vast photometric surveys with what will necessarily be more limited spectroscopy.  Success in the training
methodologies we propose here will make photometric redshifts more precise, improving the LSST Dark Energy
figure-of-merit by 40\%.  They will enable a comprehensive understanding of galaxies and their gaseous environments at
$z\sim2$, and they will reveal fossilized structures in the Milky Way, M31, and other Local Group galaxies through
chemical signatures inferred for millions of stars.



\noindent {\bf Broader Impacts:}

\clearpage
\setcounter{page}{1}

\noindent\begin{center}\mbox{\parbox{0.95\linewidth}{
%
{\it This Mid-scale Research Infrastructure-1 “Design” proposal requests
funds to complete the preliminary instrument design for the Fiber-Optic
Broadband Optical Spectrograph (FOBOS) for the William M.\ Keck
Observatory, motivated by its ability to address data-driven science
goals via the acquisition of critical spectroscopic training sets.}
%
}}\end{center}

\section{Intellectual Merit}
\label{sec:im}

\subsection{Scientific Justification} 

Led by NSF's Large Synoptic Survey Telescope\footnote{
%
LSST will be begin science operations in 2023.}
%
(LSST), astronomy is entering a new era of unprecedented deep-imaging
data sets that will survey huge volumes of the Universe.  From the
emergence of the earliest galaxies from a ``primordial soup'' of gas and
dust, to the period of its most vigorous star formation at roughly
one-half to one-third of its current age, to the modern epoch of
decreased star formation and accelerated expansion by the mysterious
influence of Dark Energy, these surveys will provide unprecedented
statistics at key epochs of cosmic history.

% Meanwhile, the rate of cosmic expansion was beginning to accelerate,
% as the Universe became increasingly dominated by ``Dark Energy,''
% whose origin remains the single greatest mystery in astronomy and
% cosmology today.

% Since Edwin Hubble's observations over 100 years ago,

Even so, major advances in our understanding of the Universe typically
follow a two-step process.  The first is to blindly take images of the
sky to discover new sources, which LSST will provide in abundance;
however, the second is to follow-up and obtain information-rich
spectroscopy to reveal the astrophysical nature of those sources.  A
modern example is the Sloan Digital Sky Survey (SDSS) whose combination
of panoramic broad-band ``imaging'' followed by dedicated spectroscopy
yielded unprecedented in-depth data on over 1 million galaxies, mapping
the present-day universe and making SDSS one of the most highly cited
surveys in the history of astronomy.

% Because a quality spectrum requires far more observing time per source
% than an image, SDSS pioneered ``high multiplex'' spectrographs,
% capable of \emph{simultaneous} spectroscopy of hundreds of objects.

LSST's all-sky images will be 1,000 times deeper and detect far more
distant galaxies than SDSS, but \textbf{no current U.S. facility is
capable of obtaining spectroscopic followup of LSST galaxies} at a level
required to capitalize on the \$1B the U.S.\ has invested in that
project.  In fact, an SDSS-like spectroscopic study of 1 million
galaxies at LSST depth would require 300 years of observing on the
largest telescopes with current instrumentation!  

The only way forward is encapsulated in one of NSF's ``10 Big Ideas,''
\emph{Harnessing the Data Revolution}: we can maximize the information
content of LSST and other imaging facilities via machine learning from
optimally designed spectroscopic training sets.  This proposal presents
a coordinated framework with three critical components necessary for
success in this endeavor: 1) Using simulated spectroscopic$+$imaging
data to define the training sets required to address ambitious
data-science challenges in Cosmology, Galaxy Formation, and Local Group
Archaeology in the LSST era; 2) Preliminary design of
FOBOS\footnote{FOBOS: Fiber-Optic Broadband Optical Spectrograph}, a
state-of-the-art spectroscopic facility for WMKO\footnote{WMKO: William
M.\ Keck Observatory operates the two twin 10m Keck Telescopes on
Maunakea, Hawaii.}, optimized for providing critical training data using
one of the world's largest telescopes; 3) Preliminary design of the
coordinated FOBOS observations required as well as the systems needed to
serve training data publicly.  This {\bf MSRI-1 Design} proposal lays
out the path for maximizing panoramic imaging from LSST,
WFIRST\footnote{WFIRST is NASA's space-based Wide-Field Infrared Survey
Telescope, expected to launch in the mid 2020's.},
Euclid\footnote{Euclid is led by the European Space Agency with
significant NASA involvement and will launch in 2021. Its primary
mission is a 15,000 deg$^2$ imaging survey in optical and near-IR
wavebands.}, and other facilities with spectroscopic follow-up
unparalleled in depth and sampling density.  Through a subsequent MSRI
proposal we will deliver on our goals with an instrument deployment in
2026, an array of spectroscopic programs, and associated public-ready
training data.  \comment{comment on machine learning vs.\ instrument
funding}

\subsection{Research Community Priority} 
\label{sec:community}

The need for spectroscopic follow-up in the LSST era was made clear in
the National Research Council's 2015 report, ``Optimizing the U.S.
Ground-Based Optical and Infrared Astronomy System'' \citep{NAP21722}:
%
\noindent\begin{center}\mbox{\parbox{0.95\linewidth}{
%
The National Science Foundation should support the development of a
wide-field, highly multiplexed spectroscopic capability on a medium- or
large-aperture telescope in the Southern Hemisphere to enable a wide
variety of science, including follow-up spectroscopy of Large Synoptic
Survey Telescope targets. Examples of enabled science are studies of
cosmology, galaxy evolution, quasars, and the Milky Way.
%
}}\end{center}

Workshops organized by the National Optical Astronomy Observatory (NOAO)
in 2013 and 2016, the latter at the NSF's request, reported specific
spectroscopic needs for LSST follow-up in all science areas.  In
particular, the 2016 report notes that a critical resource in need of a
prompt development path is to "Develop or obtain access to a highly
multiplexed, wide-field optical multi-object spectroscopic capability on
an 8m-class telescope."  Based on these recommendations, we propose the
FOBOS instrument coupled with a suite of data-driven tools to address
the spectroscopic requirements of LSST and other photometric surveys at
a final cost 20 times less than a new Southern Hemisphere facility.
Located in Hawaii, FOBOS can access more than 70\% of the LSST
footprint, more than adequate for our primary goal of building powerful
spectroscopic training sets.  Compared to the Prime Focus Spectrograph
(PFS) on Japan's Subaru Telescope, FOBOS would be 1.7$\times$ faster,
provide unique UV sensitivity (0.31--1$\mu$nm compared to
0.38--1.25$\mu$ with PFS), and offer higher-density, more flexible
target sampling over ``deep-drilling'' fields.  FOBOS would be operated
on a U.S.\ telescope with dedicated U.S.\ access and a commitment to
supporting U.S.-led photometric surveys.  FOBOS is also complementary to
future ambitious facilities that would be optimized to cover wider areas
(several deg$^2$ per pointing) at shallower depths.

\comment{mention FOBOS can do PI-led science too}

The need for deep spectroscopic followup is particularly acute for
LSST's major cosmological probes, which rely on ``photometric
redshifts:'' measures of the redshifts, $z$, of objects --- a direct
proxy of distance and look-back time --- based on imaging alone.
\citet{newman15} summarize the case for this and describe a redshift
survey that, if carried out with FOBOS, would increase LSST's Dark
Energy figure-of-merit by 40\% at a cost of less than 5\% of the LSST
budget.  The urgent case for spectroscopic redshift training has been
the subject of numerous publications \citep[e.g.,][]{laureijs11,
masters15, hemmati18}.

Meanwhile, the astronomy community recognizes that the coming ``Big
Data'' era, culminating in LSST, necessitates ``\textbf{harnessing the
data revolution}.''  Widespread community interest in advanced
data-science techniques continues to grow amidst calls for educational
programs, conference series, and research funding to support the growth
of a new field, ``Astroinformatics,'' which exploits the interface
between astrophysics and statistics \citep{borne09}.  Astronomy's
largest organizations, including the American Astronomical Society and
the International Astronomical Union, have supported active working
groups on astroinformatics and astrostatistics since 2015.  LSST itself
has built the Informatics and Statistics Science Collaboration and
partnered with NSF to fund the Data Science Fellowship Program to train
astronomy graduate students in data-science techniques.  Our proposal
builds on and contributes to these ongoing efforts.

\subsection{Science Goals and Data-Science Challenges}
\label{sec:goals}

We identify ambitious ``data-science challenges'' for the LSST era that
would address major goals within each of three core topics.  By
simulating future wide-field imaging data complemented by FOBOS spectroscopy,
we will develop astrostatistical techniques and applications over the
proposal period that will refine the FOBOS instrument requirements,
inform the emerging design and operational modes, and define required
training data.  Tackling these challenges requires a community-wide
effort and will deliver wide-spread benefits.  Our goal with this
proposal is to establish community priorities and success metrics and to
coordinate the various groups working in this area---many represented
among our Senior Personnel.

% \comment{modify the motivation here to reflect our funding request;
% move away from being LSST centric}

\begin{figure}[h!]
%
\vskip -0.2in
%
\includegraphics[width=\textwidth]{figs/Hemmati18_Fig8_VVDS_spec.png}
%
\caption{\small {\it Left}: A Self-Organizing Map
\citep[SOM;][]{1990Natur.346...24K} from \citet{hemmati18} encoding the
relation between colors in an LSST+WFIRST-like color space and redshift,
$z$.  Position in the SOM is associated with a position in the
multi-dimensional broad-band color space of galaxies.  Galaxies observed
in this space are assigned $z$ values based on the median photo-$z$ of
galaxies from the CANDELS survey \citep[color
bar;][]{2011ApJS..197...35G}.  Such SOMs can be used to optimally define
spectroscopic training samples for use with imaging surveys.  {\it
Right}: Galaxy spectra from VVDS \citep{2005A&A...439..845L}; black
crosses near the top and bottom of the SOM are plotted in the top and
bottom panels, respectively.  Note the similarity of the high-resolution
spectra associated within the SOM, suggesting that a systematic
spectroscopic exploration of the LSST color space would have
far-reaching benefits to the science return of the mission beyond the
photo-$z$ application.}
%
\label{fig:SOM}
%
\end{figure}

%-----------------------------------------------------------------------
\subsubsection{Enhancing Dark Energy Probes via Precision Cosmic Distances}
\label{sec:cosmology}

The 2011 Nobel Prize in Physics was awarded for the discovery that the
expansion of the universe is accelerating due to the mysterious ``dark
energy,'' the origin of which remains unknown.  Dark energy is one of
the most fundamental, unsolved problems in both cosmology and particle
physics.  It has inspired enormous world-wide efforts --- culminating in
LSST, Euclid, and WFIRST --- that seek highly precise measures of cosmic
structure to constrain the evolving dark-energy equation-of-state.

These measures utilize angular correlations of galaxy positions, their
gravitational lensing shear, and the cross-correlation between the two.
Unfortunately, photometric distances (via photometric redshifts, or
``photo-$z$s'') are significantly less precise than spectroscopic
redshifts (spec-$z$s), introducing significant biases.  The
spectroscopic validation of photo-$z$s we propose with FOBOS is
therefore critical to the success of {\it all} imaging surveys in this
respect. It would not only \emph{increase the dark energy
figure-of-merit in LSST by 40\%} \citep{newman15} but, importantly,
provide vital confidence in cosmological results.  FOBOS is particularly
powerful in this application because it has no redshift desert.  It has
a unique ability to measure spectroscopic redshifts above $z > 1.5$ via
rest-frame UV features, eliminating the need for expensive, space-based
near-IR spectroscopy.

\chal{photozs}
%
\begin{enumerate}[rightmargin=0.2cm,leftmargin=0.2cm]
%
\item[] {\textsf {\large Data-Science Challenge \ref{photozs}: Enable
high-precision LSST photometric redshifts with strategically designed
training spectroscopy:}}  FOBOS is ideally suited to obtaining large
($>10,000$ galaxies) and deep spectroscopic training sets required for
LSST, WFIRST, and Euclid \citep[see][] {newman15}.  Applying
machine-learning, template-based, and hybrid photo-$z$ estimators to
simulated data, we will build a methodology for designing FOBOS training
sets that minimize observing time while maintaining adequate
parameter-space sampling\footnote{
%
Of course, the design of this sample also benefits the data-science
challenges we detail in Section \ref{sec:galaxies}.}.
%
Self-Organizing Maps (SOM, Fig.~\ref {fig:SOM}) provide a
state-of-the-art representation of a high-dimensional input space in
projected 2D grid cells, allowing us to benchmark sampling of the
photometric color space under various training set designs.  We will use
Bayesian Optimization techniques to evaluate training success for
cosmological parameters, which enable extremely rapid exploration of the
optimal design space.

\end{enumerate}

%-----------------------------------------------------------------------
\begin{wrapfigure}{r}{0.5\textwidth}\small
%
\includegraphics[width=0.5\textwidth]{figs/Parks_fig.pdf}
%
\caption{Application of machine learning to find and quantify the
physical parameters of absorption by neutral hydrogen gas in spectra
taken along quasar sight lines \citep[adapted from Figs 7 and 14
from][]{parks18}.  Two absorption systems in the spectrum (top-left) are
identified (middle-left) and then labelled with an HI column density
($N_{\rm HI}$) (bottom-left) using a convolutional neural network.
Distributions for the differences in redshift, $z$, (top-right) and
$N_{\rm HI}$ (bottom-right) between the results of the CNN and direct
derivations by experts are in excellent agreement.  FOBOS will provide
rich data sets for similar transfer of physical-parameter labels to
photometric and spectroscopic data.}
%
\label{fig:absorber}
%
\end{wrapfigure}

\subsubsection{A Comprehensive Picture of the Proto-galaxy Ecosystem}
\label{sec:galaxies}

% From George:
% - fill out case for probing both galaxies and their “gas-filled
%   environments”
%    - make it more explicit that getting large numbers of redshifts
%      would make it possible to trace out large-scale structure in
%      detail
%    - enables studies of galaxy properties as a function of environment
%
% - also mention targeting galaxies along QSO lines of sight
%    - much higher target density than with LRIS, DEIMOS over larger FOV.
%
% - Worth discussing Lyman-alpha or metal-line tomography?  
%
% - More quantitative comparisons with existing data sets?
%    - What key science questions can FOBOS address that many years of
%      LRIS and DEIMOS observations have not been able to?  Surely some
%      level of the spectral tagging and photo-z training can be done
%      (and surely is being done) with existing data.  Is FOBOS going to
%      be a huge leap, or will it mainly be cleaning up neglected corners
%      of parameter space?
%
% - More excited to hear about how the FOBOS spectra will be used for
%   science directly, instead of support for LSST

Roughly three billion years after the Big Bang ($z$$\sim$2), the
universe entered a key epoch in which proto-galaxies transitioned from
turbulent, gas-rich systems into the more ordered, star-dominated
structures that populate the universe today.  This period marks the
cosmic peak of star formation and galaxy assembly.   A full physical
picture of this epoch must include studies of the entire galaxy
``ecosystem,'' including the galaxies themselves as well as their
gas-filled environments.  The goal is to build a comprehensive picture
of the physical processes that fuel proto-galaxy growth, shape their
internal structure, and influence their environment.

Although machine-learning applied to the dark-energy equation-of-state
emphasizes the power of photo-$z$ training, detailed physical parameters
can also be inferred from combining multiple imaging datasets (e.g.,
LSST, Euclid, WFIRST) and/or lower quality spectra with well designed,
high-S/N training sets (cf., Fig.~\ref{fig:absorber}).  A central theme
of the following data-science challenges is to build SDSS-like
statistics for galaxies at this key cosmic epoch.  In all of these
challenges, we will use simulated data sets to isolate uncertainties and
biases in various training-set design strategies, including the benefits
of additional imaging information like morphology and size from a wide
range of wave-bands (e.g., combining LSST, Euclid, WFIRST).  The
exercise will define requirements for FOBOS instrument performance and
the optimal training sample delivered publicly.

% LSST's panoramic imaging will detect huge numbers of galaxies at this
% epoch.  Targeted followup with FOBOS will allow us to ascribe detailed
% galaxy and environmental information from deep spectroscopic training
% samples to the much larger cosmic volumes surveyed with broad-band
% imaging.

\begin{enumerate}[rightmargin=0.2cm,leftmargin=0.2cm]
%
\chal{phot}
%
\item[] {\textsf {\large Data-Science Challenge \ref{phot}: Apply deep-
learning algorithms to infer physical properties of galaxies at
$z$$\sim$2 using using photometry.}} The range of observed spectral
types is well-constrained by broad-band imaging (Figure \ref{fig:SOM}),
suggesting a far greater potential for imaging data to reveal physical
properties with sufficient training than conventional modeling of
spectral energy distributions (SEDs) would suggest.  We will use machine
learning to deliver SDSS-like information --- star-formation histories,
stellar-population properties, dust content, inflow/outflow properties,
and stellar masses --- for millions of imaged galaxies at $z$$\sim$2.
%
\end{enumerate}

\begin{enumerate}[rightmargin=0.2cm,leftmargin=0.2cm]
%
\chal{uv}
%
\item[] {\textsf {\large Data-Science Challenge \ref{uv}: Infer stellar
and ISM indicators in UV spectra from rest-frame optical spectra}}.
There are many powerful gas and stellar spectral features just redward
of the Lyman-$\alpha$ line at 1216\AA.  By combining FOBOS UV and
existing near-IR spectroscopy at $z$$\sim$2, we can transfer physical
``labels'' determined in the rest-frame optical to spectra at UV
wavelengths, which will dramatically enhance interpretation of JWST
discoveries of the first galaxies ($z$$\sim$10) for which rest-frame UV
imaging and spectroscopy will be most accessible.  A similar application
can ascribe the escape fraction of Lyman-continuum radiation observed in
FOBOS spectra to constrain the sources responsible for ``reionization''
at $z$$\sim$6 \citep[cf.]{2018ApJ...869..123S}.

% With simulated spectral observations, we will determine the extent of
% label transfer that is possible and set requirements on training
% samples.

\end{enumerate}

\begin{enumerate}[rightmargin=0.2cm,leftmargin=0.2cm]
%
\chal{lowsnr}
%
\item[] {\textsf {\large Data-Science Challenge \ref{lowsnr}: Train
short spectroscopic exposures in combination with photometry to provide
environmental diagnostics for 1M galaxies at $z$=1--2}}.  Photometric
redshifts, while acceptable in large cosmological analyses, wash out
information about the local position of galaxies with respect to one
another.  To characterize a galaxy's local environment and identify its
neighbors requires (observationally expensive) spectroscopic redshifts
(spec-$z$s).  However, with improved photometric redshifts available
from Challenge \ref{photozs} and strong priors on spectral types
(Challenge \ref{phot}), machine-learning techniques can yield
\emph{spectroscopic} redshifts at much lower signal-to-noise than
conventional redshift measurements. Specifically, our challenge is to
develop a methodology that can measure $z$ to 300 km s$^{-1}$ accuracy
from spectra obtained in just 10 minutes with FOBOS.  This would enable
an SDSS-like environmental study of 1M galaxies at $z=1$--$2$ in just 20
nights of 10m telescope time, making it a compelling FOBOS program to
serve publicly.

\end{enumerate}

\begin{figure}[h!]
%
\vskip -0.1in
%
\includegraphics[width=\textwidth]{figs/LGplots}
%
\caption{{\it Left}: Validation of {\it The Cannon} measurements of
stellar effective temperature, $T_{\rm eff}$, and surface gravity, $\log
g$, using low-resolution LAMOST spectra (left) compared to
high-resolution APOGEE measurements
\citep[right;][]{2017ApJ...836....5H}. {\it Top-right}: Recovery of
elemental abundances from low-resolution LAMOST spectra compared to
high-resolution measurements from GALAH (Xiang et al., in prep).  {\it
Bottom-right}: The circular-speed curve of the Milky Way determined
using a data-driven model that combines stellar parameters determined
from APOGEE spectra with photometry from WISE, 2MASS, and Gaia, yielding
the most precise measurements to date \citep{2019ApJ...871..120E}.}
%
\label{fig:Cannon}
%
\end{figure}

%-----------------------------------------------------------------------
\subsubsection{Unraveling the Formation History of our Local Group of Galaxies}
\label{sec:localgroup}

Our Local Group of galaxies --- composed of the Milky Way (MW) Galaxy,
the Magellanic Clouds, the nearby Andromeda (M31) and Triangulum (M33)
Galaxies, and a multitude of satellite galaxies --- is just one
realization of the galaxy-formation process, but it is the one that we
can study in the greatest detail.  Large-scale imaging surveys, like
SDSS and Pan-STARRS, have unveiled numerous stellar streams and other
halo substructures in both the MW and M31, and we expect a hundredfold
growth in this census via the upcoming LSST and WFIRST surveys.
Follow-up spectroscopy of Local Group member stars allow us to, e.g.,
constrain the orbits of stellar streams and the present-day enclosed
mass of the galaxies they orbit \citep{2017ApJ...836..234S}, as well as
the age and chemical composition of their stellar populations
\citep{2019MNRAS.484.3425M}.  At the same time, cosmological
simulations, like IllustrisTNG \citep{2018MNRAS.475..648P}, can now
simulate the full chemo-dynamical evolution of Local-Group-like
overdensities in the Universe to which data can be meaningfully
compared.  Finally, the Gaia satellite is currently revolutionizing our
understanding of the MW by providing distances and on-sky motions for
more than a billion stars spanning the full extent of its disk.  This
simultaneous maturation of both the theoretical and observational data
will allow us to form physically motivated models for the formation
history of the Local Group and its constituents.

Programs to measure the radial velocities of stars in the MW halo or the
M31 disk using DEIMOS at Keck require observations of up to 10 hours,
depending on the population being probed \citep{2012ApJ...752...45T,
2018arXiv180904082C}.  Such long integration times is one motivatation
for FOBOS's push to maximize the number of targets observed in a single
pointing.  However, one can also appeal to machine-learning algorithms
to infer the relevant physical quantities statistically from both
multi-band imaging and lower quality spectra (low resolution and S/N)
using a relatively small, yet high-S/N, training set.  For example,
\citet{2015ApJ...808...16N} have developed {\it The Cannon}, a
supervised learning algorithm that uses spectra with known stellar
parameters to label spectra where those parameters are unknown
(Fig.~\ref{fig:Cannon}).  Additionally, \citet{2018arXiv180401530T} have
developed {\it The Payne}, which they show can determine 16
stellar-abundance labels from low-resolution spectra using a neural
network and theoretical stellar spectra.  Finally,
\citet{2018arXiv180803278T} have combined Kepler-based astroseismology
measurements with APOGEE spectra to determine stellar age to $\sim$25\%
precision using a neural network.  Our proposed effort builds on new
lines of inquiry based on these successes.

\begin{enumerate}[rightmargin=0.2cm,leftmargin=0.2cm]

\chal{mwhalo} 
%
\item[] {\textsf {\large  Data-Science Challenge \ref{mwhalo}: The
chemical evolution and assembly history of the MW stellar halo.}} LSST
and WFIRST will reveal a trove of substructure in both the MW and M31
halo.  We will design a FOBOS program to observe main-sequence turn-off
and red-giant stars in these substructures within the MW that also
leverages existing data from, e.g., APOGEE and H3.  We will build
data-driven models based on these data to measure stellar parameters
(temperature, surface gravity, metallicity, and alpha-element abundance)
for all halo stars with LSST+2MASS+WISE+WFIRST multi-band photometry,
allowing us to reconstruct the star-formation history of each disrupted
satellite. These will be combined with dynamical data and compared with
cosmological simulations to build a generative model for the assembly
history of the MW stellar halo.

\chal{m31} 
%
\item[] {\textsf {\large Data-Science Challenge \ref{m31}: The
differential chemical evolution of M31 and MW.}}  A natural extension of
Data-Science Challenge \ref{m31} is to perform the same analysis for the
halo of M31.  However, we cannot expect to obtain high-quality spectra
of individual main-sequence stars at the distance of M31 with FOBOS.
Moreover, training a chemical evolution model using spectra of Milky Way
stars may lead to systematic errors:  The Milky Way and Andromeda have
distinct evolutionary histories \citep[e.g.][]{2005MNRAS.356.1071R},
despite being relatively similar in many other respects.  We will
therefore obtain deep observations of giant stars in the M31 halo to
drive a machine-learning algorithm that combines a model of the MW halo
with results from cosmological hydrodynamical simulations to constrain
the differential history of the MW and M31 stellar halos.

\chal{gaia} 
%
\item[] {\textsf {\large Data-Science Challenge \ref{gaia}: Stellar
parameter determinations for a billion stellar spectra.}} While
providing on-sky motions and photometry for 1.7 billion stars in the MW,
fewer than 10\%, 0.3\%, and 0.1\% of stars will have a full complement
of astrometrics and kinematics, basic stellar parameters, and chemical
abundances, respectively.  Moreover, Gaia distance errors increase
quadratically with distance.  To realize Gaia's full potential, we will
design FOBOS training sets that, when combined with high-resolution
datasets from, e.g., APOGEE, WEAVE, will allow us to build data-driven
models of the absolute magnitude (yielding distance modulus),
temperature, surface-gravity, and stellar abundance for {\it all} stars
in the Gaia dataset.  These data will allow us to isolate coeval
populations in the Galactic disk that can be combined with very
high-resolution simulations of the Milky Way to provide a detailed
evolutionary history of our Galactic home.

\end{enumerate}

%-----------------------------------------------------------------------
%-----------------------------------------------------------------------
\section{Project Implementation}
\label{sec:project}

This proposal involves three coordinated activities: 1) Organizing and evaluating the results of a community-wide
effort to address simulated Data Science Challenges; 2) Completing Preliminary Design for the FOBOS
instrumentation, informed in part by refining requirements as a result of (1); 3) Designing the operational modes,
planning tools, data analysis software, and serving platforms necessary for delivery of public training sets.
Anticipating significant progress in all three activities, we will request NSF MSRI-2 funding in 2021 to build and
deploy FOBOS at the telescope, carry out required observations, and publicly serve the data products.  FOBOS would
see first light in 2027 and carry a total cost of \$32M (without contingency in 2019 dollars).  While we focus the
current request on work required for the Preliminary Design Phase, we outline the overall project plan and final
deliverables in order to motivate this work.

\subsection{FOBOS Instrument Concept}
\label{sec:concept}
% \noindent \comment{1 page}

% Here's an alternative way to put in figures if we want captions on the side (to save space)
% Could introduce a new ``counter'' to count and label figures appropriately
%\centerline{\hbox{\includegraphics[width=0.6\textwidth, angle=0]{figs/FOBOSatKeck_v1.pdf}
%    \hspace{0.1cm} \vspace{2in}
%    \parbox[b]{0.3\textwidth}{\small {\bf Figure ??:} Rendering of FOBOS instrument systems deployed at the Keck II Nasmyth port.  By mounting the FOBOS spectrographs under the Nasmyth platform, other instruments like DEIMOS can maintain access to the telescope. \vspace{2cm}}}}



 \begin{figure}[h!]
  \vskip -0.1in
  \includegraphics[width=\textwidth]{figs/FOBOS_inst.pdf} %atKeck_v1.pdf}
  \caption{\small Rendering of FOBOS instrument systems deployed at the Keck II Nasmyth port.  By mounting the FOBOS spectrographs under the Nasmyth platform, other instruments like DEIMOS can maintain access to the telescope.}\label{fig:layout}
 \end{figure}

Mounted at the Nasmyth focus of Keck II Telescope at WMKO, FOBOS will be one of
the most powerful spectroscopic facilities in the next decade.  FOBOS consists of several key components (Fig
\ref{fig:layout}).  A compensating lateral atmospheric dispersion corrector (CLADC, not pictured) ensures that target
light from all wavelengths falls on allocated fibers while also correcting image aberrations at the edges of the 20
arcmin diameter Keck field.  Each of the CLADC lenses is 946 mm in diameter, the first two closely spaced with lateral
relative motions enabled by three barrel-mounted actuators.  The final CLADC lens surface serves as the vertical
mounting plate for roaming Starbugs fiber positioners.  It translates to track focal plane tilt.  Starbugs patrol a
large on-sky area ($\sim$1 arcmin), enabling flexible and dynamic targeting configurations with adjacent fibers as
close as 10 arcsec.

A total of 1800 150 $\mu$m core diameter fibers are deployed at the curved focal plane, which rotates and translates to
maintain image positions as the telescope tracks across the sky.  The fiber run is kept at less than 10 m to
maintain high throughput at UV wavelengths, and special care is given to stress-relief cabling to minimize variable
focal ratio degradation over the fiber run.

Sets of 600 fibers feed each of three identical spectrographs.  Each spectrograph uses a series of dichroics to divide
the input light into four wavelength channels with combined coverage from 310 to 1000 nm and mid-channel spectral
resolutions of $R \sim 3500$ \comment{justify this?}.  The dispersed light in each channel is focused by an f/1.1 catadioptric camera and
recorded by an on-axis 4k$\times$4k CCD mounted at the center of the first camera lens element.  Spectrographs are
mounted in a temperature controlled housing installed under the Nasmyth Deck to allow space for other Keck instruments
above.  The end-to-end instrument throughput is greater than 30\% at all wavelengths.

FOBOS includes observatory level systems for precise instrument calibration using dome-interior screen illumination, a
metrology system for accurate fiber positioning, and guide cameras for field acquisition and guiding.  The instrument
design envisions future upgrades including alternate collecting modes that deploy multiple fiber bundles, feeds to
other fiber-based spectrographs at different wavelengths or spectral resolutions, and the ability to support and
benefit from image corrections with Ground-Layer Adaptive Optics.



\subsection{FOBOS Instrument Design Effort}
\label{sec:design}
% \noindent \comment{1 page}

FOBOS will complete its current conceptual design phase in October 2019.  Funding from this proposal will support preliminary design beginning in November 2019.  A schedule of milestones is attached and more information provided in the Project Execution Plan (PEP).  Major components of the preliminary design effort are described below.

\noindent \textbf{Atmospheric Dispersion Compensator (ADC).} The opto-mechanical design, tolerancing, lens cell design, motion systems, and software controls design of the ADC will be completed.  

\noindent \textbf{Focal Plane System.} The final ADC lens element serves as the focal plane mounting plate for the fiber positioners.  This focal plane system must rotate and translate to track the field and refraction angles from the ADC.  Mechanical design, including flexure analysis and the selection of drive mechanisms and potential vendors will be completed.  This system also defines one of the interfaces to the Keck Telescope and must comply with Keck Observatory space envelopes, servicing needs, and other requirements.  The focal plane system also interfaces with guide cameras for field acquisition and guiding.

\noindent \textbf{Starbugs fiber positionsers.} Starbugs are a positioning technology developed and deployed by the
Australian Astronomical Observatory (AAO) which has partnered with our team to generate a conceptual design for
Starbugs in the context of FOBOS.  Design requirements for Starbugs in FOBOS are more relaxed than the currently on-sky
TAIPAN instrument thanks to the larger physical plate scale at Keck.  AAO will serve as a vendor during preliminary
design but is interested in exploring a partnership and in-kind contribution model in the construction phase.  In addition to the Starbugs themselves, a fiber metrology system (for accurate closed-loop positioning) will also be developed.

\noindent \textbf{Fiber System.} We will complete the optical design and processing plan for affixing forward optics
lenses to each fiber's head (these demagnify and speed up the beam for proper fiber coupling).  A micro-lens array
solution will be developed for a central, fixed-position 4.5-arcsec diameter IFU for fast source acquisition. This
workpackage also includes the stress-relief cable system and fiber termination hardware and processing.

\noindent \textbf{Spectrographs.} The optical systems and components (slit, collimator, dichroics, gratings, and camera), an analysis of acceptable tolerances and performance, their mechanical supports, software controls, and the overall enclosure will all be advanced through preliminary design.  Detectors, cryostats, read-out electronics and systems for thermal management will be designed.

% \noindent \textbf{Calibration System.} This package includes design of an interior dome screen and projection system for injecting calibration sources with sufficient spatial uniformity and stability into the instrument.  We will work with the Observatory to develop an integration and controls plan.  No such calibration system currently exists at Keck.

% \noindent \textbf{Auxiliary Systems.} Design of auxiliary systems includes Nasmyth platform interfaces, utilities access, fiber routing and support, thermal control and vibration control systems.


\subsection{Addressing Data Science Challenges and Designing FOBOS Training Sets}
\label{sec:survey}
% \noindent \comment{1 page}

Our team includes leading experts on data science applications to astronomy and LSST specifically.  We will also use
our established connections to LSST's Informatics and Statistics Science Collaboration (ISSC) to advertise, recruit,
and coordinate efforts to tackle the Data Science Challenges described in Section \ref{sec:goals}.  Our proposal
request includes two open workshops to motivate progress and discuss results. At the end of the proposal period, we
will publish the results and developed software packages.

The Data Science Challenges require work on simulated imaging$+$spectroscopic data sets where input physical properties
(e.g., redshift) can be imposed and the output, recovered values compared against the input.  Simulated imaging data
(e.g., from LSST and WFIRST) are in-hand, while mock spectroscopy will be provided by a FOBOS instrument simulator,
an initial version of which has already been developed.  Further advances to be supported by this proposal include
improved error modeling and simulating systematic effects from detector artifacts, image quality aberrations informed
by the emerging detailed optical design, and variable observing conditions.

The resulting success in addressing each Data Science Challenge will define a level of readiness and set requirements
on the associated FOBOS training sets required, including number of sources, pointings, magnitude limits,
signal-to-noise thresholds, and observing conditions.  Preliminary observing design and a description of required
operational modes to efficiently observe these training sets will begin with this proposal.  Operational modes will set
requirements on target aggregation and prioritization systems, field acquisition speed, field rotation range, zenith
avoidance zone, reconfiguration time, calibrations, read-out time, quicklook reduction software and processing rates.
We will develop integrated program concepts that efficiently combine required observations.  Detailed survey and
execution plans will be completed in the next phase of this project (MSRI-2).  Roughly 20\% of Keck observing time is
open to the public, and as in previous federally-funded projects, we fully expect that Senior Personnel at Keck
institutions will be successful in collaborative efforts to secure significant amounts of additional telescope
observing time to enable rapid, publicly release of training data with any proprietary period waived
\citep[e.g.,][]{newman13}.

% The complete photo-$z$ training survey described in \citet{newman15} would
% require 15 independent pointings, each spanning 0.1 deg$^2$ with a target density of 6 arcmin$^{-2}$ (8 arcmin$^{-2}$
% when including $z > 1.5$ galaxies accessible in the UV with Keck-FOBOS), perfectly matched to the Keck-FOBOS
% field-of-view and target density.  With a conservative exposure time of 100 hours to reach 75\% redshift completeness
% for 40,000 galaxies with $i_{\rm AB} < 25.3$, the Neman survey would require 400 nights.  Challenge \ref{photoz} would
% reduce the required survey duration by a factor of at least four.  Meanwhile the extreme depths and flux-limited
% selection are likely also requirements for training sets associated with Challenges \ref{phot}, \ref{uv}.

% A wider and shallower survey component is envisioned for Challenges \ref{lowsnr} and \ref{gaia}.  With 10-minute
% integrations, a 52 deg$^2$ Keck-FOBOS sample of environmental diagnostics for 1 million galaxies could be carried out
% in less than 20 nights.  This program would sample at $z \sim 1.5$ the same cosmic volume as SDSS.  A program of a
% similar scale would provide training set data for inference of stellar parameters in the Milky Way.  These shallow programs would be integrated with the deeper components described above into a single survey plan.



% The components of the FPS are as follows:
%     - Photo-z training samples (Newman, Masters)
%         - z>1.5 in the blue
%         - z<1.5 in the red
%     - Ly continuum
%         - z~2 in the blue
%         - z~7 in the red
%     - Machine-learning the SDSS but at z~2


\subsection{MAISTRO: Target Allocation with Artificial Intelligence}
\label{sec:targeting}
% \noindent \comment{1/2 page}

Powered by Starbugs fiber positioners, FOBOS will enable fast, dynamic reallocation of fibers.  To efficiently
determine the best options given a wide range of possible targets and desired observing outcomes, we will develop a
preliminary design for MAISTRO\footnote{MAISTRO: Modular Artificial Intelligence System for Target Reallocation and
Observing.} an ``artificial intelligence'' (AI) targeting system that will learn optimization strategies for assigning
targets from a database of overlapping observing programs with pre-defined priorities.  The AI package will aggregate
data quality using a quick-look reduction package, science-driven performance metrics, {\it and real-time assessments
of the observing conditions} to make dynamic targeting recommendations.  For example, if conditions are
slightly less than optimal, MAISTRO would reconfigure Starbugs to brighter objects in a field or implement a different program prioritization.  MAISTRO would incorporate updated target lists and priorities from the active observer and could easily be over-ridden at any time.   Fractions of the full FOBOS multiplex might also be reserved ``manual targeting'' as required by the P.I.  

%   - maintains a database with observational progress on individual
%     targets in the survey and
%   - dynamically reallocates fibers based on real-time assessments of
%     the aggregate S/N of each target to meet the specific need of each
%     science case.

% This requires significant design and testing of a combined software
% package and hardware interface.  Specific considerations involve (1)
% fast and robust reduction procedures (cf. MaNGA DOS) that can assess
% the aggregate data and (2) a responsive database with a schema
% optimized for real-time decision making to select targets for
% (re)acquisition while accounting for collision limitations.  Provided
% enough design effort, this lends itself to a machine-learning
% application.

\subsection{Publicly Available Automated Data Products}
\label{sec:DAP}
% \noindent \comment{1/2 page}

The typical proprietary period for raw data acquired at Keck is 18
months.  However, typical of other public surveys (e.g., SDSS), this
period will be shortened to one year for the \comment{FOBOS Public Survey}.

Both as part of our design effort and for long-term use, we will develop
a data-reduction pipeline, building on work already done for other
fiber-based observations, like SDSS and DESI.  This software will
provide both the quick reduction assessments needed for our dynamic
targeting system and the more detailed reduction to produce the data for
scientific analysis.  Reduced data will be delivered to the community
(e.g., via the Keck Observatory Archive) after the proprietary periods
are finished for {\it both} PI-led and \comment{public survey} observations.

Finally, we will also provide a data-analysis pipeline that provides
high-level data products.  There are two aspects to analysis of the
data.  First, we will provide software to perform the traditional
measurements of properties like Doppler shift, emission-line strengths,
and internal kinematics that are measured from FOBOS-observed spectra.
This software will build on existing software we have built for the
SDSS-IV MaNGA survey (Westfall et al.), and it will be executed for {\it
any} data taken with FOBOS and released along with the reduced spectra.
This is a substantial effort and unheard of for observations taken
outside of a large-scale survey effort.  Second, we will provide the
results of our various machine-learning applications in the \comment{FOBOS Public
Survey} (e.g., the LSST-source redshifts as determined by the FOBOS
observed training set).

Important to the success of both the data-reduction and data-analysis
software will be PI and community involvement in their refinement to
meet the needs of specific science applications.  These software
packages will be open source and publicly served (e.g., using GitHub).

% Beyond the raw data, the survey will provide reduced and derived
% products immediately (cf. MaNGA DAP).  The latter will be true of both
% the data from the public survey (released immediately) and indeed {\it
% any} data taken with the FOBOS instrument after the nominal 18-month
% proprietary period.  For the latter, we will encourage involvement of
% the program PI in refining the data-reduction and data-analysis
% software and its execution to garner the most from its application to
% their data.  Community involvement in a common software development
% obviates the need for different groups to retread old ground.

\section{Broader Impacts}
\label{sec:bi}

% "include a discussion of student training, increased participation of
% underrepresented groups and a description of tangible benefits to the
% wider U.S. research community (access, data products, technology,
% etc.)."

\subsection{Akamai: Training the next generation of Hawaiian STEM
professionals} Led by the Institute for Scientist and Engineer Educators
(ISEE) at UCSC, the Akamai Internship Program is aimed at advancing
college students from Hawaii into the STEM workforce.  Almost 400
students have participated to date, of which 24\% are Native Hawaiian
and 38\% are women. A longitudinal study of Akamai outcomes indicated
that 87\% were still in STEM, either in the workforce or continuing STEM
studies \citep{asee_peer_31030}.  Traditionally, most Akamai interns are
pursuing engineering or computer science in their undergraduate
education.  ISEE and the Akamai program already have deep connections to
the W.~M.~Keck Observatory, to-date involving 45 interns in projects
related to instrument development and observatory operations over the
past 15 years.  We will fund two Akamai interns during our funding
period.  Specific opportunities for their involvement include assessment
of the FOBOS instrument design via a software-based simulator that will
eventually result in a sensitivity calculator and the data-reduction
pipeline, as well as the machine-learning applications discussed
throughout our proposal.  At UCO, we have already worked with an Akamai
intern, who helped us setup a fiber test-bench at UCSC during Summer
2018. We are excited by the opportunity to include funding for these
interns as part of our proposal and continue to seek new opportunities
to generate connections with the current and future Hawaiian workforce.

\subsection{Investing in future educators} Also via the ISEE, we will
support three graduate students to participate in the Professional
Development Programs (PDP) that trains graduate students and supports
them in developing their teaching skills.  The PDP trains graduate
students and postdocs to collaboratively design a well-focused inquiry
activity within a small team.  The activity is conceived, developed, and
tested within the team, and the program culminates in the team running
the lab exercise with a group of undergraduates.  The program emphasizes
inclusive and equitable learning environments.  Specifically, our team
of graduate students will develop an inquiry based lab unit on
instrument development, data-reduction procedures, and/or
machine-learning methods, which will be taught to incoming transfer
students from community colleges.  In addition to training three
graduate students in inclusive pedagogy, our team will impact 25
undergraduates coming from California community colleges, a large
fraction from underrepresented minority groups.

\subsection{Undergraduate Student Training} Multiple avenues exist
within the current curriculum to include UC undergraduate students in
the development of FOBOS and its \comment{public survey}.  At UCSC in particular,
we will guide freshman and first year transfer students through two
quarters in Astro 9, a recently started course that aims to introduce
scientific research method early in students' tenure via timely research
projects developed and led by UCSC graduate students, postdocs, and
staff.  Both PI Bundy and co-PI Westfall have been involved in projects
over the past two years, including a project to measure the rotation
curves of galaxies observed by the SDSS-IV MaNGA survey.  Introducing
undergraduates to astronomical instrumentation would be a unique
contribution to this course.
\comment{Make a note about sip here}.

% "Preliminary proposals must include an outline of ongoing operations and
% maintenance plans, including an estimate of any needs for ongoing,
% NSF-supported operations and maintenance that may be requested outside
% of the Mid-scale RI program."

% "Results from Prior NSF Support should not be included. Also, links to
% URLs may not be used."

% \vspace{-0.5cm}

% {\bf no more than 2 pages for references}

\clearpage
\setcounter{page}{1}
\bibliographystyle{nsf}
\bibliography{references}


\newpage
\setcounter{page}{1}

% \noindent{\bf Budget and Budget Justification}

% "including budgets for any subawards. For preliminary proposals cost
% estimates may be preliminary estimates with the Basis of Estimates (BoE)
% included. Copies of vendor quotations should not be included in
% preliminary proposals. If the budget includes contingency, that
% contingency should cover the "known unknowns" and be used to mitigate
% identified risks."

% \newpage

\section{Facilities, Equipment, and Other Resources}

% In order for NSF, and its reviewers, to assess the scope of a proposed
% project, all organizational resources necessary for, and available to a
% project, must be described in this section of the proposal. Proposers
% should describe only those resources that are directly applicable. The
% description should be narrative in nature and must not include any
% quantifiable financial information. Proposers should include a
% description of the internal and external resources (both physical and
% personnel) that are expected to be available to the project.  Such
% information must be provided in this section, in lieu of other parts of
% the proposal (e.g., Budget Justification, Project Description).

\begin{figure}[h!]
 \vskip -0.1in
 \includegraphics[width=\textwidth]{figs/fig_test_bench.png}
 \caption{\small The UCO Fiber Test Bench is being used to prototype fiber-lenslet coupling options and ensure high-throughput coupling at the Keck Focal plane.  The top panel shows a schematic diagram of the test bench, which allows for variable input broadband sources to be directly compared to output near- and far-field images generated after the input light passes through fiber assemblies.  The bottom panel is a photograph of the working test stand.   }\label{fig:testbench}
\end{figure}

\subsection{University of California Observatories}

University of California Observatories (UCO) manages a world-renown facility on the UC Santa Cruz campus for the
design, construction, and testing of astronomical instrumentation.  With a staff of leading optical designers,
engineers, and instrument scientists, UCO has a long heritage of producing state-of-the-art instrumentation, including
many spectrometers, as well as controls software for the Lick and Keck Observatories.  The recent delivery of K1DM3\footnote{K1DM3: Keck 1 Deployable Tertiary Mirror.} illustrates the close relationship between WMKO and UCO which allows us to leverage detailed knowledge of the observatory structure, protocols, interfaces, software and systems requirements, and instrument deliverables.

\subsection{UCO Fiber Test Bench}

UCO has built a precision fiber test bench (Figure \ref{fig:testbench}) which is currently being used to prototype lenslet coupling solutions and procedures and will serve as a valuable testing tool in the FOBOS preliminary design, especially for measuring the impact of fiber stress and motion on the focal ratio degradation and throughput variability.

\subsection{W.M.\ Keck Observatory}

WMKO has provided funding as well as technical guidance for initial stages of FOBOS development and its interface to the observatory.  The underside of the Keck Nasmyth deck (Figure \ref{fig:nasmyth_mount}) has been identified as the mounting point for FOBOS spectrographs, maintaining access space for other Keck instruments above.  Figure \ref{fig:keck_exchange} shows how railings allow instrument components, like the FOBOS focal plane system, to wheel up to the Nasmyth port or to wheel back in stow positions when not in use.


\begin{figure}[h!]
 \vskip -0.1in
 \includegraphics[width=\textwidth]{figs/nasmyth_deck.png}
 \caption{\small The underside of the Keck Nasmyth deck where the FOBOS spectrographs will be mounted, with fiber runs feeding from the focal plane system above.  Other Keck instrument facilities (e.g., components of the adaptive optics laser system) have been successfully mounted in this location.  }\label{fig:nasmyth_mount}
\end{figure}

\begin{figure}[h!]
 \vskip -0.1in
 \includegraphics[width=\textwidth]{figs/Keck_instrument_exchange.png}
 \caption{\small The FOBOS focal plane system would be mounted on railings to allow access to the Nasmyth port when its being used.  As with other instruments, it can be removed to a stow position on the extended service platform.  The photo on the left shows the Keck Cosmic Web Imager (KCWI) in the Nasmyth mounted position.  On the right, the Echellette Spectrograph and Imager (ESI) is sitting in its stow position. }\label{fig:keck_exchange}
\end{figure}

\subsection{Starbugs at the Australian Astronomical Observatory} % (fold)
\label{sec:AAO}

The Australian Astronomical Observatory (AAO) has worked with the FOBOS team during conceptual design to develop
designs for the CLADC and confirm the feasibility of Starbugs at the FOBOS focal plane.  The Starbugs fiber
positioners have been under development at AAO for nearly 15 years \citep[see][]{staszak16} and have recently been
deployed on-sky on a new instrument called TAIPAN.  Beginning science operations this year, TAIPAN
demonstrates that the level of maturity attained by the Starbugs technology makes it highly attractive to an instrument
like FOBOS beginning its preliminary design phase.

\begin{figure}[h!]
 \vskip -0.1in
 \includegraphics[width=\textwidth]{figs/starbugs.png}
 \caption{\small Starbugs fiber positioning technology from AAO.  {\it Left:} An individual Starbug assembly where the concentric piezo cylinders (colored red) which establish a vacuum seal to the focal plane plate are evident with the central fiber payload housing (white).  The outer diameter is 8 mm.  A connector directs the fiber output to the next section of the cable run. {\it Right:} The picture shows the front face of several Starbugs attached to the TAIPAN focal plane.  Three green laser dots provide closed-loop feedback on each Starbug's position.  Both figures are from \citet{staszak16.} }\label{fig:starbugs}
\end{figure}

\subsection{CMU Machine Learning Department}
The Carnegie Mellon University Machine Learning department is one of the leading institutes in the field of Machine Learning research. We intend to collaborate with Prof. Barnab{\'a}s P{\'o}czos, a internationally recognized expert in the field of Bayesian Optimization and Machine Learning, to optimize the follow-up strategy and the science output. The authors Rachel Mandelbaum and Markus Michael Rau are long time collaborators of the CMU Machine Learning department and particularly of Prof. P{\'o}czos, authoring numerous joint publications and proposals. The suvey optimization will be performed with the help of the readily available software package Dragonfly\footnote{\url{https://github.com/dragonfly/dragonfly/}} that was developed by the group of Prof. P{\'o}czos.


% subsection starbugs_at_the_australian_astronomical_obseravatory (end)

\subsection{FOBOS Conceptual Design}

FOBOS conceptual design work has proceeded in parallel with a fiber-based, Nasmyth-mounted instrument concept for the Thirty Meter Telescope (TMT) that shared many design elements in common.  A variety of related documents and reports are being produced in support of this effort:

\begin{itemize}
	\item Science Case Requirements Document: A funded effort to gather community feedback, science cases, and associated requirements has yielded the design specifications for FOBOS.  A report will be issued in summer 2019.

	\item Draft manuscript detailing requirements and demonstrating necessary performance based on existing instrumentation for faint-object fiber spectroscopy on large-aperture telescopes.

	\item Design specification documents and optical (Zemax) models for the CLADC, lenslet fiber fore-optics, spectrograph layout and cameras.

	\item Vendor quote estimates for the CLADC, Starbugs positioners, fiber run, dichroics, FSE gratings, catadioptric cameras, and detector systems.

	\item SolidWorks modeling of the focal plane system is nearly finished with spectrograph support and enclosure structures underway.  Likely vendors of focal plane motion systems identified.

	\item Detailed Microsoft Project plans through Preliminary Design and out to instrument deployment.  

	\item Risk matrix capturing 48 risk items and scoring their probability and consequence severity before and after mitigation strategies.

\end{itemize}

\end{document}

% (See the LFM/MFG. Greater detail will be required in invited full
% proposals should that occur. See Full Proposal Preparation section for
% further information.)





