
% Outline: 28 Jan 2019
%   - 0. Summary (PAPPG guidlines) - 1p
%   - 1. Project Description - 10 pp
%   - 1.1 Intellectual Merit:
%       - 1.1.1 Background \& Motivation
%           - LSST cost NSF ~ \$1B
%               - LSST Science goals
%           - Can think of LSST as the next generation SDSS
%               - But it doesn't have spectroscopy!
%           - SDSS was an Imaging + Spectroscopy survey
%               - SDSS the most cited survey ever
%               - SDSS science milestones
%               - Spectroscopy was a key aspect for the success of SDSS
%                 (MPA-JHU)
%               - Public release of data also key
%           - What would it look like to perform SDSS-like spectroscopy
%             for the LSST imagine survey?
%               - Fraction of observed objects
%               - Cost
%               - currently an intractible problem
%                   - link to photonics development?
%               - Need new techniques (machine learning) to infer
%                 statistically accurate population characteristics with
%                 small training sets
%           - LSST Science themes
%           - Science enabled by LSST spectroscopy, lay groundwork for
%             later science discussion.
%
%       - 1.1.2 Research Community Priority
%           - Review Astro2010
%               - LSST
%           - Preview of or predictions for astro2020
%           - Big data initiatives:
%               - Moore-Sloan Data-Science Environments
%                   - Berkeley institute for data science, NYU, UW
%               - Using subsets of training data to understand a
%                 population
%                   - Hogg's AAS talk...
%
%           - Spectroscopy of LSST sources to get photozs provides key
%             advantages to Cosmology/Dark Energy
%           - Time domain astrophysics
%               - AGN, CVs, stars (astroseismology)
%
%       - 1.1.3 Astrophysics and Data Science Goals
%           - A Universe in transition
%               - Gravity to Dark Energy:
%                   - Precise cosmology with LSST and trained photozs
%               - Gas to stellar dominated: The Galactic Ecosystem at z~2
%                   - environment (Cooper)
%                   - absorption-line systems (Prochaska)
%                   - stellar populations (Shapely, Siana)
%                       - Bootstrap z~2 properties to higher redshift (z~7)
%               - Neutral to ionized gas: The heterogeneity of
%                 Reionization
%                   - environment and neutral fractions (Becker)
%                   - Ly continuum (photon budget) at z~5-7 (Siana)
%           - The Chemodynamical history of the Local Group
%               - Chemical tagging of GAIA spectra on the far side of
%                 the Milky Way (Yuan-Sen)
%                   - What's gained in 0.31-0.4 range (unavailable to
%                     PFS)
%               - MW/M31/M33 (Raja, Connie)
%               - Life cycle of stellar clusters
%           - Time Domain (Foley)
%
%       - 1.1.4 FOBOS
%           - How does FOBOS meet the science goals?
%           - Instrument description as of conceptual design
%           - Development from concept to final design
%               - Sketch of timeline
%           - Public survey design
%               - Key science programs
%               - AI-informed targeting system
%               - Robust DRP and DAP
%               - Public release schedule
%           - Longer-term strategy for public release of *any* FOBOS
%             data
%               - 18-month proprietary period for PI data
%               - Observatory queue for opportunistic use of "free"
%                 fibers
%
%       - 1.2 Broader Impacts
%           - Student Training
%           - ISEE, AKAMAI (Lisa Hunter)
%           - New UCSC Astrophysics major with Data Science emphasis
%               - New course, summer projects
%   
%   - 2. References (2-page max)
%
%   - 3. Biographical Sketches (2 pages each), required for PI, co-PIs,
%     and any additional senior personnel (see PAPPG)
%
%   - 4. Budget and Budget Justification: "For preliminary proposals
%     cost estimates may be preliminary estimates with the Basis of
%     Estimates (BoE) included. Copies of vendor quotations should not
%     be included in preliminary proposals. If the budget includes
%     contingency, that contingency should cover the 'known unknowns'
%     and be used to mitigate identified risks."
%
%   - 5. Facilities, Equipment, and Other Resources: "In order for NSF,
%     and its reviewers, to assess the scope of a proposed project, all
%     organizational resources necessary for, and available to a
%     project, must be described in this section of the proposal.
%     Proposers should describe only those resources that are directly
%     applicable. The description should be narrative in nature and must
%     not include any quantifiable financial information. Proposers
%     should include a description of the internal and external
%     resources (both physical and personnel) that are expected to be
%     available to the project.  Such information must be provided in
%     this section, in lieu of other parts of the proposal (e.g., Budget
%     Justification, Project Description)."
%
%   - 6. Supplementary Documents:
%           - A list of the major team members, their affiliations, and
%             their role in the project
%           - A list of Partner Organizations to be funded via
%             subawards, and the role of each in the project
%           - An outline of the Project Execution Plan (PEP) . See the
%             LFM/MFG. Greater detail will be required in invited full
%             proposals should that occur.
%
%         "No other items or appendices should be included. Information
%         pertaining to 'Results from Prior NSF Support', 'Current and
%         Pending Support', 'Data Management Plan', and 'Postdoctoral
%         Mentoring Plan' is not required for preliminary proposals and
%         should not be included. Preliminary proposals containing items
%         other than those required above will be returned without
%         review.


%\documentclass[11pt,letterpaper]{article}
\documentclass[oneside,11pt]{amsart}

%\usepackage{a4wide}
%\usepackage{epsfig}
%\usepackage{psfig}
\usepackage{graphicx}
\usepackage{natbib,latexsym,url,enumitem,pdfpages}
\usepackage{color}
\usepackage{wrapfig}

% Some fancy commenting
\definecolor{todo}{RGB}{200,0,0}
\newcommand{\comment}[2][todo]{{\color{#1}[[{\bf #2}]]}}

% Challenge counter
\newcounter{chalno}
\newcommand{\chal}[1]{\refstepcounter{chalno}\label{#1}}

% User commands
\input{journaldefs}

\DeclareRobustCommand{\gtrsim}{%
\mathrel{\hskip-.5em\begin{array}{c}>\\[-8pt]\sim\end{array}\hskip-.5em}}
\DeclareRobustCommand{\lesssim}{%
\mathrel{\hskip-.5em\begin{array}{c}<\\[-8pt]\sim\end{array}\hskip-.5em}}

\pretolerance=10000
\textwidth=6.4in
\textheight=8.95in
\voffset = 0.in
%\voffset = -0.3in  % For my printer
\topmargin=0.0in
\headheight=0.00in
\hoffset = 0.0in
%\hoffset = 0.33in  %  For my printer
\headsep=0.00in
\oddsidemargin=0in
\evensidemargin=0in
\parindent=2em
\parskip=0.2ex
 
\renewcommand{\baselinestretch}{1.03}

\special{papersize=8.5in,11in}

\newcommand{\markus}{\textcolor{green}}

\setlength{\parskip}{0.6 ex plus 0.4ex minus 0.2ex} \flushbottom
\pagestyle{plain} 

\begin{document}
% \thispagestyle{empty}

\pagenumbering{arabic}
\begin{center}

\vspace*{-1.5cm}

%\textbf{\textsf{Training LSST with Keck-FOBOS:\\ Comprehensive, Data-Driven Models of a Universe in Transition}}
\textbf{\textsf{Training for Big Astronomical Data with Keck-FOBOS:\\ Comprehensive, Data-Driven Models of a Universe in Transition}}

\authors{K. Bundy, K. Westfall, N. MacDonald, P. Capak, A. Coil, C. Conroy, M. Cooper, R. Kupke, K.G. Lee, R.
Mandelbaum, D. Masters, J. A. Newman, X. Prochaska, C. Rockosi, J. Rhodes, M. Rich, M. Savage, A. Shapley, B. Siana, Y.-S.
Ting, G. Wilson, Markus Michael Rau}

\end{center}

% \noindent\begin{center}\mbox{\parbox{0.9\linewidth}{
% %
% This Mid-scale Research Infrastructure-1 ``Design'' proposal requests
% funds to complete the preliminary instrument design for the Fiber-Optic
% Broadband Optical Spectrograph (FOBOS) and build frameworks for enabling
% data-driven science goals via a FOBOS Public Survey.}}
% %
% \end{center}

% \smallskip
% \section{Project Summary}
% \label{sec:summary}

% \noindent\comment{1 pg, doesn't count towards 10-pg limit}

% \noindent {\bf Overview:}
% (activity that would result and methods)

% \noindent {\bf Intellectural Merit:}

% \noindent {\bf Broader Impacts:}

% \clearpage


\noindent\comment{10-page limit, excluding references.}

\smallskip
\section{Intellectual Merit}
\label{sec:im}

\subsection{Scientific Justification} 
% \noindent \comment{3/4 page}

% "including the unique research
% capabilities and lack of general availability of the requested
% infrastructure and its potential to significantly advance the Nation’s
% research infrastructure."


Led by NSF's Large Synoptic Survey Telescope (LSST\footnote{LSST will be begin science operations in 2023.}), astronomy is entering a new era of unprecedented deep-imaging data
sets that will survey huge volumes of the universe when it was only one-half or one-third its current age ($z \sim
$1--3).  These epochs mark important but poorly understood transitions in cosmic history. Early galaxies were emerging
from a ``primordial soup'' of gas and dust, assembling now-fossilized structures that may persist even within our
own Milky Way.  Meanwhile, the rate of cosmic expansion was beginning to accelerate, as the Universe became
increasingly dominated by ``Dark Energy,'' whose origin remains the single greatest mystery in astronomy and cosmology
today.

Since Edwin Hubble's observations over 100 years ago, major advances in our understanding of the universe have come
from the two-step process of first taking images of the sky to locate sources of interest and then obtaining
information-rich spectroscopy to reveal the nature of those sources.  A modern example is the Sloan Digital Sky Survey
(SDSS) whose combination of panoramic broad-band ``imaging'' followed by dedicated spectroscopy yielded 
unprecedented in-depth data on over 1 million galaxies, mapping the present-day universe and making SDSS one of the most
highly cited surveys in the history of astronomy.

% Because a quality spectrum requires far more observing time per source than an image, SDSS pioneered ``high multiplex''
% spectrographs, capable of \emph{simultaneous} spectroscopy of hundreds of objects.

LSST's all-sky images will be 1,000 times deeper and detect far more distant galaxies than SDSS, but \textbf{no current
U.S. facility is capable of obtaining spectroscopic followup of LSST galaxies} at a level required to capitalize on the \$1B U.S.\
investment in that project.  In fact, an SDSS-like spectroscopic study of 1 million galaxies at LSST depth would require 300 years of observing on the largest telescopes with current instrumentation!  

The only way forward is encapsulated in one of NSF's ``10 Big Ideas,''  \emph{Harnessing the Data Revolution}: we can maximize the information content of LSST and other imaging facilities via machine learning from
optimally-designed spectroscopic training sets.  This proposal presents a coordinated framework with three critical
components necessary for success in this endeavor: 1) Using simulated spectroscopic$+$imaging data to define the
training sets required to address ambitious data-science challenges in Cosmology, Galaxy Formation, and Local Group
Archeology in the LSST era; 2) Preliminary design of Keck-FOBOS\footnote{Keck-FOBOS: The Keck Observatory Fiber Optic
Broadband Optical Spectrograph}, a state-of-the-art spectroscopic facility on one of the world's largest telescopes
optimized for providing the required training sets; 3) Preliminary design of the coordinated Keck-FOBOS observations
required as well as the systems needed to publicly deliver training set data products.  This MSRI-1 design proposal
lays out the path for maximizing panoramic imaging from LSST, WFIRST\footnote{WFIRST is NASA's space-based Wide-Field
Infrared Survey Telescope, expected to launch in the mid 2020's.}, Euclid\footnote{Euclid is led by the European Space
Agency with significant NASA involvement and will launch in 2021. Its primary mission is a 15,000 deg$^2$ imaging
survey in optical and near-IR wavebands.}, and other facilities with unparalleled deep and high-sampling density
spectroscopic followup.  Through a subsequent MSRI proposal we will deliver on our goals with an instrument deployment
in 2026, an array of spectroscopic programs, and associated public-ready training sets.

% We discuss three major scientific areas and identify Data Science Challenges in each category:

% \begin{enumerate}
% 	\item Significantly more powerful probes of Dark Energy and Cosmology
% 	\item Comprehensive understanding of the proto-galaxy ecosystem at $z\sim2$
% 	\item Archaeological studies of our own Milky Way galaxy and its Local Group neighbors
% \end{enumerate}


% Reference test: \citet{2015ApJ...798....7B}.



\subsection{Research Community Priority} 
\label{sec:community}
% \noindent\comment{3/4 page}

% \noindent "evidence, such as workshop
% reports or other publicly available indicators, that the infrastructure
% is a priority for a research community or important for a recognized NSF
% priority area such as one of NSF’s research Big Ideas."

% The coming decade will benefit from  major investments in both the depth and breadth of astronomical observations.  The James Webb Space Telescope will provide our deepest views yet, probing the first billion years of cosmic time over narrow sightlines.  At the same time, LSST's panoramic imaging will survey broad cosmic volumes of unprecedented size, detecting the majority of luminous sources a few billion years later.  LSST's optical imaging will be complemented by space-based near-infrared imaging from Euclid and eventually's NASA's WFIRST. 

The need for spectroscopic followup in the LSST era was made clear in the National Research Council's 2015 report, ``Optimizing the U.S. Ground-Based Optical and Infrared Astronomy System'' \citep{NAP21722} which recommended:

\begin{quote}
The National Science Foundation should support the development of a wide-field, highly multiplexed spectroscopic capability on a medium- or large-aperture telescope in the Southern Hemisphere to enable a wide variety of science, including follow-up spectroscopy of Large Synoptic Survey Telescope targets. Examples of enabled science are studies of cosmology, galaxy evolution, quasars, and the Milky Way.
\end{quote}

% \begin{quote}
% The science reach of LSST could be substantially enhanced by developing for the U.S. astronomy community a very-wide-field, massively multiplexed, spectroscopic capability. This facility should be capable of overlapping the majority of the sky area covered by the LSST surveys. Such a wide-field instrument or instruments should be sufficiently multiplexed to enable spectroscopic surveys of tens of millions of objects over several years. The science case for such a capability is rich for objects of a wide range of brightnesses...
% \end{quote}

% Therefore MSE as currently envisioned:  (https://www.noao.edu/meetings/2020decadal/files/PHall_MSE_Tucson_201802v3.pdf)
% •  will have access to 74% of the primary LSST footprint
% •  will meet its science requirements over 59% of the primary
% LSST area
% •  will observe at airmass < 1.4 (the LSST limit in its primary
% footprint) over 51% of the primary LSST area.

In addition to this report, further details of spectroscopic needs for LSST in all science areas were disseminated
after a 2013 workshop on this topic organized by the National Optical Astronomy Observatory (NOAO).
\textcolor{red}{JAN: I think at least as relevant is the NSF-requested Kavli/NOAO/LSST report,
https://www.noao.edu/meetings/lsst-oir-study/, which followed up on the Elmegreen report.} Based on these
recommendations, we propose the Keck-FOBOS instrument coupled with a suite of data-driven tools to address the
spectroscopic requirements of LSST and other photometric surveys at a final cost 20 times less than a new Southern
Hemisphere facility. Located in Hawaii, Keck-FOBOS would have access to more than 70\% of the LSST footprint, more than
adequate for our primary goal of building powerful spectroscopic training sets.  Compared to Prime Focus Spectrograph
(PFS) on Japan's Subaru Telescope, Keck-FOBOS would be 1.7$\times$ faster, provide UV sensitivity with a wavelength
range of 310--1000 nm (PFS covers 380--1250 nm), and offer high-density and more flexible target sampling over
``deep-drilling'' fields.  Keck-FOBOS would be operated on a U.S.\ telescope with dedicated U.S.\ access and a
commitment to supporting U.S.-led photometric surveys.  FOBOS is also complementary to future ambitious facilities that
would be optimized to cover wider areas (several deg$^2$ per pointing) at shallower depths.

The need for deep spectroscopic followup is particularly acute for LSST's major cosmological probes which rely on
``photometric redshifts:'' measures of the redshifts of objects -- which indicate how far back in time and space we are
looking -- based on imaging alone.  \citet{newman15} summarize the case for this and describe a redshift survey which,
if carried out with Keck-FOBOS, would increase LSST's Dark Energy Figure-of-Merit by a factor of 40\% at a cost of less
than 5\% of the LSST budget.  The urgent case for spectroscopic redshift training has been the subject of numerous
publications \citep[e.g.,][]{laureijs11,masters15, hemmati18}.

Meanwhile, the astronomy community recognizes that the coming era of ``Big Data'' astronomy culminating in LSST
necessitates ``harnessing the data revolution.''  Widespread community interest in advanced data science techniques
continues to grow amidst calls for educational programs, conference series, and research funding to support the
growth of a new field, ``Astroinformatics,'' which exploits the interface between astrophysics and statistics
\citep{borne09}.  Astronomy's largest organizations, including the American Astronomical Society and the International
Astronomical Union, have supported active working groups on astroinformatics and astrostatistics since 2015.  LSST
itself has built the Informatics and Statistics Science Collaboration and partnered with NSF to fund the Data Science
Fellowship Program to train astronomy graduate students in data science techniques.  Our proposal builds on and
contributes to these ongoing efforts.

% The coming decade will benefit from  major investments in both the depth and breadth of astronomical observations.  The
% James Webb Space Telescope will provide our deepest views yet, probing the first billion years of cosmic time over
% narrow sightlines.  At the same time, LSST's panoramic imaging will survey broad cosmic volumes of unprecedented size,
% detecting the majority of luminous sources a few billion years later.  LSST's optical imaging will be complemented by
% space-based near-infrared imaging from Euclid and eventually's NASA's WFIRST.

% The lack of telescope facilities capable of providing required spectrocopic followup for LSST, Euclid, and WFIRST is an urgent problem.  


\subsection{Science Goals and Data Science Challenges}
\label{sec:goals}

We identify ambitious ``data science challenges'' for the LSST era that would address major goals within each of three
core topics.  By simulating future wide-field imaging data as well as Keck-FOBOS spectroscopy, we will develop
astrostatistics techniques and applications over the proposal period that will refine the Keck-FOBOS instrument
requirements, inform the emerging design and operational modes, and define required training sets.  Tackling these
challenges requires a community-wide effort and will deliver wide-spread benefits.  Our specific purpose with this
proposal is to establish community priorities and success metrics and to coordinate the various groups working in this
area---many represented among our Senior Personnel.

\subsubsection{Enhancing Dark Energy Probes via Precision Cosmic Distances}
\label{sec:cosmology}
% \noindent \comment{1 page}

\begin{figure}[h!]
%
\vskip -0.1in
%
\includegraphics[width=\textwidth]{Hemmati18_Fig8_VVDS_spec.png}
%
\caption{\small \emph{Left:} A Self-Organizing Map (SOM) from
\citet{hemmati18} visualizing the relationship between galaxy
brightness in different broadband filters (projected into a two-dimensional space) and observed spectroscopic redshift (indicated by the color map).
SOMs guide the optimal construction of training samples by highlighting
which galaxy classes require targeting.  \emph{Right:} The
spectra associated with localized SOM regions have similar spectra, as well as similar redshifts. \textcolor{red}{JAN: Contra the previous text, I would say that the similarity of the spectra is neither remarkable nor surprising, since to be assigned to the same cell the galaxies have to have extremely similar SEDs, and hence spectra.  A bigger open question is can you get similar apparent SEDs from multiple redshifts (e.g., are there degeneracies with other parameters); the SOM is not necessarily one-to-one with redshift, but should be deterministic of spectral shape in any event.} }
%
\label{fig:SOM}
%
\end{figure}

The 2011 Nobel Prize in Physics was awarded for the discovery that the expansion of the universe has been accelerating instead of slowing down due to gravity as previously expected, starting when it was roughly half its current age.  This accelerated expansion is often attributed to a mysterious ``Dark Energy,'' the origin of which remains unknown.

Dark Energy is {\color{magenta} one of the} most {\color{magenta} fundamental} unsolved problems in both
cosmology and particle physics.  As such, it has inspired enormous
world-wide effort and the construction of dedicated ground and
space-based facilities.  These {\color{magenta} current and future surveys include DES \citep[e.g.][]{2018ApJS..239...18A}, KiDS \citep[e.g.][]{2017MNRAS.465.1454H}, LSST \citep[e.g.][]{2018arXiv180901669T}, WFIRST \citep{2015arXiv150303757S}  and Euclid \citep[e.g.][]{2011arXiv1110.3193L}.  
Large area photometric surveys are expected to map the universe to unprecedented depth and area. This will enable us to study the growth of structure to high precision to set accurate constraints on a time evolving dark energy equation of state. 

The three dominant cosmological probes that will be used to constrain dark energy are angular correlations of galaxy positions, their shear field generated by Weak Graviational Lensing and the cross correlation between them \citep[e.g.][]{2015MNRAS.451.4424K}. The combination of these probes can be expected to test dark-energy and modified gravity models to unprecedented precision \citep{2006astro.ph..9591A}.
%\begin{equation}
%{\rm FOM} = \frac{1}{4 \sqrt{\det(F_{\rm DE}^{-1})}} \, ,
%\end{equation}
%where $F_{\rm DE}^{-1}$ denotes the 2x2 submatrix of the inverse of parametrs %that describe the time evolution of the dark energy equation of state
%\begin{equation}
%w(a) = w_0 + (1 - a) w_a \, .
%\end{equation}
However in contrast to spectroscopic surveys, photometric surveys have less accurate information of distance, or redshift. They typically use broad photometric bands, which fundamentally limits the amount of information available on the spectral energy distribution of galaxies (SED) and therefore on the galaxies redshift. 
Inaccurate redshift information is therefore one of the dominant sources of systematic uncertainty in these surveys \citep{hemmati18} that are quite sensitive to modelling biases \citep{2018MNRAS.476..151E, 10.1093/mnras/sty2902}.
Spectroscopic validation of redshift samples is therefore of paramount importance for the success of all these missions. 

The two primary techniques to obtain photometric redshifts are Machine Learning based approaches that derive a flexible mapping between the color space of the galaxy and it's redshift. This mapping is derived directly from a dataset with both photometric and spectroscopic observations. The alternative approach uses models for the SED of galaxies and fits them to their observed photometry. Both these methods require spectroscopic validation as they can fail in practise. The Machine Learning based approach can `learn' the artificial selection function of the spectroscopic survey both in color space or in radial direction. Especially the latter scenario is hard to detect due to the degeneracy between color and redshift in broad band photometry and has been recognized as a challenge in large area photometric surveys \citep[see e.g.][]{2016PhRvD..94d2005B}. Similarly the model based template fitting approach can be biased by incomplete sets of SED models that don't sufficiently cover the color space. 

Keck-FOBOS will be able to provide photometric redshift training and validation samples that can not only \emph{increase the Dark Energy Figure of Merit in LSST by 40\%} \citep{newman15} but, more importantly, provide vital confidence in the cosmological results obtained by the aforementioned surveys. We argue that Keck-FOBOS is particularly powerful in this respect,
because it has no redshift desert and can measure spectroscopic redshifts
above $z > 1.5$ via rest-frame UV features, which eliminates the need for
expensive, space-based near-IR spectroscopy. 

}


\medskip
%
\chal{photozs}
%
\noindent {\bf Data-Science Challenge \ref{photozs}: Enable High-precision LSST
Photometric Redshifts ($\sigma_z/(1+z) \lesssim 0.02$ at i(AB) $<$
25.3) with Targeted Training Spectroscopy}.  Delivering optimal photometric redshifts with minimal errors per object will require sets of $>10,000$ spectra for training purely machine learning-based algorithms or optimizing our knowledge of galaxy spectra and calibration errors for template-based and hybrid algorithms. {\color{magenta} Specifically it will enable us to refine and study the SED models used for photometric redshift estimation. This will not only improve photometric redshift accuracy, but will also allow us to derive galaxy properties like stalla mass or age for these photometric samples. This effectively enables the study of stellar evoution on cosmological scales. }  Our proposed FOBOS instrument is ideally suited to provide {\color{magenta} these datasets} considering the requirements in \citet{newman15}.  

{\color{magenta} To reach this science goal it is paramount to optimize the observation strategy to efficiently populate the color space of a photometric sample, while avoiding significant selection biases. 
A state-of-the-art technique to represent the color space in low dimensions is the Self-Organizing Map (SOM, Fig \ref {fig:SOM}) technique that allows the representation of a high dimensional input space in 2D grid cells. The grid cells shown in Fig. \ref{fig:SOM} represent coherent cells of input in high dimensions. We propose to use simulated spectroscopic and photometric survey data that mimic the expected selection biases of the FOBOS spectroscpopic sample. We then benchmark the coverage of photometric color space wrt. our spectroscopic sample and, similarly, the color-space coverage of our derived SED models as a function of the choosen FOBOS observing strategy. Obtaining these simulations and benchmarking the color space coverage is a time consuming task. In order to optimize our observing strategy we therefore propose to use Bayesian optimization \citep[see e.g.][]{2017arXiv170306240K,2018arXiv180512168P,  2019arXiv190111515N}. 
This methodology is especially suited to optimize target functions, e.g. the minimal coverage of spectroscopic observations in color space, that requite considerable time to evaluate, which is certainly the case here. The idea of these methods is to guide the optimization steps via an interpolation scheme. 
As this methodology is very general and can be used to jointly optimize multiple target functions, we will also be able to account for the requirements of different science goals, that might need slightly different observation strategies. 
}

%Our goal is to design and deliver an optimized set of spectroscopic redshifts which will enable photometric redshifts to be accurately 'painted on' to LSST imaging-only objects, and thereby improve both dark energy and galaxy evolution science from LSST.

%, but a complete program would
%require a 400-night investment in 10 m telescope time.  This challenge
%demands a reduction in the required FOBOS training sample by a factor of
%$\sim$4 via clever application of state-of-the-art machine learning
%techniques.
% JAN: I don't agree with that text, but also think it's way down in the weeds for a pre-proposal.  
%Neural network trained photo-$z$s have long been recognized for
%providing the best precision when sufficient training sets are available
%\citep[e.g.][]{bundy06}, and significant effort is underway in
%optimizing their application to future cosmological imaging surveys.
% JAN:  NNs are not necessarily leading the field right now, but they also don't solve this issue.  The requirement is set based on being able to obtain a <0.002(1+z) calibration from the training set if redshift success is high.  
%\citet{hemmati18} for example have exploited Self-Organizing Maps (SOMs,
%Fig \ref {fig:SOM}) to sort multiband photometric data by observed
%redshift in order to select optimized training samples for spectroscopic
%followup.
% JAN: That does not change the math for direct calibration, which the SOM fundamentally is doing..



\subsubsection{A Comprehensive Picture of the Proto-galaxy Ecosystem}
\label{sec:galaxies}
\noindent \comment{1 page}

Roughly 4 billion years after the Big Bang ($z \sim 2$), the universe
entered a key epoch in which proto-galaxies transitioned from
interacting, gas-rich systems into the more ordered, star-dominated
structures that populate the universe today.  This period marks the peak
of global star formation rate and galaxy assembly history.   To
understand it, we must not only study the galaxy population at this
epoch but the entire galaxy ``ecosystem'' which includes their
gas-filled environments.  The goal is to build a comprehensive picture
of the physical processes that fuel proto-galaxy growth, shape their
internal structure, and influence their environment.

LSST's panoramic imaging will detect huge numbers of galaxies at this
epoch.  Targeted followup with Keck-FOBOS will allow us to ascribe
detailed galaxy and environmental information from deep spectroscopic
training samples to the much larger cosmic volumes surveyed with
broad-band imaging.

\begin{figure}[h!]
%
\vskip -0.1in
%
\includegraphics[width=0.5\textwidth]{Parks18_Fig7_fig_labels.pdf}
%
\caption{\small Example of machine learning applied to absorption
features in rest-frame UV spectroscopy to detect two intervening gas
clouds along the line-of-sight \citep[from][]{parks18}.  A spectrum is
shown at top with lower panels indicating associated labels conveying
physical information.  Keck-FOBOS will provide a rich set of similar
features and the opportunity to transfer labels to imaging and higher
redshift data sets.}
%
\label{fig:absorber}
%
\end{figure}

\medskip
%
\chal{phot}
%
\noindent {\bf Data-Science Challenge \ref{phot}: Apply Deep Learning to
infer star formation rates and formation histories, dust content, wind
properties, and stellar masses from $z \sim 2$ photometry}.  The range
of observed spectral types is remarkably constrained by broad-band
imaging (Figure \ref{fig:SOM}, right panel), suggesting a far greater
potential for imaging data to reveal physical properties with sufficient
training than conventional modeling of spectral energy distributions
(SEDs) would suggest.  Applying machine learning, our challenge is to
deliver SDSS-like information for millions of imaged galaxies at $z \sim
2$.  With simulated data sets, we will investigate derived uncertainties
and biases and explore benefits from incorporating additional imaging
information like morphology, structure, and size from a wide range of
wave-bands (e.g., LSST plus Euclid plus WFIRST).  The exercise will
define requirements for Keck-FOBOS instrument performance and the FOBOS
Public Survey design.

\medskip
%
\chal{uv}
%
\noindent {\bf Data-Science Challenge \ref{uv}: Enable label transfer
from rest-frame optical to UV stellar and ISM indicators}.  There are
many powerful gas and stellar spectral features just redward of the
Lyman-$\alpha$ line at 1216 \AA.  By combining Keck-FOBOS UV and near-IR
spectroscopy (e.g., from PFS) at $z \sim 2$, we can transfer ``labels''
best modeled in the rest-frame optical to spectra at UV wavelengths, at
least for certain types of galaxies.  This ``label transfer'' will
dramatically enhance interpretation of JWST discoveries of the first
galaxies ($z \sim 10$) for which rest-frame UV imaging and spectroscopy
will be most accessible.  A similar application can ascribe the escape
fraction of Lyman continuum radiation observed in the Keck-FOBOS Public
Survey to constrain the sources responsible for ``reionization'' at $z
\sim 6$.  With simulated spectral observations, we will determine the
extent of label transfer that is possible and set requirements on
training samples.


\medskip
%
\chal{lowsnr}
%
\noindent {\bf Data-Science Challenge \ref{lowsnr}: Train short
spectroscopic exposures in combination with LSST photometry to provide
environmental diagnostics for 1M galaxies at $z=1$--$2$}.  Photometric
redshifts, while acceptable in large cosmological analyses, wash out
information about the local position of galaxies with respect to one
another.  To characterize a galaxy's local environment and identify its
neighbors requires (observationally expensive) spectroscopic redshifts
(spec-$z$s).  However, with improved photometric redshifts available
from Challenge \ref{photozs} and strong priors on spectral types
(Challenge \ref{phot}), machine learning techniques can yield
\emph{spectroscpic} redshifts at much lower signal-to-noise than
conventional redshift measurements. Specifically, our challenge is to
develop a methodology that can measure 300 km s$^{-1}$ accuracy
spec-$z$s on spectra obtained in just 10 minutes with Keck-FOBOS.  This
would enable an SDSS-like environmental study of 1M galaxies at
$z=1$--$2$ in just 20 nights of 10 m telescope time, making it a
compelling sub-component of the Keck-FOBOS Public Survey.






%\begin{wrapfigure}{r}{0.6\textwidth}\small
%%
%\includegraphics[width=0.6\textwidth]{TestBench.jpg}
%%
%\caption{\label{fig:testbench} A schematic of the current UCO fiber test
%bench.  The components to the right of the first fiber positioner will
%be repackaged and delivered to Keck for our on-sky experiment; the
%fibers themselves will be fed light from Keck after being plugged into a
%DEIMOS mask.  Image Credit: Jaren Ashcroft.}
%%
%\end{wrapfigure}


% \begin{wrapfigure}{r}{0.5\textwidth}
%  \includegraphics[width=0.5\textwidth]{madau_plot.pdf}
%  \caption{\small From Madau \& Dickinson (2014) }\label{fig:Madau_plot}
% \end{wrapfigure}

%A critical transition in evolution of galaxies in our Universe emerges
%when one considers the volumetric star-formation rate as a function of
%cosmic time 
%
% - cosmic star-formation rate shows the unique phases of the growth of
% galaxies in the Universe
%
% - the detailed cause for the decline in the CSFR since z~2 remains
% unclear, but it could be thought of as the starvation of galaxies from
% the rapid gas accretion they enjoyed in the early universe.
%
% - the outbreak of star-formation sparked by free-falling accretion of
% gas is now stymied by the size of the galaxies themselves.
%
% - a number of fundamental properties of galaxies have emerged over the
% past ~30 years: mass, size, metallicity, star-formation rate, angular
% momentum, and gas fraction.  Environment
%
%
%The rate at which stars have been born in the Universe has varied
%dramatically over cosmic history.  At roughly half its current age, the
%Universe was on average forming 10 stars for every one star formed
%today \comment{ref}.  At these epochs, galaxies looked rather different
%then they do now, dominated by all-consuming star-formation regions
%\comment{ref}.  These morphologies are only seen in rare star-burst
%galaxies in the Local Volume \comment{ref}.  The 

\begin{figure}[h!]
%
\vskip -0.1in
%
\includegraphics[width=\textwidth]{CannonLAMOST.jpg}
%
\caption{\small Verification of application of {\it The Cannon} to
LAMOST spectra based on stellar parameters determined by the
high-resolution APOGEE data from Ho et al.\ (2017).  Each panel shows
the derived effective temperature, $T_{\rm eff}$, and surface gravity,
$\log g$, for each star, with the color representing the density of
stars at each position.  The left panel shows the results for the LAMOST
spectra using a direct fitting approach, the right panel shows the
results derived from the high-resolution APOGEE data, and the middle
panel shows the results of using {\it The Cannon} to determine the
stellar parameters using the low-resolution LAMOST spectra trained by
the APOGEE-derived parameters.  Results from {\it The Cannon} are more
accurate and astrophysically plausible.}
%
\label{fig:Cannon}
%
\end{figure}

\subsubsection{Unraveling the Formation History of our Local Group of Galaxies}
\label{sec:localgroup}
\noindent \comment{1 page}

Our Local Group of galaxies --- composed of the Milky Way (MW) Galaxy,
the Magellanic Clouds, the nearby Andromeda (M31) and Triangulum (M33)
Galaxies, and a multitude of satellite galaxies --- is just one
realization of the galaxy-formation process, but it is the one that we
can study in the greatest detail.  Large-scale imaging surveys executed
over the past 25 years have revolutionized our census of the Local
Group.  In particular, SDSS and Pan-STARRS have unveiled numerous
stellar streams and other halo substructures in both the MW and M31,
including a stellar bridge stretching between M31 and M33.  We expect a
hundredfold growth in the census of halo substructures in the MW via the
upcoming LSST and WFIRST surveys.  Follow-up spectroscopy of Local Group
member stars allow us to, e.g., constrain the orbits of stellar streams
and the present-day enclosed mass of the galaxies they orbit (refs), as
well as the age and chemical composition of their stellar populations
(refs).  At the same time, cosmological simulations [refs], like
IllustrisTNG [others], can now simulate the full chemo-dynamical
evolution of Local-Group-like overdensities in the Universe to which
data can be meaningfully compared.  Finally, the Gaia satellite (ref) is
currently revolutionizing our understanding of the MW by providing
distances and on-sky motions for more than a billion stars spanning the
full extent of its disk.  This simultaneous maturation of both the
theoretical and observational data will allow us to form physically
motivated models for the formation history of the Local Group and its
constituents.

As we obtain deeper images and more varied data sets, follow-up
spectroscopy of Local Group objects of interest becomes ever more
difficult.  As it is, e.g., Keck-DEIMOS programs to measure the radial
velocities of stars in the MW halo or the M31 disk require observations
of up to 10 hours, depending on the population being probed [Tollerud,
Dorman, Cunningham].  Given such long integration times, one approach is
to maximize the number of targets observed in a single pointing.
However, one can also appeal to machine-learning algorithms to infer the
relevant physical quantities statistically from both multi-band imaging
and lower quality spectra (low resolution and S/N) using a relatively
small, yet high-S/N, training set.  There has been a significant push
over the past 5 years toward building such machine-learning
applications.

For example, [Ness+] have developed {\it The Cannon}, a supervised
learning algorithm that uses spectra with known stellar parameters to
label spectra where those parameters are unknown.  In one application,
they determined three fundamental parameters for 55000 APOGEE spectra
using a 1\% training sample.  Additionally, [Ting+] have developed {\it
The Payne}, which uses a neural network and theoretical stellar spectra
to determine 25 stellar-abundance labels providing the detailed chemical
make-up of each observed star.  Example applications of these techniques
are shown in Figure X.

Our proposed effort builds on new lines of inquiry based on these
successes, both in terms of application of these machine-learning
techniques to new data sets and development of new techniques as we
discuss below.

\medskip \chal{mwhalo} \noindent {\bf Data-Science Challenge
\ref{mwhalo}: The chemical evolution and assembly history of the MW
stellar halo.} LSST and WFIRST will reveal a trove of substructure in
both the MW and M31 halo.  We will design an observational program that
would employ Keck-FOBOS to observe main-sequence turn-off and red-giant
stars in these substructures within the MW.  These and additional data
(APOGEE, H3) will be used as training sets to build data-driven models
for the stellar parameters (temperature, surface gravity, metallicity)
for all halo stars with LSST+2MASS+WISE+WFIRST multi-band photometry.
These will be combined with dynamical data and compared with
cosmological simulations to build a generative model for the assembly
history of the MW stellar halo.

\medskip \chal{m31} \noindent {\bf Data-Science Challenge \ref{m31}: The
differential chemical evolution of M31 and MW.}  A natural extension of
Data-Science Challenge \ref{m31} is to perform the same analysis for the
halo of M31.  However, we cannot expect to obtain high-quality spectra
of individual main-sequence stars at the distance of M31 with
Keck-FOBOS.  Moreover, training a chemical evolution model based on
training sets composed of stars in the Milky Way could lead to
systematic errors:  The Milky Way and Andromeda have distinct chemical
evolution histories (ref), despite being relatively similar in many
other respects.  We will therefore design an observational program that
will obtain deep observations of giant stars in the M31 halo.  These
data will drive a machine-learning algorithm that combines a model of
the MW halo with results from cosmological hydrodynamical simulations to
constrain the differential history of the MW and M31 stellar halos.

\medskip \chal{gaia} \noindent {\bf Data-Science Challenge \ref{gaia}:
Stellar parameter determinations for a billion stellar spectra.} While
providing on-sky motions and photometry for 1.7 billion stars in the MW,
fewer than 10\% of stars will have a full complement of
three-dimensional space motions, fewer than 0.3\% will have basic
stellar parameters, and only 0.1\% will have measured chemical
abundances.  In addition, Gaia distance measurements have errors that
increase quadratically with distance.  To realize the full potential of
the Gaia astrometric catalog, one needs 3D position and 3D velocity
vectors and chemical abundances for each star.  We will therefore design
training sets to observe with Keck-FOBOS that, when combined with
existing high-resolution datasets (e.g., APOGEE, WEAVE) will allow us to
build data-driven models of distance, temperature, surface-gravity, and
stellar abundance for {\it all} stars in the Gaia dataset.  These data
will allow us to isolate coeval populations in the Galactic disk that
can be combined with very high-resolution simulations of the Milky Way
to provide a detailed evolutionary history of our Galactic home.


% Deeper in their potential wells, the shorter dynamical times mean that
% on-sky density is insufficient to isolate coeval and/or comoving
% stellar groups.  However, simulations show that stars born within
% stellar clusters that then diffuse into, e.g., the disk of the Milky
% Way maintain their coherency in phase space over long timescales.
% Therefore, gy grouping stars by the chemical makeup of their stellar
% atmospheres --- a procedure called chemical tagging, we can isolate
% coeval stellar groups.

% Existing and ongoing surveys, like SEGUE, RAVE, and APOGEE, and
% planned surveys, like WEAVE and 4MOST (refs), provide critical
% supplementary observations, but these efforts only just begin to
% fulfill the potential of the Gaia astrometric data.  Directly
% observing the all 1.7 billion stars is simply untenable for the
% foreseeable future.


% To realize the full potential of the Gaia astrometric catalog, one
% needs the full 6D phase space for each star and its chemical abundance
% pattern.  Although anything more than simple total metallicity
% measurements are out of reach for FOBOS, a strategically developed
% training set observed at high S/N can be used for kinematics and
% simple stellar parameter determination.

% The use of FOBOS to compare the chemodynamical evolution of the M31
% and MW halos.

% \medskip \chal{gaia} \noindent {\bf Data-Science Challenge \ref{gaia}:
% Stellar parameter determinations for a billion stellar spectra.}  To
% realize the full potential of the Gaia astrometric catalog, one needs
% the full 6D phase space for each star and its chemical abundance
% pattern.  Although anything more than simple total metallicity
% measurements are out of reach for FOBOS, a strategically developed
% training set observed at high S/N can be used for kinematics and
% simple stellar parameter determination.

% Each survey chooses to balance the two key limitations of such surveys
% in different ways, favoring either on-sky area covered (e.g., our
% Milky Way galaxy fills the entire sky) or smaller areas to greater
% depth.

% (e.g., SDSS, SEGUE, APOGEE, Rave, Pan-STARRS, PAndAS, PHAT, Gaia)

% While only a single realization of the galaxy-formation process, our
% Local Group provide unparalleled opportunities to study galaxies in
% exquisite detail, by virtue of being able to spatially resolve their
% constituent parts.

% Instead, a growing number of studies aim to infer the necessary
% measurements using novel applications of machine-learning.  For
% example, Ness et al.\ have developed {\it The Cannon}, a supervised
% learning algorithm that uses spectra with known stellar parameters to
% label spectra where those parameters are unknown.  In one application,
% they determined three fundamental parameters for 55000 APOGEE spectra
% using a 1\% training sample.  Additionally, Ting et al.\ have
% developed {\it The Payne}, which uses a neural network and theoretical
% stellar spectra to determine 25 stellar labels.  In both of these
% applications, careful attention has to be given to the limitations of
% the training sample, similar to the SOM example provided in Figure 1.
% At their core, machine-learning algorithms, particularly supervised
% learning, are only as good as the parameter space spanned by their
% training sets.

% Notes from Yuan-Sen:
%   - One has to check the survey coverage more carefully. I believe
%     most surveys that you mentioned are focusing the disk, and FOBOS
%     will be a great complementary survey.
%   - This might be too details for such proposals, but it might useful
%     to mention that low-resolution spectroscopy (i.e., fitting the
%     full spectra) is only recently made possible by The Cannon and The
%     Payne.  And at low-resolution, ab initio spectral models are
%     typically not good enough, and one might want to resort to
%     data-driven model.  Having large (ideally, high S/N) empirical
%     library will be very useful. And FOBOS might provide that.
%   - Complementing Gaia with spectra are not only about chemical
%     abundances or simply RV. Knowing stellar parameters are *crucial*
%     to get very precise distances. Gaia parallax suffers from the fact
%     that the errors grows quadratically with distance. So Gaia
%     typically does not give good distances. But combining stellar
%     spectroscopic parameters + parallax, we will be able to predict
%     the absolute magnitude of stars, and hence gives much more precise
%     distances -- Hogg, Eiler, Rix+ 18, Ting, Hawkins, Rix+ 18. I think
%     simply having spectra for all halo stars Gaia survey will be a
%     tremendous service of Gaia. And this might be a good sell.
%   - "Although anything more than simple total metallicity measurements
%     are out of reach for FOBOS, a strategically developed training set
%     observed at high S/N can be used for kinematics and simple stellar
%     parameter determination." Actually, that's the whole point of The
%     Payne -- we can measure > 16 elemental abundances now from LAMOST
%     (R=1800, optical full range, S/N=30). So having FOBOS + smart way
%     (The Cannon/The Payne) should give amazing chemical information.
%   - Then combined with the fact that Gaia + FOBOS (stellar parameters
%     + parallax -> stellar absolute magnitude -> distance to ~10\%),
%     and TESS/PLATO + FOBOS which should give you stellar ages from
%     stellar spectra (you can get stellar ages to 0.1 dex with LAMOST
%     spectra -- Ting & Rix, 2018), we would have (a) 20+ elemental
%     abundances, (b) distances to ~10\%, (c) stellar ages to 0.1 dex
%     (20\%).  And this should tell us tremendously much on the Galactic
%     halo. But how to frame it as a "challenge" might be tricky.
%   - Abundance recovery of GALAH using LAMOST (FOBOS-like resolution)
%     spectra.  Figure provided.
%   - MW rotation curve measurements from Eilers et al.  Gaia+spectral
%     distances.
%   - Age determination from LAMOST, Ting & Rix 2018.

% Why couldn't we do the rotation curve using DEIMOS + Gaia.  anything
% unique about fobos?

% Beyond its internal chemodynamical history, LSST will also reveal
% hundreds of faint Milky Way satellites, as well as discover new
% satellites of the Andromeda and Triangulum galaxies, extending the
% census of these galaxies.  These faint galaxies provide critical tests
% of the hierarchical formation of the Local Group.

% \medskip \noindent {\bf Data-Science Challenge 7: Dynamics of newly
% discovered Ultra-Faint dwarfs.}

% \subsubsection{Time Domain}
% \label{sec:timedomain}
% \noindent \comment{1/2 page}



\section{Project Implementation}
\label{sec:project}

This proposal involves three coordinated activities: 1) Organizing and evaluating the results of a community-wide
effort to address simulated Data Science Challenges; 2) Completing Preliminary Design for the Keck-FOBOS
instrumentation, informed in part by refining requirements as a result of (1); 3) Designing the operational modes,
planning tools, data analysis software, and serving platforms necessary for delivery of public training sets.
Anticipating significant progress in all three activities, we will request NSF MSRI-2 funding in 2021 to build and
deploy Keck-FOBOS at the telescope, carry out required observations, and publicly serve the data products.  FOBOS would
see first light in 2027 and carry a total cost of \$32M (without contingency in 2019 dollars).  While we focus the
current request on work required for the Preliminary Design Phase, we outline the overall project plan and final
deliverables in order to motivate this work.

\subsection{Keck-FOBOS Instrument Concept}
\label{sec:concept}
% \noindent \comment{1 page}

% Here's an alternative way to put in figures if we want captions on the side (to save space)
% Could introduce a new ``counter'' to count and label figures appropriately
\centerline{\hbox{\includegraphics[width=0.6\textwidth, angle=0]{FOBOSatKeck_v1.pdf}
    \hspace{0.1cm} \vspace{2in}
    \parbox[b]{0.3\textwidth}{\small {\bf Figure ??:} Rendering of FOBOS instrument systems deployed at the Keck II Nasmyth port.  By mounting the FOBOS spectrographs under the Nasmyth platform, other instruments like DEIMOS can maintain access to the telescope. \vspace{2cm}}}}



% \begin{figure}[h!]
%  \vskip -0.1in
%  \includegraphics[width=\textwidth]{FOBOSatKeck_v1.pdf}
%  \caption{\small Rendering of FOBOS instrument systems deployed at the Keck II Nasmyth port.  By mounting the FOBOS spectrographs under the Nasmyth platform, other instruments like DEIMOS can maintain access to the telescope.}\label{fig:layout}
% \end{figure}

Mounted at the Nasmyth focus of Keck II Telescope at WMKO\footnote{WMKO: William M.\ Keck Observatory operates the two twin 10 m Keck Telescopes on Mauna Kea, Hawaii.}, the Fiber Optic Broadband Optical Spectrograph (FOBOS) will be one of
the most powerful spectroscopic facilities in the next decade.  FOBOS consists of several key components (Fig
\ref{fig:layout}).  A compensating lateral atmospheric dispersion corrector (CLADC, not pictured) ensures that target
light from all wavelengths falls on allocated fibers while also correcting image aberrations at the edges of the 20
arcmin diameter Keck field.  Each of the CLADC lenses is 946 mm in diameter, the first two closely spaced with lateral
relative motions enabled by three barrel-mounted actuators.  The final CLADC lens surface serves as the vertical
mounting plate for roaming Starbugs fiber positioners.  It translates to track focal plane tilt.  Starbugs patrol a
large on-sky area ($\sim$1 arcmin), enabling flexible and dynamic targeting configurations with adjacent fibers as
close as 10 arcsec.

A total of 1800 150 $\mu$m core diameter fibers are deployed at the curved focal plane, which rotates and translates to
maintain image positions as the telescope tracks across the sky.  The fiber run is kept at less than 10 m to
maintain high throughput at UV wavelengths, and special care is given to stress-relief cabling to minimize variable
focal ratio degradation over the fiber run.

Sets of 600 fibers feed each of three identical spectrographs.  Each spectrograph uses a series of dichroics to divide
the input light into four wavelength channels with combined coverage from 310 to 1000 nm and mid-channel spectral
resolutions of $R \sim 3500$.  The dispersed light in each channel is focused by an f/1.1 catadioptric camera and
recorded by an on-axis 4k$\times$4k CCD mounted at the center of the first camera lens element.  Spectrographs are
mounted in a temperature controlled housing installed under the Nasmyth Deck to allow space for other Keck instruments
above.  The end-to-end instrument throughput is greater than 30\% at all wavelengths.

FOBOS includes observatory level systems for precise instrument calibration using dome-interior screen illumination, a
metrology system for accurate fiber positioning, and guide cameras for field acquisition and guiding.  The instrument
design envisions future upgrades including alternate collecting modes that deploy multiple fiber bundles, feeds to
other fiber-based spectrographs at different wavelengths or spectral resolutions, and the ability to support and
benefit from image corrections with Ground-Layer Adaptive Optics.



\subsection{Keck-FOBOS Instrument Design Effort}
\label{sec:design}
% \noindent \comment{1 page}

Keck-FOBOS will complete its current conceptual design phase in October 2019.  Funding from this proposal will support preliminary design beginning in November 2019.  A schedule of milestones is attached and more information provided in the Project Execution Plan (PEP).  Major components of the preliminary design effort are described below.

\noindent \textbf{Atmospheric Dispersion Compensator (ADC).} The opto-mechanical design, tolerancing, lens cell design, motion systems, and software controls design of the ADC will be completed.  

\noindent \textbf{Focal Plane System.} The final ADC lens element serves as the focal plane mounting plate for the fiber positioners.  This focal plane system must rotate and translate to track the field and refraction angles from the ADC.  Mechanical design, including flexure analysis and the selection of drive mechanisms and potential vendors will be completed.  This system also defines one of the interfaces to the Keck Telescope and must comply with Keck Observatory space envelopes, servicing needs, and other requirements.  The focal plane system also interfaces with guide cameras for field acquisition and guiding.

\noindent \textbf{Starbugs fiber positionsers.} Starbugs are a positioning technology developed and deployed by the
Australian Astronomical Observatory (AAO) which has partnered with our team to generate a conceptual design for
Starbugs in the context of FOBOS.  Design requirements for Starbugs in FOBOS are more relaxed than the currently on-sky
TAIPAN instrument thanks to the larger physical plate scale at Keck.  AAO will serve as a vendor during preliminary
design but is interested in exploring a partnership and in-kind contribution model in the construction phase.  In addition to the Starbugs themselves, a fiber metrology system (for accurate closed-loop positioning) will also be developed.

\noindent \textbf{Fiber System.} We will complete the optical design and processing plan for affixing forward optics
lenses to each fiber's head (these demagnify and speed up the beam for proper fiber coupling).  A micro-lens array
solution will be developed for a central, fixed-position 4.5-arcsec diameter IFU for fast source acquisition. This
workpackage also includes the stress-relief cable system and fiber termination hardware and processing.

\noindent \textbf{Spectrographs.} The optical systems and components (slit, collimator, dichroics, gratings, and camera), an analysis of acceptable tolerances and performance, their mechanical supports, software controls, and the overall enclosure will all be advanced through preliminary design.  Detectors, cryostats, read-out electronics and systems for thermal management will be designed.

% \noindent \textbf{Calibration System.} This package includes design of an interior dome screen and projection system for injecting calibration sources with sufficient spatial uniformity and stability into the instrument.  We will work with the Observatory to develop an integration and controls plan.  No such calibration system currently exists at Keck.

% \noindent \textbf{Auxiliary Systems.} Design of auxiliary systems includes Nasmyth platform interfaces, utilities access, fiber routing and support, thermal control and vibration control systems.


\subsection{Addressing Data Science Challenges and Designing FOBOS Training Sets}
\label{sec:survey}
% \noindent \comment{1 page}

Our team includes leading experts on data science applications to astronomy and LSST specifically.  We will also use
our established connections to LSST's Informatics and Statistics Science Collaboration (ISSC) to advertise, recruit,
and coordinate efforts to tackle the Data Science Challenges described in Section \ref{sec:goals}.  Our proposal
request includes two open workshops to motivate progress and discuss results. At the end of the proposal period, we
will publish the results and developed software packages.

The Data Science Challenges require work on simulated imaging$+$spectroscopic data sets where input physical properties
(e.g., redshift) can be imposed and the output, recovered values compared against the input.  Simulated imaging data
(e.g., from LSST and WFIRST) are in-hand, while mock spectroscopy will be provided by a Keck-FOBOS instrument simulator,
an initial version of which has already been developed.  Further advances to be supported by this proposal include
improved error modeling and simulating systematic effects from detector artifacts, image quality aberrations informed
by the emerging detailed optical design, and variable observing conditions.

The resulting success in addressing each Data Science Challenge will define a level of readiness and set requirements
on the associated Keck-FOBOS training sets required, including number of sources, pointings, magnitude limits,
signal-to-noise thresholds, and observing conditions.  Preliminary observing design and a description of required
operational modes to efficiently observe these training sets will begin with this proposal.  Operational modes will set
requirements on target aggregation and prioritization systems, field acquisition speed, field rotation range, zenith
avoidance zone, reconfiguration time, calibrations, read-out time, quicklook reduction software and processing rates.
We will develop integrated program concepts that efficiently combine required observations.  Detailed survey and
execution plans will be completed in the next phase of this project (MSRI-2).  Roughly 20\% of Keck observing time is
open to the public, and as in previous federally-funded projects, we fully expect that Senior Personnel at Keck
institutions will be successful in collaborative efforts to secure significant amounts of additional telescope
observing time to enable rapid, publicly release of training data with any proprietary period waived
\citep[e.g.,][]{newman13}.

% The complete photo-$z$ training survey described in \citet{newman15} would
% require 15 independent pointings, each spanning 0.1 deg$^2$ with a target density of 6 arcmin$^{-2}$ (8 arcmin$^{-2}$
% when including $z > 1.5$ galaxies accessible in the UV with Keck-FOBOS), perfectly matched to the Keck-FOBOS
% field-of-view and target density.  With a conservative exposure time of 100 hours to reach 75\% redshift completeness
% for 40,000 galaxies with $i_{\rm AB} < 25.3$, the Neman survey would require 400 nights.  Challenge \ref{photoz} would
% reduce the required survey duration by a factor of at least four.  Meanwhile the extreme depths and flux-limited
% selection are likely also requirements for training sets associated with Challenges \ref{phot}, \ref{uv}.

% A wider and shallower survey component is envisioned for Challenges \ref{lowsnr} and \ref{gaia}.  With 10-minute
% integrations, a 52 deg$^2$ Keck-FOBOS sample of environmental diagnostics for 1 million galaxies could be carried out
% in less than 20 nights.  This program would sample at $z \sim 1.5$ the same cosmic volume as SDSS.  A program of a
% similar scale would provide training set data for inference of stellar parameters in the Milky Way.  These shallow programs would be integrated with the deeper components described above into a single survey plan.



% The components of the FPS are as follows:
%     - Photo-z training samples (Newman, Masters)
%         - z>1.5 in the blue
%         - z<1.5 in the red
%     - Ly continuum
%         - z~2 in the blue
%         - z~7 in the red
%     - Machine-learning the SDSS but at z~2


\subsection{MAISTRO: Target Allocation with Artificial Intelligence}
\label{sec:targeting}
% \noindent \comment{1/2 page}

Powered by Starbugs fiber positioners, Keck-FOBOS will enable fast, dynamic reallocation of fibers.  To efficiently
determine the best options given a wide range of possible targets and desired observing outcomes, we will develop a
preliminary design for MAISTRO\footnote{MAISTRO: Modular Artificial Intelligence System for Target Reallocation and
Observing.} an ``artificial intelligence'' (AI) targeting system that will learn optimization strategies for assigning
targets from a database of overlapping observing programs with pre-defined priorities.  The AI package will aggregate
data quality using a quick-look reduction package, science-driven performance metrics, {\it and real-time assessments
of the observing conditions} to make dynamic targeting recommendations.  For example, if conditions are
slightly less than optimal, MAISTRO would reconfigure Starbugs to brighter objects in a field or implement a different program prioritization.  MAISTRO would incorporate updated target lists and priorities from the active observer and could easily be over-ridden at any time.   Fractions of the full Keck-FOBOS multiplex might also be reserved ``manual targeting'' as required by the P.I.  

%   - maintains a database with observational progress on individual
%     targets in the survey and
%   - dynamically reallocates fibers based on real-time assessments of
%     the aggregate S/N of each target to meet the specific need of each
%     science case.

% This requires significant design and testing of a combined software
% package and hardware interface.  Specific considerations involve (1)
% fast and robust reduction procedures (cf. MaNGA DOS) that can assess
% the aggregate data and (2) a responsive database with a schema
% optimized for real-time decision making to select targets for
% (re)acquisition while accounting for collision limitations.  Provided
% enough design effort, this lends itself to a machine-learning
% application.

\subsection{Publicly Available Automated Data Products}
\label{sec:DAP}
% \noindent \comment{1/2 page}

The typical proprietary period for raw data acquired at Keck is 18
months.  However, typical of other public surveys (e.g., SDSS), this
period will be shortened to one year for the FOBOS Public Survey.

Both as part of our design effort and for long-term use, we will develop
a data-reduction pipeline, building on work already done for other
fiber-based observations, like SDSS and DESI.  This software will
provide both the quick reduction assessments needed for our dynamic
targeting system and the more detailed reduction to produce the data for
scientific analysis.  Reduced data will be delivered to the community
(e.g., via the Keck Observatory Archive) after the proprietary periods
are finished for {\it both} PI-led and public survey observations.

Finally, we will also provide a data-analysis pipeline that provides
high-level data products.  There are two aspects to analysis of the
data.  First, we will provide software to perform the traditional
measurements of properties like Doppler shift, emission-line strengths,
and internal kinematics that are measured from FOBOS-observed spectra.
This software will build on existing software we have built for the
SDSS-IV MaNGA survey (Westfall et al.), and it will be executed for {\it
any} data taken with FOBOS and released along with the reduced spectra.
This is a substantial effort and unheard of for observations taken
outside of a large-scale survey effort.  Second, we will provide the
results of our various machine-learning applications in the FOBOS Public
Survey (e.g., the LSST-source redshifts as determined by the FOBOS
observed training set).

Important to the success of both the data-reduction and data-analysis
software will be PI and community involvement in their refinement to
meet the needs of specific science applications.  These software
packages will be open source and publicly served (e.g., using GitHub).

% Beyond the raw data, the survey will provide reduced and derived
% products immediately (cf. MaNGA DAP).  The latter will be true of both
% the data from the public survey (released immediately) and indeed {\it
% any} data taken with the FOBOS instrument after the nominal 18-month
% proprietary period.  For the latter, we will encourage involvement of
% the program PI in refining the data-reduction and data-analysis
% software and its execution to garner the most from its application to
% their data.  Community involvement in a common software development
% obviates the need for different groups to retread old ground.

\section{Broader Impacts}
\label{sec:bi}

"include a discussion of student training, increased participation of
underrepresented groups and a description of tangible benefits to the
wider U.S. research community (access, data products, technology,
etc.)."

\subsection{Akamai: Training the next generation of Hawaiian STEM
professionals} Led by the Institute for Scientist and Engineer Educators
(ISEE) at UCSC, the Akamai Internship Program is aimed at advancing
college students from Hawaii into the STEM workforce.  Almost 400
students have participated to date, of which 24\% are Native Hawaiian
and 38\% are women. A longitudinal study of Akamai outcomes indicated
that 87\% were still in STEM, either in the workforce or continuing STEM
studies \citep{asee_peer_31030}.  Traditionally, most Akamai interns are
pursuing engineering or computer science in their undergraduate
education.  ISEE and the Akamai program already have deep connections to
the W.~M.~Keck Observatory, involving many interns in projects related
to instrument development and observatory operations over the past 15
years.  We will fund two Akamai interns during our funding period.
Specific opportunities for their involvement include assessment of the
FOBOS instrument design via a software-based simulator that will
eventually result in a sensitivity calculator and the data-reduction
pipeline, as well as the machine-learning applications discussed
throughout our proposal.  At UCO, we have already worked with an Akamai
intern, who helped us setup a fiber test-bench at UCSC during Summer
2018. We are excited by the opportunity to include funding for these
interns as part of our proposal and continue to seek new opportunities
to generate connections with the current and future Hawaiian workforce.

\subsection{Investing in future educators} Also via the ISEE, we will
support three graduate students to participate in the Professional
Development Programs (PDP) that trains graduate students and supports
them in developing their teaching skills.  The PDP trains graduate
students and postdocs to collaboratively design a well-focused inquiry
activity within a small team.  The activity is conceived, developed, and
tested within the team, and the program culminates in the team running
the lab exercise with a group of undergraduates.  The program emphasizes
inclusive and equitable learning environments.  Specifically, our team
of graduate students will develop an inquiry based lab unit on
instrument development, data-reduction procedures, and/or
machine-learning methods, which will be taught to incoming transfer
students from community colleges.  In addition to training three
graduate students in inclusive pedagogy, our team will impact 25
undergraduates coming from California community colleges, a large
fraction from underrepresented minority groups.

\subsection{Undergraduate Student Training} Multiple avenues exist
within the current curriculum to include UC undergraduate students in
the development of FOBOS and its public survey.  At UCSC in particular,
we will guide freshman and first year transfer students through two
quarters in Astro 9, a recently started course that aims to introduce
scientific research method early in students' tenure via timely research
projects developed and led by UCSC graduate students, postdocs, and
staff.  Both PI Bundy and co-PI Westfall have been involved in projects
over the past two years, including a project to measure the rotation
curves of galaxies observed by the SDSS-IV MaNGA survey.  Introducing
undergraduates to astronomical instrumentation would be a unique
contribution to this course.

% "Preliminary proposals must include an outline of ongoing operations and
% maintenance plans, including an estimate of any needs for ongoing,
% NSF-supported operations and maintenance that may be requested outside
% of the Mid-scale RI program."

% "Results from Prior NSF Support should not be included. Also, links to
% URLs may not be used."

% \vspace{-0.5cm}

% {\bf no more than 2 pages for references}

\clearpage
\bibliographystyle{nsf}
\bibliography{references}


\newpage

% \noindent{\bf Budget and Budget Justification}

% "including budgets for any subawards. For preliminary proposals cost
% estimates may be preliminary estimates with the Basis of Estimates (BoE)
% included. Copies of vendor quotations should not be included in
% preliminary proposals. If the budget includes contingency, that
% contingency should cover the "known unknowns" and be used to mitigate
% identified risks."

% \newpage

\section{Facilities, Equipment, and Other Resources}

% In order for NSF, and its reviewers, to assess the scope of a proposed
% project, all organizational resources necessary for, and available to a
% project, must be described in this section of the proposal. Proposers
% should describe only those resources that are directly applicable. The
% description should be narrative in nature and must not include any
% quantifiable financial information. Proposers should include a
% description of the internal and external resources (both physical and
% personnel) that are expected to be available to the project.  Such
% information must be provided in this section, in lieu of other parts of
% the proposal (e.g., Budget Justification, Project Description).

\begin{figure}[h!]
 \vskip -0.1in
 \includegraphics[width=\textwidth]{fig_test_bench.png}
 \caption{\small The UCO Fiber Test Bench is being used to prototype fiber-lenslet coupling options and ensure high-throughput coupling at the Keck Focal plane.  The top panel shows a schematic diagram of the test bench, which allows for variable input broadband sources to be directly compared to output near- and far-field images generated after the input light passes through fiber assemblies.  The bottom panel is a photograph of the working test stand.   }\label{fig:testbench}
\end{figure}

\subsection{University of California Observatories}

University of California Observatories (UCO) manages a world-renown facility on the UC Santa Cruz campus for the
design, construction, and testing of astronomical instrumentation.  With a staff of leading optical designers,
engineers, and instrument scientists, UCO has a long heritage of producing state-of-the-art instrumentation, including
many spectrometers, as well as controls software for the Lick and Keck Observatories.  The recent delivery of K1DM3\footnote{K1DM3: Keck 1 Deployable Tertiary Mirror.} illustrates the close relationship between WMKO and UCO which allows us to leverage detailed knowledge of the observatory structure, protocols, interfaces, software and systems requirements, and instrument deliverables.

\subsection{UCO Fiber Test Bench}

UCO has built a precision fiber test bench (Figure \ref{fig:testbench}) which is currently being used to prototype lenslet coupling solutions and procedures and will serve as a valuable testing tool in the Keck-FOBOS preliminary design, especially for measuring the impact of fiber stress and motion on the focal ratio degradation and throughput variability.

\subsection{W.M.\ Keck Observatory}

WMKO has provided funding as well as technical guidance for initial stages of FOBOS development and its interface to the observatory.  The underside of the Keck Nasmyth deck (Figure \ref{fig:nasmyth_mount}) has been identified as the mounting point for Keck-FOBOS spectrographs, maintaining access space for other Keck instruments above.  Figure \ref{fig:keck_exchange} shows how railings allow instrument components, like the Keck-FOBOS focal plane system, to wheel up to the Nasmyth port or to wheel back in stow positions when not in use.


\begin{figure}[h!]
 \vskip -0.1in
 \includegraphics[width=\textwidth]{nasmyth_deck.png}
 \caption{\small The underside of the Keck Nasmyth deck where the Keck-FOBOS spectrographs will be mounted, with fiber runs feeding from the focal plane system above.  Other Keck instrument facilities (e.g., components of the adaptive optics laser system) have been successfully mounted in this location.  }\label{fig:nasmyth_mount}
\end{figure}

\begin{figure}[h!]
 \vskip -0.1in
 \includegraphics[width=\textwidth]{Keck_instrument_exchange.png}
 \caption{\small The Keck-FOBOS focal plane system would be mounted on railings to allow access to the Nasmyth port when its being used.  As with other instruments, it can be removed to a stow position on the extended service platform.  The photo on the left shows the Keck Cosmic Web Imager (KCWI) in the Nasmyth mounted position.  On the right, the Echellette Spectrograph and Imager (ESI) is sitting in its stow position. }\label{fig:keck_exchange}
\end{figure}

\subsection{Starbugs at the Australian Astronomical Observatory} % (fold)
\label{sec:AAO}

The Australian Astronomical Observatory (AAO) has worked with the Keck-FOBOS team during conceptual design to develop
designs for the CLADC and confirm the feasibility of Starbugs at the Keck-FOBOS focal plane.  The Starbugs fiber
positioners have been under development at AAO for nearly 15 years \citep[see][]{staszak16} and have recently been
deployed on-sky on a new instrument called TAIPAN.  Beginning science operations this year, TAIPAN
demonstrates that the level of maturity attained by the Starbugs technology makes it highly attractive to an instrument
like Keck-FOBOS beginning its preliminary design phase.

\begin{figure}[h!]
 \vskip -0.1in
 \includegraphics[width=\textwidth]{starbugs.png}
 \caption{\small Starbugs fiber positioning technology from AAO.  {\it Left:} An individual Starbug assembly where the concentric piezo cylinders (colored red) which establish a vacuum seal to the focal plane plate are evident with the central fiber payload housing (white).  The outer diameter is 8 mm.  A connector directs the fiber output to the next section of the cable run. {\it Right:} The picture shows the front face of several Starbugs attached to the TAIPAN focal plane.  Three green laser dots provide closed-loop feedback on each Starbug's position.  Both figures are from \citet{staszak16.} }\label{fig:starbugs}
\end{figure}




% subsection starbugs_at_the_australian_astronomical_obseravatory (end)

\subsection{Keck-FOBOS Conceptual Design}

Keck-FOBOS conceptual design work has proceeded in parallel with a fiber-based, Nasmyth-mounted instrument concept for the Thirty Meter Telescope (TMT) that shared many design elements in common.  A variety of related documents and reports are being produced in support of this effort:

\begin{itemize}
	\item Science Case Requirements Document: A funded effort to gather community feedback, science cases, and associated requirements has yielded the design specifications for Keck-FOBOS.  A report will be issued in summer 2019.

	\item Draft manuscript detailing requirements and demonstrating necessary performance based on existing instrumentation for faint-object fiber spectroscopy on large-aperture telescopes.

	\item Design specification documents and optical (Zemax) models for the CLADC, lenslet fiber fore-optics, spectrograph layout and cameras.

	\item Vendor quote estimates for the CLADC, Starbugs positioners, fiber run, dichroics, FSE gratings, catadioptric cameras, and detector systems.

	\item SolidWorks modeling of the focal plane system is nearly finished with spectrograph support and enclosure structures underway.  Likely vendors of focal plane motion systems identified.

	\item Detailed Microsoft Project plans through Preliminary Design and out to instrument deployment.  

	\item Risk matrix capturing 48 risk items and scoring their probability and consequence severity before and after mitigation strategies.

\end{itemize}

\newpage
\noindent{\bf Supplementary Documents:}

(to be entered in the Supplementary Documents section of FastLane)

\begin{enumerate}
%
\item A list of the major team members, their affiliations, and their
role in the project;
%
\item A list of Partner Organizations to be funded via subawards, and
the role of each in the project;
%
\item An outline of the Project Execution Plan (PEP).
%  see https://www.nsf.gov/bfa/lfo/lfo_documents.jsp
\end{enumerate}

% (See the LFM/MFG. Greater detail will be required in invited full
% proposals should that occur. See Full Proposal Preparation section for
% further information.)

%\begin{figure}[h!]
%  \vskip -0.1in
%  \includegraphics[width=\textwidth]{red_geyser_fig.pdf}
%  \caption{\small Discovery of the ``red geyser'' phenomenon in MaNGA which may
%      provide key missing evidence for how early-type galaxies remain quiescent.  Our prototypical example is the
%      elliptical galaxy (MaNGA ID 1-217022) pictured to the right in the SDSS image (top panel) and tidally interacting
%      with a low-mass companion.  The purple hexagons show the MaNGA integral-field footprint.  Bisymmetric features in
%      the H$\alpha$ equivalent width map (EW, central panel) suggest outflowing material.  A cool gas component traced
%      by Na D is offset, however, with material apparently falling into the galaxy's center (center-right panels).
%      Further evidence for an outflowing ionized gas is apparent emission line velocity field ($v_{\rm gas}$,
%      bottom-center panel).  Schematic diagrams of a wind model are shown to the left while the bottom-right panel
%      demonstrates how features in the $v_{\rm gas}$ field can be reproduced by the model.  The enhanced H$\alpha$
%      emission (over-plotted white contours) can be explained by shocks or over-densities occurring along the central
%      wind axis (highlighted in green in the wind drawings).  Other examples of MaNGA red geysers are shown in Figure
%      \ref{fig:montage}. }\label{fig:splash}
%\end{figure}


% \begin{enumerate}[rightmargin=0.2cm,leftmargin=0.2cm]
%
%  \item[] {\bf Preparatory Work}: Using a parent sample of 2400 MaNGA galaxies we apply a red sequence color cut of
%  $NUV-r>5$ \citep[see][]{salim09} which yields $\sim$1000 quiescent galaxies from which we visually identify 83 red
%  geyser candidates.  We have built a control sample of 442 galaxies by matching on redshift, $M_*$, and axis ratio
%  ($b/a$).  Stellar mass and redshift have been shown to correlate with radio emission and thus must be controlled for
%  \citep[e.g.,] []{best07}.  We additionally consider a control sub-sample with MaNGA-detected emission lines (LIER
%  galaxies).
%
%  We have downloaded FIRST radio image data and extracted cutouts centered on both the red geyser and control samples.
%   Among these we have identified radio detections and performed initial median stacks which are shown in Figure
%   \ref{fig:stacks}.  We find that 19/83 ($\approx23\%$) red geysers and 39/442 ($\approx9\%$) control galaxies are
%   radio-detected.  Our prelminary stacked photometry indicates that with detections removed, the radio flux associated
%   with red geysers is $\approx$7 times stronger than in the control samples.
%
%  \item[] {\bf Proposed Work}: Our preliminary analysis of the FIRST data points to an important result: radio
%  emission, most likely associated with AGN, appears to be elevated among red geysers bolstering our AGN wind
%  interpretation.  Here we propose to complete and publish this analysis.  We will update the red geyser sample using
%  the latest MaNGA catalog ($\sim$5000 galaxies, see Section \ref{sec:stats}) and more carefully vet the stacked
%  samples.  Using deep star formation rate estimates from \citet{chang15}, we will test that radio flux
%  contamination from residual star formation does not affect our conclusions.  Most important, we will develop a Monte
%  Carlo technique to measure the photometric errors on our stacked flux estimates.  This will enable us to assign a
%  confidence level to the significance of the stronger radio flux associated with red geysers.  
%

%\end{enumerate}


\includepdf[pages=-,landscape=true]{FOBOS_HighLevel_Schedule_v1.pdf}

\end{document}

Proto-galaxy:

    - Alice comments:
        - Better Figure 1
        - Show Fig. 2 from Steidel et al 2016, richness of rest UV
        - Show rest spectra from Kriek et al. 2015, sanders et al. 2015,
          steidel et al 2014.
        - lyman-continuum escape fraction references
        - Justify resolution requirements


    - George's comments:
        - fill out case for probing both galaxies and their “gas-filled
          environments”
            - make it more explicit that getting large numbers of
              redshifts would make it possible to trace out large-scale
              structure in detail
            - enables studies of galaxy properties as a function of
              environment
        - also mention targeting galaxies along QSO lines of sight
            - much higher target density than with LRIS, DEIMOS over
              larger FOV.
        - Worth discussing Lyman-alpha or metal-line tomography?  
        - More quantitative comparisons with existing data sets?
            - What key science questions can FOBOS address that many
              years of LRIS and DEIMOS observations have not been able
              to?  Surely some level of the spectral tagging and photo-z
              training can be done (and surely is being done) with
              existing data.  Is FOBOS going to be a huge leap, or will
              it mainly be cleaning up neglected corners of parameter
              space?
        - More excited to hear about how the FOBOS spectra will be used
          for science directly, instead of support for LSST

    
