%\documentclass[11pt,letterpaper]{article}
\documentclass[oneside,11pt]{amsart}

%\usepackage{a4wide}
%\usepackage{epsfig}
%\usepackage{psfig}
\usepackage{graphicx}
\usepackage{natbib,latexsym,url,enumitem,pdfpages}
\usepackage{color}
\usepackage{wrapfig}
\usepackage{caption}

\captionsetup{
    justification=justified,
    margin=0pt,
    font=small}

% Some fancy commenting
\definecolor{todo}{RGB}{200,0,0}
\newcommand{\comment}[2][todo]{{\color{#1}[[{\bf #2}]]}}

% Challenge counter
\newcounter{chalno}
\newcommand{\chal}[1]{\refstepcounter{chalno}\label{#1}}

% User commands
\input{journaldefs}

\DeclareRobustCommand{\gtrsim}{%
\mathrel{\hskip-.5em\begin{array}{c}>\\[-8pt]\sim\end{array}\hskip-.5em}}
\DeclareRobustCommand{\lesssim}{%
\mathrel{\hskip-.5em\begin{array}{c}<\\[-8pt]\sim\end{array}\hskip-.5em}}

\pretolerance=10000
\textwidth=6.4in
\textheight=8.95in
\voffset = 0.in
%\voffset = -0.3in  % For my printer
\topmargin=0.0in
\headheight=0.00in
\hoffset = 0.0in
%\hoffset = 0.33in  %  For my printer
\headsep=0.00in
\oddsidemargin=0in
\evensidemargin=0in
\parindent=2em
\parskip=0.2ex
 
\renewcommand{\baselinestretch}{1.03}

\special{papersize=8.5in,11in}

\newcommand{\markus}{\textcolor{green}}

\setlength{\parskip}{0.6 ex plus 0.4ex minus 0.2ex} \flushbottom
\pagestyle{plain} 

% From the call:
%
% For Phase A system design proposals, the project description should
% be no more than 10 pages devoted to science cases, conceptual
% designs, and a preliminary budget and schedule for the full build of
% the instrument. Up to two additional pages are allowed for the Phase
% A system design budget, milestones, deliverables, and references for
% the proposed work. For both proposal types, the requested budget
% should be in summary form identifying how the money will be spent on
% major study costs. To gain an initial understanding of technical
% issues such as existing instrument configurations, observatory
% interfaces, and guidelines for the standard WMKO instrument
% development process, proposers are strongly encouraged to contact the
% WMKO Instrument Program Manager, Marc Kassis.

\begin{document}
% \thispagestyle{empty}

\pagenumbering{arabic}

\vspace*{-1.5cm}

\centerline{\includegraphics[width=0.5\textwidth]{figs/FOBOS_inst_v2.pdf}}
\centerline{\textsc {\Large FOBOS: The Fiber-Optic Broadband Optical Spectrograph}}
\smallskip
\centerline{\large\it Keck's next-generation spectroscopic facility}

%% Executive Summary and Overview
%%%%
% -- Overview Material
% --     FOBOS Keck White Paper 2019
%%%%

\centerline{{\large\bf Executive Summary}}

% Soon, our community will be inundated with sources of interest from
% large-scale ground- and space-based surveys requiring spectroscopic
% follow-up for physical characterization. In fact, with the recent
% {\it Gaia} releases, the spectroscopic needs in Galactic Astronomy
% are already dire, a situation that will be acutely felt by
% extragalactic community once LSST, Euclid, and WFIRST begin
% delivering data. These sources will span the full range of scientific
% interests of our current Keck partners and Keck's current
% spectroscopic capabilities will not e sufficient to meet these needs.

High-multiplex and deep spectroscopic followup of LSST and other
panoramic deep-imaging surveys is a widely recognized necessity.
Reports in 2015 and 2016 by the National Research Council and
National Optical Astronomical Observatory specifically recommended
investment in new spectroscopic facilities to meet these needs
because none currently exist or are planned for U.S.\ observatories.
We seek seed funding from WMKO to continue conceptual design of
FOBOS, a powerful new spectrograph for the Keck II telescope that increases its current survey speed by a factor of X.

Led by NSF's Large Synoptic Survey Telescope\footnote{
%
LSST will be begin science operations in 2023.}
%
(LSST), astronomy is entering a new era of unprecedented deep-imaging data sets that will survey huge volumes of the
Universe.  From the emergence of the earliest galaxies from a ``primordial soup'' of gas and dust, to the peak of
cosmic star formation and the current era of accelerated expansion, these surveys will provide unprecedented statistics
at key epochs of cosmic history.


Even so, gaining physical insight from panoramic imaging surveys will require intensive spectroscopic follow-up.  The
power of combining photometry and dedicated spectroscopy is widely appreciated and perhaps best illustrated by the
success of the Sloan Digital Sky Survey (SDSS) which used this combination to record the properties of over 1 million
galaxies, mapping the present-day universe and making SDSS one of the most highly cited surveys in the history of
astronomy.

LSST's all-sky images will be 1,000 times deeper and detect far more
distant galaxies than SDSS, but \textbf{no current U.S. facility is
capable of obtaining spectroscopic follow-up of LSST galaxies} at a level
required to capitalize on the \$1B the U.S.\ has invested in that
project.  In fact, an SDSS-like spectroscopic study of 1 million
galaxies at LSST depth would require 300 years of observing on the
largest telescopes with current instrumentation!  

Solving this problem requires not only more powerful spectroscopic facilities, but new ways to take better advantage of
what will be necessarily more limited spectroscopy compared to the vast imaging surveys of the LSST era.  The more
promising path is encapsulated in one of NSF's ``10 Big Ideas,'' \emph{Harnessing the Data Revolution}: we can maximize
the information content of LSST and other imaging facilities via machine learning from optimally designed spectroscopic
training sets.

This proposal presents a coordinated framework with three critical
components necessary for success in this endeavor: 1) Using simulated
spectroscopic$+$imaging data to define the training sets required to
address ambitious data-science challenges in Cosmology, Galaxy
Formation, and Local Group Archaeology in the LSST era; 2) Preliminary
design of FOBOS,\footnote{FOBOS: Fiber-Optic Broadband Optical
Spectrograph} a state-of-the-art spectroscopic facility for
WMKO,\footnote{WMKO: W.~M.\ Keck Observatory operates the two twin 10m
Keck Telescopes on Maunakea, Hawaii.} optimized for providing critical
training data using one of the world's largest telescopes; 3)
Preliminary design of the coordinated FOBOS observations required as
well as the systems needed to serve training data publicly.  This {\bf
MSRI-1 Design} proposal lays out the path for maximizing panoramic
imaging from LSST, WFIRST,\footnote{WFIRST is NASA's space-based
Wide-Field Infrared Survey Telescope, expected to launch in the mid
2020's.} Euclid,\footnote{Euclid is led by the European Space Agency
with significant NASA involvement and will launch in 2021. Its primary
mission is a 15,000 deg$^2$ imaging survey in optical and near-IR
wavebands.} and other facilities with spectroscopic follow-up
unparalleled in depth and sampling density.  Through a subsequent MSRI
proposal we will deliver on our goals with an instrument deployment in
2026, an array of spectroscopic programs, and associated public-ready
training data.



Led by NSF's Large Synoptic Survey Telescope\footnote{
%
LSST will be begin science operations in 2023.}
%
(LSST), astronomy is entering a new era of unprecedented deep-imaging data sets that will survey huge volumes of the
Universe.  From the emergence of the earliest galaxies from a ``primordial soup'' of gas and dust, to the peak of
cosmic star formation and the current era of accelerated expansion, these surveys will provide unprecedented statistics
at key epochs of cosmic history.

% Meanwhile, the rate of cosmic expansion was beginning to accelerate,
% as the Universe became increasingly dominated by ``Dark Energy,''
% whose origin remains the single greatest mystery in astronomy and
% cosmology today.

% Since Edwin Hubble's observations over 100 years ago,

Even so, gaining physical insight from panoramic imaging surveys will require intensive spectroscopic follow-up.  The
power of combining photometry and dedicated spectroscopy is widely appreciated and perhaps best illustrated by the
success of the Sloan Digital Sky Survey (SDSS) which used this combination to record the properties of over 1 million
galaxies, mapping the present-day universe and making SDSS one of the most highly cited surveys in the history of
astronomy.

% Because a quality spectrum requires far more observing time per source
% than an image, SDSS pioneered ``high multiplex'' spectrographs,
% capable of \emph{simultaneous} spectroscopy of hundreds of objects.

LSST's all-sky images will be 1,000 times deeper and detect far more
distant galaxies than SDSS, but \textbf{no current U.S. facility is
capable of obtaining spectroscopic follow-up of LSST galaxies} at a level
required to capitalize on the \$1B the U.S.\ has invested in that
project.  In fact, an SDSS-like spectroscopic study of 1 million
galaxies at LSST depth would require 300 years of observing on the
largest telescopes with current instrumentation!  

Solving this problem requires not only more powerful spectroscopic facilities, but new ways to take better advantage of
what will be necessarily more limited spectroscopy compared to the vast imaging surveys of the LSST era.  The more
promising path is encapsulated in one of NSF's ``10 Big Ideas,'' \emph{Harnessing the Data Revolution}: we can maximize
the information content of LSST and other imaging facilities via machine learning from optimally designed spectroscopic
training sets.

This proposal presents a coordinated framework with three critical
components necessary for success in this endeavor: 1) Using simulated
spectroscopic$+$imaging data to define the training sets required to
address ambitious data-science challenges in Cosmology, Galaxy
Formation, and Local Group Archaeology in the LSST era; 2) Preliminary
design of FOBOS,\footnote{FOBOS: Fiber-Optic Broadband Optical
Spectrograph} a state-of-the-art spectroscopic facility for
WMKO,\footnote{WMKO: W.~M.\ Keck Observatory operates the two twin 10m
Keck Telescopes on Maunakea, Hawaii.} optimized for providing critical
training data using one of the world's largest telescopes; 3)
Preliminary design of the coordinated FOBOS observations required as
well as the systems needed to serve training data publicly.  This {\bf
MSRI-1 Design} proposal lays out the path for maximizing panoramic
imaging from LSST, WFIRST,\footnote{WFIRST is NASA's space-based
Wide-Field Infrared Survey Telescope, expected to launch in the mid
2020's.} Euclid,\footnote{Euclid is led by the European Space Agency
with significant NASA involvement and will launch in 2021. Its primary
mission is a 15,000 deg$^2$ imaging survey in optical and near-IR
wavebands.} and other facilities with spectroscopic follow-up
unparalleled in depth and sampling density.  Through a subsequent MSRI
proposal we will deliver on our goals with an instrument deployment in
2026, an array of spectroscopic programs, and associated public-ready
training data.

The need for spectroscopic follow-up in the LSST era was made clear in
the National Research Council's 2015 report, ``Optimizing the U.S.
Ground-Based Optical and Infrared Astronomy System'' \citep{NAP21722}:
%
\noindent\begin{center}\mbox{\parbox{0.95\linewidth}{
%
The National Science Foundation should support the development of a
wide-field, highly multiplexed spectroscopic capability on a medium- or
large-aperture telescope in the Southern Hemisphere to enable a wide
variety of science, including follow-up spectroscopy of Large Synoptic
Survey Telescope targets. Examples of enabled science are studies of
cosmology, galaxy evolution, quasars, and the Milky Way.
%
}}\end{center}

Workshops organized by the National Optical Astronomy Observatory (NOAO)
in 2013 and 2016, the latter at the NSF's request, reported specific
spectroscopic needs for LSST follow-up in all science areas.  In
particular, the 2016 report notes that a critical resource in need of
prompt development is to ``Develop or obtain access to a highly
multiplexed, wide-field optical multi-object spectroscopic capability on
an 8m-class telescope.''  Based on these recommendations, we propose the
FOBOS instrument coupled with a suite of data-driven tools to address
the spectroscopic requirements of LSST and other photometric surveys at
a final cost 20 times less than a new Southern Hemisphere facility.
Located in Hawaii, FOBOS can access more than 70\% of the LSST
footprint, more than adequate for building powerful
spectroscopic training sets.  Compared to the Prime Focus Spectrograph
(PFS) on Japan's Subaru Telescope, FOBOS would be 1.7$\times$ faster,
provide unique UV sensitivity (0.31--1 $\mu$m compared to
0.38--1.25 $\mu$m with PFS), and offer higher-density, more flexible
target sampling over ``deep-drilling'' fields.  Unlike PFS, FOBOS would be operated
on a U.S.\ telescope with dedicated U.S.\ access and a commitment to
supporting U.S.-led imaging facilities.  FOBOS is also complementary to
future telescopes and instruments that would be optimized to cover wider areas
(several deg$^2$ per pointing) at shallower depths.

%\comment{mention FOBOS can do PI-led science too}

The need for deep spectroscopic follow-up is particularly acute for the major cosmological probes to be carried out by
LSST, Euclid, and WFIRST, which all rely on ``photometric redshifts:'' measures of galaxy redshift, $z$
--- a direct proxy of distance and look-back time---based on imaging alone. \citet{newman15} summarize the case for a
    significant spectroscopic campaign to calibrate and train LSST photometric redshifts in order to improve cosmological constraints.  They describe a redshift survey that,
    if carried out with FOBOS, would increase LSST's Dark Energy figure-of-merit by 40\% at a cost of less than 5\% of
    the LSST budget.  The urgent case for spectroscopic redshift training has been the subject of numerous publications
    \citep[e.g.,][]{laureijs11, masters15, hemmati18}.

Meanwhile, the astronomy community recognizes that the coming ``Big
Data'' era, culminating in LSST, necessitates ``\textbf{harnessing the
data revolution}.''  Widespread community interest in advanced
data-science techniques continues to grow amidst calls for educational
programs, conference series, and research funding to support the growth
of a new field, ``Astroinformatics,'' which exploits the interface
between astrophysics and statistics \citep{borne09}.  Astronomy's
largest organizations, including the American Astronomical Society and
the International Astronomical Union, have supported active working
groups on astroinformatics and astrostatistics since 2015.  LSST itself
has supported the Informatics and Statistics Science Collaboration and
partnered with NSF on the Data Science Fellowship Program to train
astronomy graduate students in data-science techniques.  Our proposal
builds on and contributes to these ongoing efforts.



%-----------------------------------------------------------------------
% Section 1
\section{Science Drivers}
\label{sec:goals}

Keck$+$FOBOS occupies a unique space among existing and forthcoming
facilities for optical spectroscopy in the 8--10m class.
Specifically, no other facility can match its instantaneous
multi-aperture sampling density or UV sensitivity. In terms of
spectral range, FOBOS will be unique in its sensitivity to
wavelengths down to the atmospheric cut-off, maintaining a key
advantage of Keck spectroscopy. Additionally, FOBOS will provide
three sampling modes: single-fiber apertures, multiple mini-IFU
aperture arrays, and a single monolithic IFU. Finally, given the
quick reconfiguration time and flexibility of the Starbugs
positioning system, FOBOS presents exciting opportunities to
simultaneously deliver spectroscopy for multiple science programs.
Although alreaady achieved with existing Keck instruments
\citep[e.g., Halo7D with DEIMOS;][]{2018arXiv180904082C}, the
integration of multiple science programs with FOBOS will lose far
less to management overhead and be able to respond dynamically to
observatory conditions {\it on the night(s) data are taken}. The
science cases below take advantage of these unique aspects of FOBOS.

% For example, although Subaru-PFS has a higher multiplex, the
% combination of its larger FOV and positioning system yield an
% instantaneous targeting density that is an order of magnitude below
% that of Keck-FOBOS.

%%%%
% -- Local Group Science Cases
% --     FOBOS Keck White Paper 2019
%%%%

\subsection{Unraveling the Formation History of our Local Group of Galaxies}
\label{sec:localgroup}

Our Local Group of galaxies --- the Milky Way (MW), the Magellanic
Clouds, Andromeda (M31), Triangulum (M33), and numerous 
satellite galaxies --- allows us to study one realization of the
galaxy-formation process in superb detail. In the next decade, LSST
and WFIRST will increase the census of stellar streams and halo
substructure in these galaxies by a hundredfold. Follow-up
\emph{stellar} spectroscopy will constrain stream orbits and the
total mass they enclose \citep{2017ApJ...836..234S}, as well as the
associated age and chemical composition (see below).

\noindent\comment{Weisz: Dwarf galaxies?}

\noindent\comment{Guhathakurta: M31?}

\noindent\comment{Rockosi: Milkway Halo \& DESI connection?}


% \chal{mwhalo} 
% %
% \item[] {\textsf {\large  Data-Science Challenge \ref{mwhalo}: The
% chemical evolution and assembly history of the MW stellar halo.}}  Using current MW halo models, we will simulate FOBOS
% stellar spectroscopy of main-sequence turn-off
% and red-giant stars in these substructures within the MW that also
% leverages existing data from, e.g., APOGEE and H3.  We will build
% data-driven models based on these data to measure stellar parameters
% (temperature, surface gravity, metallicity, and alpha-element abundance)
% for all halo stars with LSST+2MASS+WISE+WFIRST multi-band photometry,
% allowing us to reconstruct the star-formation history of each disrupted
% satellite. These will be combined with dynamical data and compared with
% cosmological simulations to build a generative model for the assembly
% history of the MW stellar halo.

% \chal{m31} 
% %
% \item[] {\textsf {\large Data-Science Challenge \ref{m31}: The
% differential chemical evolution of M31 and MW.}}  A natural extension of
% Data-Science Challenge \ref{mwhalo} is to perform the same analysis for the
% halo of M31.  However, we cannot expect to obtain high-quality spectra
% of individual main-sequence stars at the distance of M31 with FOBOS.
% Moreover, training a chemical evolution model using spectra of Milky Way
% stars may lead to systematic errors:  The Milky Way and Andromeda have
% distinct evolutionary histories \citep[e.g.][]{2005MNRAS.356.1071R},
% despite being relatively similar in many other respects.  We will
% therefore obtain deep observations of giant stars in the M31 halo to
% drive a machine-learning algorithm that combines a model of the MW halo
% with results from cosmological hydrodynamical simulations to constrain
% the differential history of the MW and M31 stellar halos.

% \chal{gaia} 
% %
% \item[] {\textsf {\large Data-Science Challenge \ref{gaia}: Stellar
% parameter determinations for a billion stellar spectra.}} While
% providing on-sky motions and photometry for 1.7 billion stars in the MW,
% fewer than 10\%, 0.3\%, and 0.1\% of stars will have a full complement
% of astrometrics and kinematics, basic stellar parameters, and chemical
% abundances, respectively.  Moreover, Gaia distance errors increase
% quadratically with distance.  To realize Gaia's full potential, we will
% design FOBOS training sets that, when combined with high-resolution
% datasets from, e.g., APOGEE, WEAVE, will allow us to build data-driven
% models of the absolute magnitude (yielding distance modulus),
% temperature, surface-gravity, and stellar abundance for {\it all} stars
% in the Gaia dataset.  These data will allow us to isolate coeval
% populations in the Galactic disk that can be combined with very
% high-resolution simulations of the Milky Way to provide a detailed
% evolutionary history of our Galactic home.


%%%%
% -- Galaxies Science Cases
% --     FOBOS Keck White Paper 2019
%%%%

\subsection{Dynamical masses of galaxies large and small}

Using point sources at large radius one can obtain the mass profiles of galaxies well into the regime where the mass is dominated by the dark-matter halo.  FOBOS makes these observations more than order of magnitude more efficient than can be currently done with DEIMOS, both because of the larger FOV and the higher multiplex. 

\noindent\comment{Alabi, Romanowsky -- Elliptical + PNe, GCs
dynamics, ultra faints?; or scratch that for something more recent}

\subsection{The volume density and chemistry of gas in between galaxies}

Not sure if there's anything new here.

\noindent\comment{X, Joe B.: Halos?  Anything new here}

\noindent\comment{KG, Joe H. - IGM tomography}

\subsection{The Proto-galaxy Ecosystem at $z$$\sim$2}
\label{sec:galaxies}

% From George:
% - fill out case for probing both galaxies and their “gas-filled
%   environments”
%    - make it more explicit that getting large numbers of redshifts
%      would make it possible to trace out large-scale structure in
%      detail
%    - enables studies of galaxy properties as a function of environment
%
% - also mention targeting galaxies along QSO lines of sight
%    - much higher target density than with LRIS, DEIMOS over larger FOV.
%
% - Worth discussing Lyman-alpha or metal-line tomography?  
%
% - More quantitative comparisons with existing data sets?
%    - What key science questions can FOBOS address that many years of
%      LRIS and DEIMOS observations have not been able to?  Surely some
%      level of the spectral tagging and photo-z training can be done
%      (and surely is being done) with existing data.  Is FOBOS going to
%      be a huge leap, or will it mainly be cleaning up neglected corners
%      of parameter space?
%
% - More excited to hear about how the FOBOS spectra will be used for
%   science directly, instead of support for LSST

%-----------------------------------------------------------------------

Understand the $z \sim 2$ galaxy ``ecosystem,'' including not only
the galaxies themselves but their gas-filled environments. The goal
is to build a comprehensive picture of the physical processes that
fuel proto-galaxy growth, shape their internal structure, and
influence their environment.

\noindent\comment{Cooper?} Build SDSS-like statistics for galaxies at
this key cosmic epoch. Exploit short spectroscopic exposures in
combination with photometry to provide environmental diagnostics for
1M galaxies at $z$=1--2. Photometric redshifts, while acceptable in
large cosmological analyses, wash out information about the local
position of galaxies with respect to one another. To characterize a
galaxy's local environment and identify its neighbors requires
(observationally expensive) spectroscopic redshifts. However, with
improved photo-$z$s available from Challenge \ref{photozs} and strong
priors on spectral types (Challenge \ref{phot}), the challenge here
is to push machine-learning techniques to deliver
\emph{spectroscopic} redshifts (with 300 km s$^{-1}$ accuracy) at the
lowest signal-to-noise possible. Reductions by factors of 4--5 in
exposure time would enable FOBOS to complete a 1M galaxy environment
survey at $z=1$--$2$ in just 20-30 nights.

\noindent\comment{Westfall, Bundy, Max -- Resolved spectroscopy}

\noindent\comment{Siana -- Lensed galaxies behing clusters}

\noindent\comment{Shapley, Siana: Additional section on Lyman-alpha continuum a $z$$\sim$3?}


%%%%
% -- Cosmology Science
% --     FOBOS Keck White Paper 2019
%%%%

\subsection{Enhancing Dark Energy Probes via Precision Cosmic Distances.}
\label{sec:cosmology}

Panoramic imaging surveys --- culminating in LSST, Euclid, and WFIRST --- are seeking to constrain the dark-energy
equation-of-state at $z \lesssim 1$ through measurements of angular correlations of galaxy positions, their
gravitational lensing shear, and the cross-correlation between the two.  These surveys rely on photometric redshifts
(``photo-$z$s''), whose uncertainties and potential biases are the major limitation and source of systematic error in
these efforts.  \citet{newman15} define a \emph{spectroscopic} survey for photo-$z$ training that would \emph{increase
the dark energy figure-of-merit in LSST by 40\%}.  The survey program is ideally matched to FOBOS.  It requires 10
independent fields, each 20 arcmin in diameter, with a sampling density of 6 arcmin$^{-2}$, and the ability to go very
deep ($i_{\rm AB} < 25.3$).  FOBOS's lack of a ``redshift desert'' further eliminates the need for expensive, space-based\footnote{Ground-based near-IR spectroscopy is too contaminated by
sky-line emission to provide spec-$z$s at the required level of completeness \citep{newman15}.} near-IR spectroscopy to train photo-$z$s with $z > 1.5$.  Highly accurate photo-$z$s will enable science applications with massive imaging surveys that go beyond cosmology.

\subsection{Cosmology with LBG--CMB cross correlation.}
\label{sec:LBG}

High-S/N CMB maps from next-generation CMB observatories (e.g., Simons Observatory and CMB-S4) will provide a cosmic
``reference background'' for measurements of gravitational lensing induced by matter along the line of sight.  After
cross-correlating with Lyman Break Galaxy (LBG) samples, a relatively flat lensing ``kernel'' with power at $z = 2$--5
enables powerful constraints on the Inflation-sensitive matter power spectrum, Horizon-scale General Relativity, cosmic
curvature and neutrino masses, and early Dark Energy.  \citet{wilson19} explore these constraints in detail and
highlight the need for spectroscopic determination of accurate redshift distributions for the employed LBG samples.
FOBOS would address this need in two ways.  First, several deep-drilling fields targeting $\sim$1000 LBGs BX, $u$, $g$,
and $r$ drop-out candidates per pointing ($\sim$10,000 deg$^{-2}$) would establish the interloper rate and intrinsic
redshift distribution of LBG samples to sufficient precision (this program would likely overlap with the photo-$z$
program described above).  Second, $\sim$200 LBGs per pointing (2000 deg$^{-2}$) could be included as a background
program when FOBOS observes other sources across the sky, eventually building a 50-100 deg$^2$ data set of sparse
high-$z$ spectroscopy for LBG dN/d$z$ calibration via clustering redshifts \citep[see][]{wilson19}.


\subsection{Kinematic weak lensing.}
\label{sec:kinematic_lensing}

Kinematic weak lensing is a promising new technique that measures shear in projected velocity fields of source galaxies to infer the presence of mass along the line-of-sight.  While compared to traditional photometric lensing, kinematic lensing requires more expensive resolved spectroscopy, it also yields shear measurements with 10 times the S/N per source galaxy \citep{huff13}.  This is because the result of velocity field lensing is uniquely defined whereas photometric lensing depends on the (unknown) intrinsic shape and orientation of the galaxy.  The kinematic lensing observable can be expressed as the offset between the kinematic major axis position angle (PA) and the photometric PA.  Small, 7-fiber IFU bundles can measure kinematic PAs to $\sim$1 degree precision (DiGiorgio in prep.), motivating a FOBOS deployment of 250 such IFUs per field that could measure mass profiles of galaxy cluster halos, perform targeted galaxy-galaxy lensing experiments, or provide checks on cosmological photometric lensing results from surveys like LSST.  Further work on developing this science case is underway. 




%%%%
% -- Data Science
% --     FOBOS Keck White Paper 2019
%%%%

\subsubsection{FOBOS as an ideal spectroscopic training instrument}
\label{sec:datascience}

\begin{figure}[h!]
%
\vskip -0.1in
%
\includegraphics[width=\textwidth]{figs/LGplots.pdf}
%
\caption{{\it Left}: Validation of {\it The Cannon} measurements of
stellar effective temperature, $T_{\rm eff}$, and surface gravity, $\log
g$, using low-resolution LAMOST spectra (left) compared to
high-resolution APOGEE measurements
\citep[right;][]{2017ApJ...836....5H}. {\it Top-right}: Recovery of
elemental abundances from low-resolution LAMOST spectra compared to
high-resolution measurements from GALAH (Xiang et al., in prep).  {\it
Bottom-right}: The circular-speed curve of the Milky Way determined
using a data-driven model that combines stellar parameters determined
from APOGEE spectra with photometry from WISE, 2MASS, and Gaia, yielding
the most precise measurements to date \citep{2019ApJ...871..120E}.}
%
\label{fig:Cannon}
%
\end{figure}

Radial velocity studies of stars in the MW halo or the M31 disk require
observations of up to 10 hours on large telescopes
\citep[e.g.,][]{2018arXiv180904082C}.  This again motivates
machine-learning algorithms to extract physical quantities from both
multi-band imaging and lower quality spectra (low resolution and S/N)
using relatively small, yet high-S/N, training sets.  For example,
\citet{2015ApJ...808...16N} have developed {\it The Cannon}, a
supervised learning approach that uses spectra with known stellar
parameters to label spectra where those parameters are unknown
(Fig.~\ref{fig:Cannon}).  Additionally, \citet{2018arXiv180401530T} have
developed {\it The Payne} which can infer 16 stellar-abundance labels
from low-resolution spectra using a neural network and theoretical
stellar spectra.  Finally, \citet{2018arXiv180803278T} have combined
Kepler-based astroseismology measurements with APOGEE spectra to
determine stellar age to $\sim$25\% precision using a neural network.
Our proposed effort builds on new lines of inquiry based on these
successes.


 A nested network of stellar parameter training
samples for resolved Milky Way and Local Group studies via extracting maximum information from photometry, in this case stellar parameters.  Our goal is to reach
magnitudes significantly fainter than the detection limit of current and upcoming spectroscopic surveys of the Milky
Way including Gaia, APOGEE,\footnote{APOGEE, the Apache Point Observatory Galaxy Evolution Experiment has observed in
both SDSS-III and SDSS-IV.} the SDSS-V Milky Way Mapper, planned programs with 4MOST\footnote{4MOST: 4-meter
Multi-object Spectroscopic Telescope.} and the Dark Energy Spectroscopic Instrument (DESI) Milky Way Survey, among
others. Inferring
stellar parameters beyond V$\sim$18 will open up studies of the Milk Way's outer halo, the halo of M31, and stellar
populations in local dwarf galaxies.

The immediate challenge is to design an optimized, nested set of training samples that connect data from the surveys
above.  This nested set will span high-S/N to low-S/N and high spectral resolution to low spectral resolution for
sufficiently large, overlapping stellar samples.  Subsets will have astroseismology from TESS\footnote{TESS is NASA's
Transiting Exoplanet Survey Satellite.} and PLATO.\footnote{PLATO is ESA's PLAnetary Transits and Oscillations
mission.}  Using simulated spectra with known input parameters, we will test methods for ``label transfer'' from
information-rich spectra to information-poor spectra as we work down to fainter magnitudes, landing eventually at
multi-band photometry alone. Within this nested set, low-resolution FOBOS data will fill in gaps at both high-S/N,
where we will be training FOBOS data on higher resolution spectroscopy, as well as lower-S/N where we will be training
photometry on FOBOS spectroscopy.  The success of this multi-layered label transfer depends not only on the size of the
training sets we can access or observe, but on how representative they are.  Label transfer to WFIRST imaging of the
M31 halo, or Local Group dwarfs in either hemisphere, is a particular concern.  We will test it by evaluating label
recovery on simulated stellar spectra with cosmologically-informed formation histories for M31 and dwarf galaxies,
suitably differentiated from the Milky Way stars that anchor the training network.


\subsection{Addressing Data Science Challenges and Designing FOBOS Training Sets}
\label{sec:survey}


Our team includes leading experts on data science applications to
astronomy and, specifically, LSST.  We will also use our established
connections to LSST's Informatics and Statistics Science Collaboration
(ISSC) to advertise, recruit, and coordinate efforts to tackle the Data
Science Challenges described in Section \ref{sec:goals}.  Our proposal
request includes two community workshops to motivate progress and discuss
results. At the end of the proposal period, we will publish the results
and developed software packages.

Our data-science challenges require work on simulated
imaging$+$spectroscopic data sets where input physical properties (e.g.,
redshift) can be compared to output recovered values.  Simulated imaging
data (e.g., from LSST and WFIRST) are in-hand, while mock spectroscopy
will be provided by a FOBOS instrument simulator, an initial version of
which has already been developed.  Further advances to be supported by
this proposal include improved error modeling and simulating systematic
effects from detector artifacts, image quality aberrations informed by
the emerging detailed optical design, and variable observing conditions.

The resulting success in addressing each data-science challenge will
define a level of readiness and set requirements on desired FOBOS
training sets, including number of sources, pointings, magnitude limits,
signal-to-noise thresholds, and observing conditions.  Preliminary
observing design and a description of required operational modes to
efficiently observe these training sets will begin with this proposal.
Operational modes will set requirements on target aggregation and
prioritization systems, field acquisition speed, field rotation range,
zenith avoidance zone, reconfiguration time, calibrations, read-out
time, quick-look reduction software and processing rates.  We will
develop integrated program concepts that efficiently combine required
observations.  Detailed survey and execution plans will be completed in
the next phase of this project (MSRI-2).  Roughly 20\% of Keck observing
time is open to the public, and as in previous federally-funded
projects, we fully expect that Senior Personnel at Keck institutions
will be successful in collaborative efforts to secure significant
amounts of additional telescope observing time to enable rapid, public
release of FOBOS training data \citep[e.g.,][]{newman13}.

\begin{enumerate}[rightmargin=0.2cm,leftmargin=0.2cm]
%
\chal{phot}
%
\item[] {\textsf {\large Data-Science Challenge \ref{phot}: Apply deep-
learning algorithms to infer physical properties of galaxies at
$z$$\sim$2 using using photometry.}} The range of observed spectral
types is well-constrained by broad-band imaging (Figure \ref{fig:SOM}),
suggesting a far greater potential for imaging data to reveal physical
properties with sufficient training than conventional modeling of
spectral energy distributions (SEDs) would suggest.  The challenge here is to identify the extent to which machine
learning can deliver SDSS-like information --- e.g., star-formation histories,
stellar-population properties, dust content, inflow/outflow properties,
and stellar masses --- and determine design parameters for future training sets that will enable such inferences for millions of imaged galaxies at $z$$\sim$2.
%
\end{enumerate}

\begin{figure}[h!]
%
\vskip -0.1in
%
\includegraphics[width=\textwidth]{figs/Hemmati18_Fig8_VVDS_spec.png}
%
\caption{\small {\it Left}: A Self-Organizing Map
\citep[SOM;][]{1990Natur.346...24K} from \citet{hemmati18} encoding the
relation between colors in an LSST+WFIRST-like color space and redshift,
$z$.  Position in the SOM is associated with a position in the
multi-dimensional broad-band color space of galaxies.  Galaxies observed
in this space are assigned $z$ values based on the median photo-$z$ of
galaxies from the CANDELS survey \citep[color
bar;][]{2011ApJS..197...35G}.  Such SOMs can be used to optimally define
spectroscopic training samples for use with imaging surveys.  {\it
Right}: Galaxy spectra from VVDS \citep{2005A&A...439..845L}; black
crosses near the top and bottom of the SOM are plotted in the top and
bottom panels, respectively.  Note the similarity of the high-resolution
spectra associated within the SOM, suggesting that a systematic
spectroscopic exploration of the LSST color space would have
far-reaching benefits to the science return of the mission beyond the
photo-$z$ application.}
%
\label{fig:SOM}
%
Self-Organizing Maps (SOM, Fig.~\ref {fig:SOM}) provide a
state-of-the-art representation of a high-dimensional input space in
projected 2D grid cells, allowing us to benchmark sampling of the
photometric color space under various training set designs.  We will use
Bayesian Optimization techniques to evaluate the success of simulated training sets against the fidelity of full  
cosmological analyses that employ them.  This will enable extremely rapid exploration of the
optimal design space.

\end{figure}

% The complete photo-$z$ training survey described in \citet{newman15}
% would require 15 independent pointings, each spanning 0.1 deg$^2$ with
% a target density of 6 arcmin$^{-2}$ (8 arcmin$^{-2}$ when including $z
% > 1.5$ galaxies accessible in the UV with Keck-FOBOS), perfectly
% matched to the Keck-FOBOS field-of-view and target density.  With a
% conservative exposure time of 100 hours to reach 75\% redshift
% completeness for 40,000 galaxies with $i_{\rm AB} < 25.3$, the Neman
% survey would require 400 nights.  Challenge \ref{photoz} would reduce
% the required survey duration by a factor of at least four.  Meanwhile
% the extreme depths and flux-limited selection are likely also
% requirements for training sets associated with Challenges \ref{phot},
% \ref{uv}.

% A wider and shallower survey component is envisioned for Challenges
% \ref{lowsnr} and \ref{gaia}.  With 10-minute integrations, a 52
% deg$^2$ Keck-FOBOS sample of environmental diagnostics for 1 million
% galaxies could be carried out in less than 20 nights.  This program
% would sample at $z \sim 1.5$ the same cosmic volume as SDSS.  A
% program of a similar scale would provide training set data for
% inference of stellar parameters in the Milky Way.  These shallow
% programs would be integrated with the deeper components described
% above into a single survey plan.

% Section 2
%%%%
% -- Instrument Description
% --     FOBOS Keck White Paper 2019
%%%%


\section{Instrument Overview}
\label{sec:concept}
% \noindent \comment{1 page}

% Here's an alternative way to put in figures if we want captions on the side (to save space)
% Could introduce a new ``counter'' to count and label figures appropriately
%\centerline{\hbox{\includegraphics[width=0.6\textwidth, angle=0]{figs/FOBOSatKeck_v1.pdf}
%    \hspace{0.1cm} \vspace{2in}
%    \parbox[b]{0.3\textwidth}{\small {\bf Figure ??:} Rendering of FOBOS instrument systems deployed at the Keck II Nasmyth port.  By mounting the FOBOS spectrographs under the Nasmyth platform, other instruments like DEIMOS can maintain access to the telescope. \vspace{2cm}}}}

%%%%%%%%%%%%%%%%%%%%%%%%%%%%%%%%%%%%%%%%%%%%%%%%%%%%%%%%%%%%%%%%%%%%%%%%
\begin{figure}[h!]
\vskip -0.1in
%\includegraphics[width=\textwidth]{figs/FOBOS_FocalPlane.pdf}
\includegraphics[width=\textwidth]{figs/FOBOS_inst_v2.pdf}
\caption{\small \comment{Changed figure to get it to compile.} {\it
Left}: Rendering of FOBOS focal plane system deployed at the Keck II
Nasmyth port. By mounting the FOBOS spectrographs under the Nasmyth
platform, other instruments like DEIMOS can maintain access to the
telescope. {\it Right}: Rendering of the ADC and focal surface with
Starbugs mounted (red cylinders). {\it Bottom-left}: Starbugs
deployed on the TAIPAN instrument.}
\label{fig:focalplane}
\end{figure}
%%%%%%%%%%%%%%%%%%%%%%%%%%%%%%%%%%%%%%%%%%%%%%%%%%%%%%%%%%%%%%%%%%%%%%%%

Mounted at the Nasmyth focus of Keck II Telescope at WMKO, FOBOS
(Figure \ref{fig:layout}) will be one of the most powerful
spectroscopic facilities deployed in the next decade. FOBOS includes
a compensating lateral atmospheric dispersion corrector (CLADC, not
pictured) to ensure that target light from all wavelengths falls on
allocated fibers while also correcting image aberrations at the edges
of the 20-arcmin diameter Keck field. Each of the CLADC lenses is 946
mm in diameter, the first two are closely spaced with lateral
relative motions enabled by three barrel-mounted actuators. The final
CLADC lens surface translates to track focal plane tilt, and it
serves as the vertical mounting plate for roaming Starbugs fiber
positioners \comment{ref}. Starbugs patrol a large on-sky area
($\sim$1 arcmin), enabling flexible and dynamic targeting
configurations with adjacent fibers as close as 10 arcsec.

% How much do we go into risks?

A total of 1800 fibers with 150-$\mu$m core diameter are deployed at
the curved focal plane. Microlens fore-optics convert the f/15 Keck
input beam to a faster f/3 focal ratio, which both demagnifies the
entrance aperture and allows for better coupling to the fiber
numerical aperture by minimizing losses from focal ratio degradation.
The focal-plane plate rotates and translates to follow image
positions as the telescope tracks across the sky. The fiber run is
kept at less than 10m to maintain high throughput at UV wavelengths
(a 10m Polymicro Silica fiber transmits $\sim$70\% and $\sim$85\% of
light at 310nm and 350nm, respectively). Special care is given to
stress-relief cabling to minimize instabilities (e.g., variable focal
ratio degradation) over the fiber run.

Sets of 600 fibers feed each of three identical spectrographs (Fig
\ref{fig:layout}). Each spectrograph uses a series of dichroics to
divide the 259mm-diameter collimated beam into four wavelength
channels, providing an instantaneous broad-band coverage from 0.31--1
$\mu$m. Fused-silica etched (FSE) gratings provide mid-channel
spectral resolutions of $R \sim 3500$ at high diffraction efficiency
in each channel. The dispersed light is focused by an f/1.1
catadioptric camera\footnote{Based on the camera design for the
Multi-Object Optical and Near-infrared Spectrograph (MOONS) on the
Very Large Telescope (VLT).} and recorded by an on-axis 4k$\times$4k
CCD mounted at the center of the first camera lens element.
Spectrographs are mounted in a permanent temperature-controlled
housing on the Nasmyth deck, whereas the focal-plane system can be
unmounted and stowed alongside existing Keck instruments. The
end-to-end instrument throughput peaks at 60\% and is greater than
30\% at {\it all} wavelengths.

FOBOS includes observatory level systems for precise instrument
calibration using dome-interior screen illumination, a metrology
system for accurate fiber positioning, and guide cameras for field
acquisition and guiding. \comment{make the following consistent with
what we say in the summary}. Initial deployment of the focal-plane
will focus on a single-fiber format, with a secondary deployment of
multi-format fiber bundles to follow. Additional instrument upgrades
include integration of fibers that feed additional spectographs ---
these spectrographs could provide increased multiplex capacity,
higher spectral resolution, and/or observe different spectral regions
--- and additional front-end sensing equipment that fully support and
benefit from image corrections with Ground-Layer Adaptive Optics.


% Section 3
%%%%
% -- Instrument Description
% --     FOBOS Keck White Paper 2019
%%%%


\section{Recent and Ongoing Progress}
\label{sec:progress}

FOBOS development over the past year has focused on (1) engaging the
scientific and instrumentalist communities in discussions about the
FOBOS design, particularly with regard to science cases and
instrument feasibility, (2) initial development of an instrument
simulator and exposure-time calculator, (3) advancing the
optomechanical designs of the focal-plane and ADC systems, (4)
consolidating development of Fiber-WFOS for TMT and repurposing much
of the design for FOBOS at Keck, (5) working with Keck instrument
scientists and engineers toward a practical integration plan, and (6)
developing a detailed project execution plan, schedule, and short-term budget.
Many of these activities were enabled by funds provided in response
to our previous white-paper submission.

Visits to Keck-user institutes have been particularly helpful in
building interest around the instrument and refining its
specifications. As of this writing, Bundy, MacDonald, and/or Westfall
have visited UCR (also joined by Michael Cooper from UCI), UCLA, Keck
observatory, LBNL, and UCSB. We plan to continue with visits to UCB, UCD, and CIT before the end of the year.

Feedback from the Keck community has emphasized the desire for very high multiplex, flexibility of the focal-plane
sampling, and sensitivity toward the UV. Although the capabilities of PFS are a common comparison for FOBOS, many of
the science goals now enumerated in Section 1 require target densities and/or wavelength coverage that PFS will not
provide. Another strong message from Keck users was the desire for deployable IFUs. Starbugs are attractive for this
reason becuase they make swapping focal-plane formats straightforward.

% Our
% initial concepts for these different focal-plane formats provided a
% secondary motivation for FOBOS's large pseudo-slit capacity: it can
% provide 15-100 smaller, freely deployable IFUs. Finally, Keck
% remains unique in its throughput toward the atmospheric limit. No
% other current or planned \comment{check this again} spectrograph for
% 10m-class telescopes provides sensitivity below $\sim$380nm,
% providing FOBOS with a unique capability among the landscape of
% forthcoming instrumentation.

% a large-format monolithic IFU and 

% yield a monolithic IFU with a FOV that is competitive with VLT/MUSE
% and can

Another area of progress involves use of UCO funding to address risks in the
coupling of the microlens fore-optics to the fiber and the coupling
of the microlens entrance aperture to the Keck II focal plane.  This ``FIDDLES'' prototyping effort will test coupling performance in labs at UCO and LBL as well as on-sky at Lick and Keck Observatories.  Initial tests have begun and optomechanical designs were recently completed to enable hardware purchases.

% Finally, with the permission of the SSC, we proposed for
% instrument-design funds from the NSF via its new Mid-scale Research
% Infrastructure scheme. Although the proposal was ultimately
% unsuccessful, preparation of the proposal led to a ground swell of
% development in our project planning and science development, much of
% which is included in this submission.

\subsection{New technical expertise from DESI and PFS} Building on the FOBOS instrument team's experience with SDSS projects like BOSS, APOGEE, and MaNGA, we have recruited new expertise from DESI and PFS.  Claire Poppett at SSL/LBNL is DESI's Lead Fiber Scientist and has joined FOBOS to spearhead the fiber system and fore-optics development.  Tim Miller, also at SSL/LBNL, developed the DESI corrector's optical design and will take on the FOBOS ADC.  Renbin Yan, a scientist and instrument builder, has spent the last year at Princeton working with Jim Gunn on PFS-like calibration systems for future instruments.  Renbin was a heavy Keck observer for DEEP2 and will apply PFS knowledge on the calibration problem to Keck and FOBOS.



% Section 4
%%%%
% -- Proposed Work and Budget
% --     FOBOS Keck White Paper 2019
%%%%

\section{Proposed Work}
\label{sec:design}

The work proposed here complements ongoing FOBOS development efforts
in preparation for future proposals, particularly an NSF MSIP call
expected for 2020. We will complete a full FOBOS conceptual design
and focus particularly on design efforts for sub-systems we believe
are the highest risk: Starbugs fiber positioners, fiber fore-optics,
and the calibration system. This work will enable instrument
performance evaluations against proposed science use cases, as well
as initial total cost estimates. To support this effort, we request:

\textbf{
\begin{enumerate}
\item Approval to pursue available federal and private funding opportunities in support of FOBOS design, construction, integration, and commissioning.
% I think we want input from Mark and John on this fist item.  I could
% see them balking at permission through construction and
% commissioning-NKM.  
\item WMKO Phase-A funding of \request{} over two years
(FY 2020 and 2021) to advance the instrument design in preparation for NSF MSIP and MsRI, and other funding opportunities.
\item Engineering support from WMKO staff on key telescope-instrument interfaces.  
% The
% necessary tasks list for WMKO effort has been iterated on between the
% FOBOS team and the Observatory in preparation for the MsRI-1
% pre-proposal submitted earlier this year. These tasks are detailed in
% the included schedule.
\end{enumerate}}

\subsection{Instrument Design Work and Project Planning} We list below
the primary instrument-design and project planning activities that we
have identified as top priorities for FOBOS development in the lead up
to larger funding requests (see also the attached schedule and budget schematic).  

% Engineering effort for this development phase will come from both
% UCO and Space Sciences Lab (SSL) at UC Berkeley (in addition to contributions from WMKO).  Staff at SSL working
% on the DESI project are becoming available and have interest in FOBOS
% and expertise from DESI which significantly strengthen the FOBOS team.

\noindent \textbf{Project Planning} Continued development of the
schedule, budget, project execution plan, and systems engineering
structure needed for compliance with the NSF Major Facilities Guide.
This work will focus on preliminary and final design phases in
preparation for an MSIP design proposal followed by early planning for
construction needed for a future MsRI-2 proposal. 

\noindent \textbf{Atmospheric Dispersion Compensator (ADC):} Continued
optical design work with a goal of pushing the current 17~arcmin field-of-view
to the full 20~arcmin field of view available at Keck.  The
optomechanical design of the lens cells.  Further development of the
motion systems for ADC articulation and field rotation.

\noindent \textbf{Focal Plane System:} Continued mechanical design;
fiber and Starbug actuator support, interface to front end module, and
interface to storage position. This system also defines one of the
interfaces to the Keck II Telescope and must comply with WMKO space
envelopes, servicing needs, and other requirements. The focal plane
system also inputs the guide cameras. 

\noindent \textbf{Starbugs fiber positioners:} Starbugs are a
positioning technology developed and deployed by the Australian Astronomical Optics (AAO), which has partnered with our
team to generate a conceptual design for use of Starbugs by FOBOS.  Starbugs are currently being tested on-sky with the
TAIPAN instrument at UK Schmidt Telescope and published results on their performance are expected in summer 2019.  An
AAO contract included in this request (\$60k) will ensure that performance details as they relate to FOBOS, a
quantitative risk assessment, and project planning required for the MSIP proposal will be carried out.  During a recent
visit to UCO, Jon Lawrence, AAO's Head of Technology, agreed to support the rapid MSIP proposal schedule.

\noindent \textbf{Fiber System:} Outside of this funding request, a UCO
mini-grant {\it FIDDLES} is underway to specifically retire
risks associated with the micro-optics at the focal plane.  This work is
on-going and will be a major step forward in developing this sub-system.
Work on the fiber system under this proposal will focus on system-level
design of the fiber system including cabling, support, and stress relief. 

% Further
% development of the slit input side of the fiber system and development
% of the micro-lens array for a central, fixed-position 4.5~arcsec
% diameter IFU for fast target of opportunity acquisition.

\noindent \textbf{Spectrographs.} The conceptual design of the
spectrograph comes from work developed for Fiber-WFOS.  Continuing work
on the spectrograph will focus on documentation and modifications needed
in preparation for the MSIP proposal.  We will evaluate the need for a trade study on the use of catadioptric versus refractive cameras and optimize the optical design against fiber size and numerical aperture.

\noindent \textbf{Calibration System.} Given the need for a dome screen
and lamp projection system the calibration system is a joint development
between WMKO and the FOBOS team.  The work in this proposal will focus
on defining the calibration requirements followed by conceptual design
development of the screen and projection system in partnership with the
WMKO engineering staff.  Note that our team now includes Renbin Yan, who has specific expertise on the Subaru PFS calibration system and has designed similar calibration systems for other telescopes.

% \noindent \textbf{Auxiliary Systems.} Design of auxiliary systems
% inputs Nasmyth platform interfaces, utilities access, fiber routing
% and support, thermal control and vibration control systems.

\noindent \textbf{Data Systems.} We will develop the requirements and initial concept for the FOBOS data simulator following instrument forward-modeling techniques developed by DESI.  The simulator will enable tests of potential FOBOS science cases (e.g., exposure time estimates, redshifting success).  For the MSIP proposal, we will construct a detailed plan for evolving the data simulator into a data reduction pipeline and using that as the basis for a data analysis pipeline (DAP).  The FOBOS DAP will take advantage of the fixed spectral format and common target
classes to provide high-level data products, including Doppler shifts, emission-line strengths, and template continuum
fits (cf., Westfall et al.; SDSS-IV MaNGA DAP).  Planning will include development of user-friendly platforms built on
the Keck Observatory Archive for serving raw data, reduced spectra, and DAP science products.

\noindent \textbf{Operations.} Powered by Starbugs fiber positioners, FOBOS
will enable fast, dynamic reallocation of fibers.  We will develop an initial target allocation simulator to determine efficiencies for various science programs and explore options for program combination and optimization under different observing scenarios.  This work requires planning interfaces with the fiber positioning control software and the Keck user.

% \begin{deluxetable}{lcccccccc}
% \tablecaption{MGE Parameters}
% \tabletypesize{\footnotesize}
% \tablewidth{0pt}
% \tablecolumns{9}
% \tablehead{
% \multicolumn{1}{c}{} & \multicolumn{2}{c}{$\Sigma_{\rm Exp}$} & \multicolumn{2}{c}{$\Sigma_{\rm deV}$} & \multicolumn{2}{c}{$\Sigma_{\rm Exp}^{\rm SDSS}$} & \multicolumn{2}{c}{$\Sigma_{\rm deV}^{\rm SDSS}$} \\
% \cline{2-3} \cline{4-5}  \cline{6-7}  \cline{8-9} \\
% \colhead{N$_{\rm terms}$} & \colhead{$\sigma_j$} & \colhead{$A_{j}$} & \colhead{$\sigma_j$} & \colhead{$A_{j}$} & \colhead{$\sigma_j$} & \colhead{$A_{j}$} & \colhead{$\sigma_j$} & \colhead{$A_{j}$} \\
% %\cline{1-12} \\
% %\multicolumn{12}{c}{EGS Field}
% }

% \startdata
% 1  &  5.1  &  0.45  &  0.1  &  328.94  &  3.5  &  0.31  &  1.5  &  30.33  \\
% \enddata
% \label{table:mge}
% \tablecomments{The components are normalized so that the sum yields $\Sigma(R_e) \equiv \Sigma_e = 1.0$ in arbitrary units of surface flux density.  The units
%   of $\sigma_j$ are $0.01 R_e$.}
% \end{deluxetable}


\begin{table}[h!]
\centering
\scriptsize
\caption{FOBOS Instrument Specifications and Status}
\label{tab:specs}
\vspace*{-10pt}
\begin{tabular}{l r l}
Component                     & Current Value & Notes \\
\hline
ADC \\
\hline
\hspace{0.2cm} ADC Type & CLADC & Enhance image quality, telecentricity; Starbugs surface \\
\hspace{0.2cm} FoV & 17$^\prime$ & 20$^\prime$ goal; further optical design optimization in Phase-A \\
\hspace{0.2cm} Mech.~Design & conceptual & Diameter: 0.8 m; cells, bearings, and one motion axis defined \\
\hline
Fore-optics \\
\hline
\hspace{0.2cm} Type & TBD & Both focal-plane and pupil imaging designs in-hand \\
\hspace{0.2cm} Sampling & 0.9$^{\prime\prime}$ & Single-fiber; Phase-A optimize against FWHM distribution \\
\hspace{0.2cm} f/\# & f/3.2 & Demagnification factor of 4.7 \\
\hspace{0.2cm} Mech.~Design & prelim. &  $\approx$1$\times$5 mm; prototype ready for fabrication (\emph{FIDDLES}) \\
\hspace{0.2cm} Coupling loss & 10\% & Assumed; Phase-A testing in lab and on-sky (\emph{FIDDLES}) \\
\hspace{0.2cm} Micro-lens array & TBD & Phase-A conceptual design for IFU modes \\
\hspace{0.2cm} Lab test bench & Yes & DESI lab; UCO fiber test bench built (calibration required) \\
\hline
Positioners \\
\hline
\hspace{0.2cm} Type & Starbugs & Starbugs are baselined; but trade study in Phase-A \\
\hspace{0.2cm} Reqs & TAIPAN &  Config time: 2 min; Accuracy: 0.02$^{\prime\prime}$; Collision radius: 10$^{\prime\prime}$ \\
\hspace{0.2cm} Metrology & TBD & Conceptual design of metrology system in Phase-A \\
\hline
Fiber System \\
\hline
\hspace{0.2cm} Wrap & IGUS & IGUS energy chain or similar \citep[e.g.,][]{poppett16}  \\
\hspace{0.2cm} Length & $<10$ m & Preserve UV throughput  \\
\hspace{0.2cm} Cabling & Planetary & Heritage from \citet{soukup10}  \\
\hspace{0.2cm} Fiber & 155 $\mu m$ core & Polymicro FBP; smaller fiber study in Phase-A \\
\hline
Spectrographs \\
\hline
\hspace{0.2cm} Number & 3 & Driven by desired on-sky fiber density \\
\hspace{0.2cm} Fibers & 3 $\times$ 600 & ``Slit density'' to be set by image quailty, simulated extractions \\
\hspace{0.2cm} Channels & 4 & Three channels may be possible (Phase-A study) \\
\hspace{0.2cm} Resolution & $R \sim 3500$ & Measured at each channel's central wavelength \\
\hspace{0.2cm} Gratings & FSE & Grating efficiency $>$80\% at all $\lambda$, used in Littrow; quotes from Horiba \& Fraunhofer \\
\hspace{0.2cm} Pseudoslit length & 150 mm & Slit heritage from DESI and MaNGA \\
\hspace{0.2cm} Collimator & f/3 & Accepts f/3.2 fiber beam with FRD margin; beam size: 259 mm \\
\hspace{0.2cm} $\Delta\lambda$ & 0.31-1.0 $\mu m$ & [310--425], [415--565], [555--755], [745--1000] nm \\
\hspace{0.2cm} Cameras & Catadioptric & Schmidt; Study refractive solutions in Phase-A \\
\hspace{0.3cm} Obscuration loss & 15-20\% &  \\
\hline
Detectors \\
\hline
\hspace{0.2cm} Format & 4k$\times$4k & 15 $\mu$m pixel size \\
\hspace{0.2cm} Sampling & 3.75 px & Spatial; Could decrease; Higher slit density possible \\
\hspace{0.2cm} QE & $>$94\% & Over 410--900 nm; $>$80\% at 310--410 nm; tapered AR coating critical in UV \\
\hline




\end{tabular}
\end{table}


% To efficiently
% determine the best options given a wide range of possible targets and
% desired observing outcomes, we will develop a conceptual design for
% MAISTRO,\footnote{MAISTRO: Modular Artificial Intelligence System for
% Target Reallocation and Observing.} an ``artificial intelligence''
% (AI) targeting system that will learn optimization strategies for
% assigning targets from a database of overlapping observing programs
% with pre-defined priorities. The AI package will aggregate data
% quality using a quick-look reduction package, science-driven
% performance metrics, {\it and real-time assessments of the observing
% conditions} to make dynamic targeting recommendations. For example,
% if conditions are slightly less than optimal, MAISTRO would
% reconfigure Starbugs to brighter objects in a field or implement a
% different program prioritization. MAISTRO will incorporate updated
% target lists and priorities from the active observer and could easily
% be over-ridden at any time. Fractions of the full FOBOS multiplex
% might also be reserved ``manual targeting'' as required by the
% program PI.

%   - maintains a database with observational progress on individual
%     targets in the survey and
%   - dynamically reallocates fibers based on real-time assessments of
%     the aggregate S/N of each target to meet the specific need of each
%     science case.

% This requires significant design and testing of a combined software
% package and hardware interface.  Specific considerations involve (1)
% fast and robust reduction procedures (cf. MaNGA DOS) that can assess
% the aggregate data and (2) a responsive database with a schema
% optimized for real-time decision making to select targets for
% (re)acquisition while accounting for collision limitations.  Provided
% enough design effort, this lends itself to a machine-learning
% application.




%-----------------------------------------------------------------------

% Everything after this limited to two pages.
\newpage


% Section 5
%%%%
% -- Schedule and Budget breakdown and funding path
% --     FOBOS Keck White Paper 2019
%%%%

\section{Schedule, Budget, and Funding Path}
\label{sec:budget}

\subsection{Funding Path}

Early funding for FOBOS has been obtained through a number of
sources. First, conceptual development of Fiber-WFOS for TMT has
provided the backbone for the current FOBOS design. Second, a number
of smaller grants were used to develop and assess specific aspects of
FOBOS. Namely, a 2017 UCO mini-grant (\$XXk) to KG Lee was used for
conceptual development of the microlens optics; a 2017 UCO mini-grant
(\$55k) to K. Bundy allowed for a study of sky-subtraction fidelity
in existing fiber systems; WMKO white-paper funds (\$40k) provided to
K. Bundy in 2018 are still being used to develop science cases, build
invested science teams, and perform focal-plane and spectrograph
design studies; and a 2018 UCO mini-grant (\$120k) to K. Bundy was
provided for design and fabrication of a microlens-coupled fiber
system primarily to demonstrate its throughput at Keck compared to
lab measurements and to simultaneous observations with DEIMOS. The
latter builds on UCO's ongoing investment in a fiber-testing facility
needed for a number of internal projects has helped to further the
FOBOS design. This Phase A funding request is designed to bring the
project to a stage of readiness needed to submit proposals to the NSF
MSIP (2020) and MsRI-2 (2023) programs.

We intend to propose for FOBOS design funds via an MSIP solicitation
expected in early 2020; if successful, this will fund the full
instrument-design phase. The construction funding would then come
from a MsRI-2 proposal in 2023. We have requested ``Phase A'' funding
from late 2019 to mid 2021 to allow for continued development between
submitting our MSIP proposal and when the funding is made available.
In the event that we are unsuccessful in the MSIP proposal, our
requested ``Phase A'' funding in the last half of this proposal would
be used for preparation towards a MsRI-1 proposal in 2021. The
included schedule shows our NSF funding plan as funding windows at
the top of the Gantt chart.

We intend to apply for other smaller funding opportunities as they
become available, including the UCO mini-grant and NSF ATI grants.
Both of these would be targeted at relatively self-contained design
components of FOBOS's overall system. Other government funding and
private funding is also being pursued as opportunities become
available.



\setcounter{page}{1}
\bibliographystyle{apj}
\bibliography{references}

\end{document}


