%%%%
% -- Proposed Work and Budget
% --     FOBOS Keck White Paper 2019
%%%%

\section{Proposed Work}
\label{sec:design}

The work proposed here complements ongoing FOBOS development efforts
in preparation for future proposals, particularly an NSF MSIP call
expected for 2020. We will complete a full FOBOS conceptual design
and focus particularly on design efforts for sub-systems we believe
are the highest risk: Starbugs fiber positioners, fiber fore-optics,
and the calibration system. This work will enable instrument
performance evaluations against proposed science use cases, as well
as initial total cost estimates. To support this effort, we request:

\textbf{
\begin{enumerate}
\item Approval to pursue available federal and private funding opportunities in support of FOBOS design, construction, integration, and commissioning.
% I think we want input from Mark and John on this fist item.  I could
% see them balking at permission through construction and
% commissioning-NKM.  
\item WMKO Phase-A funding of \request{} over two years
(FY 2020 and 2021) to advance the instrument design in preparation for NSF MSIP and MsRI, and other funding opportunities.
\item Engineering support from WMKO staff on key telescope-instrument interfaces.  
% The
% necessary tasks list for WMKO effort has been iterated on between the
% FOBOS team and the Observatory in preparation for the MsRI-1
% pre-proposal submitted earlier this year. These tasks are detailed in
% the included schedule.
\end{enumerate}}

\subsection{Instrument Design Work and Project Planning} We list below
the primary instrument-design and project planning activities that we
have identified as top priorities for FOBOS development in the lead up
to larger funding requests (see also the attached schedule and budget schematic).  

% Engineering effort for this development phase will come from both
% UCO and Space Sciences Lab (SSL) at UC Berkeley (in addition to contributions from WMKO).  Staff at SSL working
% on the DESI project are becoming available and have interest in FOBOS
% and expertise from DESI which significantly strengthen the FOBOS team.

\noindent \textbf{Project Planning} Continued development of the
schedule, budget, project execution plan, and systems engineering
structure needed for compliance with the NSF Major Facilities Guide.
This work will focus on preliminary and final design phases in
preparation for an MSIP design proposal followed by early planning for
construction needed for a future MsRI-2 proposal. 

\noindent \textbf{Atmospheric Dispersion Compensator (ADC):} Continued
optical design work with a goal of pushing the current 17~arcmin field-of-view
to the full 20~arcmin field of view available at Keck.  The
optomechanical design of the lens cells.  Further development of the
motion systems for ADC articulation and field rotation.

\noindent \textbf{Focal Plane System:} Continued mechanical design;
fiber and Starbug actuator support, interface to front end module, and
interface to storage position. This system also defines one of the
interfaces to the Keck II Telescope and must comply with WMKO space
envelopes, servicing needs, and other requirements. The focal plane
system also inputs the guide cameras. 

\noindent \textbf{Starbugs fiber positioners:} Starbugs are a
positioning technology developed and deployed by the Australian Astronomical Optics (AAO), which has partnered with our
team to generate a conceptual design for use of Starbugs by FOBOS.  Starbugs are currently being tested on-sky with the
TAIPAN instrument at UK Schmidt Telescope and published results on their performance are expected in summer 2019.  An
AAO contract included in this request (\$60k) will ensure that performance details as they relate to FOBOS, a
quantitative risk assessment, and project planning required for the MSIP proposal will be carried out.  During a recent
visit to UCO, Jon Lawrence, AAO's Head of Technology, agreed to support the rapid MSIP proposal schedule.

\noindent \textbf{Fiber System:} Outside of this funding request, a UCO
mini-grant {\it FIDDLES} is underway to specifically retire
risks associated with the micro-optics at the focal plane.  This work is
on-going and will be a major step forward in developing this sub-system.
Work on the fiber system under this proposal will focus on system-level
design of the fiber system including cabling, support, and stress relief. 

% Further
% development of the slit input side of the fiber system and development
% of the micro-lens array for a central, fixed-position 4.5~arcsec
% diameter IFU for fast target of opportunity acquisition.

\noindent \textbf{Spectrographs.} The conceptual design of the
spectrograph comes from work developed for Fiber-WFOS.  Continuing work
on the spectrograph will focus on documentation and modifications needed
in preparation for the MSIP proposal.  We will evaluate the need for a trade study on the use of catadioptric versus refractive cameras and optimize the optical design against fiber size and numerical aperture.

\noindent \textbf{Calibration System.} Given the need for a dome screen
and lamp projection system the calibration system is a joint development
between WMKO and the FOBOS team.  The work in this proposal will focus
on defining the calibration requirements followed by conceptual design
development of the screen and projection system in partnership with the
WMKO engineering staff.  Note that our team now includes Renbin Yan, who has specific expertise on the Subaru PFS calibration system and has designed similar calibration systems for other telescopes.

% \noindent \textbf{Auxiliary Systems.} Design of auxiliary systems
% inputs Nasmyth platform interfaces, utilities access, fiber routing
% and support, thermal control and vibration control systems.

\noindent \textbf{Data Systems.} We will develop the requirements and initial concept for the FOBOS data simulator following instrument forward-modeling techniques developed by DESI.  The simulator will enable tests of potential FOBOS science cases (e.g., exposure time estimates, redshifting success).  For the MSIP proposal, we will construct a detailed plan for evolving the data simulator into a data reduction pipeline and using that as the basis for a data analysis pipeline (DAP).  The FOBOS DAP will take advantage of the fixed spectral format and common target
classes to provide high-level data products, including Doppler shifts, emission-line strengths, and template continuum
fits (cf., Westfall et al.; SDSS-IV MaNGA DAP).  Planning will include development of user-friendly platforms built on
the Keck Observatory Archive for serving raw data, reduced spectra, and DAP science products.

\noindent \textbf{Operations.} Powered by Starbugs fiber positioners, FOBOS
will enable fast, dynamic reallocation of fibers.  We will develop an initial target allocation simulator to determine efficiencies for various science programs and explore options for program combination and optimization under different observing scenarios.  This work requires planning interfaces with the fiber positioning control software and the Keck user.

% \begin{deluxetable}{lcccccccc}
% \tablecaption{MGE Parameters}
% \tabletypesize{\footnotesize}
% \tablewidth{0pt}
% \tablecolumns{9}
% \tablehead{
% \multicolumn{1}{c}{} & \multicolumn{2}{c}{$\Sigma_{\rm Exp}$} & \multicolumn{2}{c}{$\Sigma_{\rm deV}$} & \multicolumn{2}{c}{$\Sigma_{\rm Exp}^{\rm SDSS}$} & \multicolumn{2}{c}{$\Sigma_{\rm deV}^{\rm SDSS}$} \\
% \cline{2-3} \cline{4-5}  \cline{6-7}  \cline{8-9} \\
% \colhead{N$_{\rm terms}$} & \colhead{$\sigma_j$} & \colhead{$A_{j}$} & \colhead{$\sigma_j$} & \colhead{$A_{j}$} & \colhead{$\sigma_j$} & \colhead{$A_{j}$} & \colhead{$\sigma_j$} & \colhead{$A_{j}$} \\
% %\cline{1-12} \\
% %\multicolumn{12}{c}{EGS Field}
% }

% \startdata
% 1  &  5.1  &  0.45  &  0.1  &  328.94  &  3.5  &  0.31  &  1.5  &  30.33  \\
% \enddata
% \label{table:mge}
% \tablecomments{The components are normalized so that the sum yields $\Sigma(R_e) \equiv \Sigma_e = 1.0$ in arbitrary units of surface flux density.  The units
%   of $\sigma_j$ are $0.01 R_e$.}
% \end{deluxetable}


\begin{table}[h!]
\centering
\scriptsize
\caption{FOBOS Instrument Specifications and Status}
\label{tab:specs}
\vspace*{-10pt}
\begin{tabular}{l r l}
Component                     & Current Value & Notes \\
\hline
ADC \\
\hline
\hspace{0.2cm} ADC Type & CLADC & Enhance image quality, telecentricity; Starbugs surface \\
\hspace{0.2cm} FoV & 17$^\prime$ & 20$^\prime$ goal; further optical design optimization in Phase-A \\
\hspace{0.2cm} Mech.~Design & conceptual & Diameter: 0.8 m; cells, bearings, and one motion axis defined \\
\hline
Fore-optics \\
\hline
\hspace{0.2cm} Type & TBD & Both focal-plane and pupil imaging designs in-hand \\
\hspace{0.2cm} Sampling & 0.9$^{\prime\prime}$ & Single-fiber; Phase-A optimize against FWHM distribution \\
\hspace{0.2cm} f/\# & f/3.2 & Demagnification factor of 4.7 \\
\hspace{0.2cm} Mech.~Design & prelim. &  $\approx$1$\times$5 mm; prototype ready for fabrication (\emph{FIDDLES}) \\
\hspace{0.2cm} Coupling loss & 10\% & Assumed; Phase-A testing in lab and on-sky (\emph{FIDDLES}) \\
\hspace{0.2cm} Micro-lens array & TBD & Phase-A conceptual design for IFU modes \\
\hspace{0.2cm} Lab test bench & Yes & DESI lab; UCO fiber test bench built (calibration required) \\
\hline
Positioners \\
\hline
\hspace{0.2cm} Type & Starbugs & Starbugs are baselined; but trade study in Phase-A \\
\hspace{0.2cm} Reqs & TAIPAN &  Config time: 2 min; Accuracy: 0.02$^{\prime\prime}$; Collision radius: 10$^{\prime\prime}$ \\
\hspace{0.2cm} Metrology & TBD & Conceptual design of metrology system in Phase-A \\
\hline
Fiber System \\
\hline
\hspace{0.2cm} Wrap & IGUS & IGUS energy chain or similar \citep[e.g.,][]{poppett16}  \\
\hspace{0.2cm} Length & $<10$ m & Preserve UV throughput  \\
\hspace{0.2cm} Cabling & Planetary & Heritage from \citet{soukup10}  \\
\hspace{0.2cm} Fiber & 155 $\mu m$ core & Polymicro FBP; smaller fiber study in Phase-A \\
\hline
Spectrographs \\
\hline
\hspace{0.2cm} Number & 3 & Driven by desired on-sky fiber density \\
\hspace{0.2cm} Fibers & 3 $\times$ 600 & ``Slit density'' to be set by image quailty, simulated extractions \\
\hspace{0.2cm} Channels & 4 & Three channels may be possible (Phase-A study) \\
\hspace{0.2cm} Resolution & $R \sim 3500$ & Measured at each channel's central wavelength \\
\hspace{0.2cm} Gratings & FSE & Grating efficiency $>$80\% at all $\lambda$, used in Littrow; quotes from Horiba \& Fraunhofer \\
\hspace{0.2cm} Pseudoslit length & 150 mm & Slit heritage from DESI and MaNGA \\
\hspace{0.2cm} Collimator & f/3 & Accepts f/3.2 fiber beam with FRD margin; beam size: 259 mm \\
\hspace{0.2cm} $\Delta\lambda$ & 0.31-1.0 $\mu m$ & [310--425], [415--565], [555--755], [745--1000] nm \\
\hspace{0.2cm} Cameras & Catadioptric & Schmidt; Study refractive solutions in Phase-A \\
\hspace{0.3cm} Obscuration loss & 15-20\% &  \\
\hline
Detectors \\
\hline
\hspace{0.2cm} Format & 4k$\times$4k & 15 $\mu$m pixel size \\
\hspace{0.2cm} Sampling & 3.75 px & Spatial; Could decrease; Higher slit density possible \\
\hspace{0.2cm} QE & $>$94\% & Over 410--900 nm; $>$80\% at 310--410 nm; tapered AR coating critical in UV \\
\hline




\end{tabular}
\end{table}


% To efficiently
% determine the best options given a wide range of possible targets and
% desired observing outcomes, we will develop a conceptual design for
% MAISTRO,\footnote{MAISTRO: Modular Artificial Intelligence System for
% Target Reallocation and Observing.} an ``artificial intelligence''
% (AI) targeting system that will learn optimization strategies for
% assigning targets from a database of overlapping observing programs
% with pre-defined priorities. The AI package will aggregate data
% quality using a quick-look reduction package, science-driven
% performance metrics, {\it and real-time assessments of the observing
% conditions} to make dynamic targeting recommendations. For example,
% if conditions are slightly less than optimal, MAISTRO would
% reconfigure Starbugs to brighter objects in a field or implement a
% different program prioritization. MAISTRO will incorporate updated
% target lists and priorities from the active observer and could easily
% be over-ridden at any time. Fractions of the full FOBOS multiplex
% might also be reserved ``manual targeting'' as required by the
% program PI.

%   - maintains a database with observational progress on individual
%     targets in the survey and
%   - dynamically reallocates fibers based on real-time assessments of
%     the aggregate S/N of each target to meet the specific need of each
%     science case.

% This requires significant design and testing of a combined software
% package and hardware interface.  Specific considerations involve (1)
% fast and robust reduction procedures (cf. MaNGA DOS) that can assess
% the aggregate data and (2) a responsive database with a schema
% optimized for real-time decision making to select targets for
% (re)acquisition while accounting for collision limitations.  Provided
% enough design effort, this lends itself to a machine-learning
% application.

