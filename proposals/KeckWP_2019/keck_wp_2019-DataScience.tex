%%%%
% -- Data Science
% --     FOBOS Keck White Paper 2019
%%%%

\subsubsection{FOBOS as an ideal spectroscopic training instrument}
\label{sec:datascience}

\subsection{Addressing Data Science Challenges and Designing FOBOS Training Sets}
\label{sec:survey}

Our team includes leading experts on data science applications to
astronomy and, specifically, LSST.  We will also use our established
connections to LSST's Informatics and Statistics Science Collaboration
(ISSC) to advertise, recruit, and coordinate efforts to tackle the Data
Science Challenges described in Section \ref{sec:goals}.  Our proposal
request includes two community workshops to motivate progress and discuss
results. At the end of the proposal period, we will publish the results
and developed software packages.

Our data-science challenges require work on simulated
imaging$+$spectroscopic data sets where input physical properties (e.g.,
redshift) can be compared to output recovered values.  Simulated imaging
data (e.g., from LSST and WFIRST) are in-hand, while mock spectroscopy
will be provided by a FOBOS instrument simulator, an initial version of
which has already been developed.  Further advances to be supported by
this proposal include improved error modeling and simulating systematic
effects from detector artifacts, image quality aberrations informed by
the emerging detailed optical design, and variable observing conditions.

The resulting success in addressing each data-science challenge will
define a level of readiness and set requirements on desired FOBOS
training sets, including number of sources, pointings, magnitude limits,
signal-to-noise thresholds, and observing conditions.  Preliminary
observing design and a description of required operational modes to
efficiently observe these training sets will begin with this proposal.
Operational modes will set requirements on target aggregation and
prioritization systems, field acquisition speed, field rotation range,
zenith avoidance zone, reconfiguration time, calibrations, read-out
time, quick-look reduction software and processing rates.  We will
develop integrated program concepts that efficiently combine required
observations.  Detailed survey and execution plans will be completed in
the next phase of this project (MSRI-2).  Roughly 20\% of Keck observing
time is open to the public, and as in previous federally-funded
projects, we fully expect that Senior Personnel at Keck institutions
will be successful in collaborative efforts to secure significant
amounts of additional telescope observing time to enable rapid, public
release of FOBOS training data \citep[e.g.,][]{newman13}.

% The complete photo-$z$ training survey described in \citet{newman15}
% would require 15 independent pointings, each spanning 0.1 deg$^2$ with
% a target density of 6 arcmin$^{-2}$ (8 arcmin$^{-2}$ when including $z
% > 1.5$ galaxies accessible in the UV with Keck-FOBOS), perfectly
% matched to the Keck-FOBOS field-of-view and target density.  With a
% conservative exposure time of 100 hours to reach 75\% redshift
% completeness for 40,000 galaxies with $i_{\rm AB} < 25.3$, the Neman
% survey would require 400 nights.  Challenge \ref{photoz} would reduce
% the required survey duration by a factor of at least four.  Meanwhile
% the extreme depths and flux-limited selection are likely also
% requirements for training sets associated with Challenges \ref{phot},
% \ref{uv}.

% A wider and shallower survey component is envisioned for Challenges
% \ref{lowsnr} and \ref{gaia}.  With 10-minute integrations, a 52
% deg$^2$ Keck-FOBOS sample of environmental diagnostics for 1 million
% galaxies could be carried out in less than 20 nights.  This program
% would sample at $z \sim 1.5$ the same cosmic volume as SDSS.  A
% program of a similar scale would provide training set data for
% inference of stellar parameters in the Milky Way.  These shallow
% programs would be integrated with the deeper components described
% above into a single survey plan.