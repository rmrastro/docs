%%%%
% -- Data Science
% --     FOBOS Keck White Paper 2019
%%%%

\subsubsection{FOBOS as an ideal spectroscopic training instrument}
\label{sec:datascience}

\begin{figure}[h!]
%
\vskip -0.1in
%
\includegraphics[width=\textwidth]{figs/LGplots.pdf}
%
\caption{{\it Left}: Validation of {\it The Cannon} measurements of
stellar effective temperature, $T_{\rm eff}$, and surface gravity, $\log
g$, using low-resolution LAMOST spectra (left) compared to
high-resolution APOGEE measurements
\citep[right;][]{2017ApJ...836....5H}. {\it Top-right}: Recovery of
elemental abundances from low-resolution LAMOST spectra compared to
high-resolution measurements from GALAH (Xiang et al., in prep).  {\it
Bottom-right}: The circular-speed curve of the Milky Way determined
using a data-driven model that combines stellar parameters determined
from APOGEE spectra with photometry from WISE, 2MASS, and Gaia, yielding
the most precise measurements to date \citep{2019ApJ...871..120E}.}
%
\label{fig:Cannon}
%
\end{figure}

Radial velocity studies of stars in the MW halo or the M31 disk require
observations of up to 10 hours on large telescopes
\citep[e.g.,][]{2018arXiv180904082C}.  This again motivates
machine-learning algorithms to extract physical quantities from both
multi-band imaging and lower quality spectra (low resolution and S/N)
using relatively small, yet high-S/N, training sets.  For example,
\citet{2015ApJ...808...16N} have developed {\it The Cannon}, a
supervised learning approach that uses spectra with known stellar
parameters to label spectra where those parameters are unknown
(Fig.~\ref{fig:Cannon}).  Additionally, \citet{2018arXiv180401530T} have
developed {\it The Payne} which can infer 16 stellar-abundance labels
from low-resolution spectra using a neural network and theoretical
stellar spectra.  Finally, \citet{2018arXiv180803278T} have combined
Kepler-based astroseismology measurements with APOGEE spectra to
determine stellar age to $\sim$25\% precision using a neural network.
Our proposed effort builds on new lines of inquiry based on these
successes.


 A nested network of stellar parameter training
samples for resolved Milky Way and Local Group studies via extracting maximum information from photometry, in this case stellar parameters.  Our goal is to reach
magnitudes significantly fainter than the detection limit of current and upcoming spectroscopic surveys of the Milky
Way including Gaia, APOGEE,\footnote{APOGEE, the Apache Point Observatory Galaxy Evolution Experiment has observed in
both SDSS-III and SDSS-IV.} the SDSS-V Milky Way Mapper, planned programs with 4MOST\footnote{4MOST: 4-meter
Multi-object Spectroscopic Telescope.} and the Dark Energy Spectroscopic Instrument (DESI) Milky Way Survey, among
others. Inferring
stellar parameters beyond V$\sim$18 will open up studies of the Milk Way's outer halo, the halo of M31, and stellar
populations in local dwarf galaxies.

The immediate challenge is to design an optimized, nested set of training samples that connect data from the surveys
above.  This nested set will span high-S/N to low-S/N and high spectral resolution to low spectral resolution for
sufficiently large, overlapping stellar samples.  Subsets will have astroseismology from TESS\footnote{TESS is NASA's
Transiting Exoplanet Survey Satellite.} and PLATO.\footnote{PLATO is ESA's PLAnetary Transits and Oscillations
mission.}  Using simulated spectra with known input parameters, we will test methods for ``label transfer'' from
information-rich spectra to information-poor spectra as we work down to fainter magnitudes, landing eventually at
multi-band photometry alone. Within this nested set, low-resolution FOBOS data will fill in gaps at both high-S/N,
where we will be training FOBOS data on higher resolution spectroscopy, as well as lower-S/N where we will be training
photometry on FOBOS spectroscopy.  The success of this multi-layered label transfer depends not only on the size of the
training sets we can access or observe, but on how representative they are.  Label transfer to WFIRST imaging of the
M31 halo, or Local Group dwarfs in either hemisphere, is a particular concern.  We will test it by evaluating label
recovery on simulated stellar spectra with cosmologically-informed formation histories for M31 and dwarf galaxies,
suitably differentiated from the Milky Way stars that anchor the training network.


\subsection{Addressing Data Science Challenges and Designing FOBOS Training Sets}
\label{sec:survey}


Our team includes leading experts on data science applications to
astronomy and, specifically, LSST.  We will also use our established
connections to LSST's Informatics and Statistics Science Collaboration
(ISSC) to advertise, recruit, and coordinate efforts to tackle the Data
Science Challenges described in Section \ref{sec:goals}.  Our proposal
request includes two community workshops to motivate progress and discuss
results. At the end of the proposal period, we will publish the results
and developed software packages.

Our data-science challenges require work on simulated
imaging$+$spectroscopic data sets where input physical properties (e.g.,
redshift) can be compared to output recovered values.  Simulated imaging
data (e.g., from LSST and WFIRST) are in-hand, while mock spectroscopy
will be provided by a FOBOS instrument simulator, an initial version of
which has already been developed.  Further advances to be supported by
this proposal include improved error modeling and simulating systematic
effects from detector artifacts, image quality aberrations informed by
the emerging detailed optical design, and variable observing conditions.

The resulting success in addressing each data-science challenge will
define a level of readiness and set requirements on desired FOBOS
training sets, including number of sources, pointings, magnitude limits,
signal-to-noise thresholds, and observing conditions.  Preliminary
observing design and a description of required operational modes to
efficiently observe these training sets will begin with this proposal.
Operational modes will set requirements on target aggregation and
prioritization systems, field acquisition speed, field rotation range,
zenith avoidance zone, reconfiguration time, calibrations, read-out
time, quick-look reduction software and processing rates.  We will
develop integrated program concepts that efficiently combine required
observations.  Detailed survey and execution plans will be completed in
the next phase of this project (MSRI-2).  Roughly 20\% of Keck observing
time is open to the public, and as in previous federally-funded
projects, we fully expect that Senior Personnel at Keck institutions
will be successful in collaborative efforts to secure significant
amounts of additional telescope observing time to enable rapid, public
release of FOBOS training data \citep[e.g.,][]{newman13}.

\begin{enumerate}[rightmargin=0.2cm,leftmargin=0.2cm]
%
\chal{phot}
%
\item[] {\textsf {\large Data-Science Challenge \ref{phot}: Apply deep-
learning algorithms to infer physical properties of galaxies at
$z$$\sim$2 using using photometry.}} The range of observed spectral
types is well-constrained by broad-band imaging (Figure \ref{fig:SOM}),
suggesting a far greater potential for imaging data to reveal physical
properties with sufficient training than conventional modeling of
spectral energy distributions (SEDs) would suggest.  The challenge here is to identify the extent to which machine
learning can deliver SDSS-like information --- e.g., star-formation histories,
stellar-population properties, dust content, inflow/outflow properties,
and stellar masses --- and determine design parameters for future training sets that will enable such inferences for millions of imaged galaxies at $z$$\sim$2.
%
\end{enumerate}

\begin{figure}[h!]
%
\vskip -0.1in
%
\includegraphics[width=\textwidth]{figs/Hemmati18_Fig8_VVDS_spec.png}
%
\caption{\small {\it Left}: A Self-Organizing Map
\citep[SOM;][]{1990Natur.346...24K} from \citet{hemmati18} encoding the
relation between colors in an LSST+WFIRST-like color space and redshift,
$z$.  Position in the SOM is associated with a position in the
multi-dimensional broad-band color space of galaxies.  Galaxies observed
in this space are assigned $z$ values based on the median photo-$z$ of
galaxies from the CANDELS survey \citep[color
bar;][]{2011ApJS..197...35G}.  Such SOMs can be used to optimally define
spectroscopic training samples for use with imaging surveys.  {\it
Right}: Galaxy spectra from VVDS \citep{2005A&A...439..845L}; black
crosses near the top and bottom of the SOM are plotted in the top and
bottom panels, respectively.  Note the similarity of the high-resolution
spectra associated within the SOM, suggesting that a systematic
spectroscopic exploration of the LSST color space would have
far-reaching benefits to the science return of the mission beyond the
photo-$z$ application.}
%
\label{fig:SOM}
%
Self-Organizing Maps (SOM, Fig.~\ref {fig:SOM}) provide a
state-of-the-art representation of a high-dimensional input space in
projected 2D grid cells, allowing us to benchmark sampling of the
photometric color space under various training set designs.  We will use
Bayesian Optimization techniques to evaluate the success of simulated training sets against the fidelity of full  
cosmological analyses that employ them.  This will enable extremely rapid exploration of the
optimal design space.

\end{figure}

% The complete photo-$z$ training survey described in \citet{newman15}
% would require 15 independent pointings, each spanning 0.1 deg$^2$ with
% a target density of 6 arcmin$^{-2}$ (8 arcmin$^{-2}$ when including $z
% > 1.5$ galaxies accessible in the UV with Keck-FOBOS), perfectly
% matched to the Keck-FOBOS field-of-view and target density.  With a
% conservative exposure time of 100 hours to reach 75\% redshift
% completeness for 40,000 galaxies with $i_{\rm AB} < 25.3$, the Neman
% survey would require 400 nights.  Challenge \ref{photoz} would reduce
% the required survey duration by a factor of at least four.  Meanwhile
% the extreme depths and flux-limited selection are likely also
% requirements for training sets associated with Challenges \ref{phot},
% \ref{uv}.

% A wider and shallower survey component is envisioned for Challenges
% \ref{lowsnr} and \ref{gaia}.  With 10-minute integrations, a 52
% deg$^2$ Keck-FOBOS sample of environmental diagnostics for 1 million
% galaxies could be carried out in less than 20 nights.  This program
% would sample at $z \sim 1.5$ the same cosmic volume as SDSS.  A
% program of a similar scale would provide training set data for
% inference of stellar parameters in the Milky Way.  These shallow
% programs would be integrated with the deeper components described
% above into a single survey plan.