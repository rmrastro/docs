%%%%
% -- Instrument Description
% --     FOBOS Keck White Paper 2019
%%%%


\section{Instrument Overview}
\label{sec:concept}
% \noindent \comment{1 page}

% Here's an alternative way to put in figures if we want captions on the side (to save space)
% Could introduce a new ``counter'' to count and label figures appropriately
%\centerline{\hbox{\includegraphics[width=0.6\textwidth, angle=0]{figs/FOBOSatKeck_v1.pdf}
%    \hspace{0.1cm} \vspace{2in}
%    \parbox[b]{0.3\textwidth}{\small {\bf Figure ??:} Rendering of FOBOS instrument systems deployed at the Keck II Nasmyth port.  By mounting the FOBOS spectrographs under the Nasmyth platform, other instruments like DEIMOS can maintain access to the telescope. \vspace{2cm}}}}

%%%%%%%%%%%%%%%%%%%%%%%%%%%%%%%%%%%%%%%%%%%%%%%%%%%%%%%%%%%%%%%%%%%%%%%%
\begin{figure}[h!]
\vskip -0.1in
%\includegraphics[width=\textwidth]{figs/FOBOS_FocalPlane.pdf}
\includegraphics[width=\textwidth]{figs/FOBOS_FocalPlane.pdf}
\caption{\small {\it
Left}: Rendering of FOBOS focal plane system deployed at the Keck II
Nasmyth port. {\it Right}: Rendering of the ADC and focal surface with
Starbugs mounted (red cylinders). {\it Bottom-left}: Photograph of actual Starbugs
deployed on the TAIPAN instrument.}
\label{fig:focalplane}
\end{figure}
%%%%%%%%%%%%%%%%%%%%%%%%%%%%%%%%%%%%%%%%%%%%%%%%%%%%%%%%%%%%%%%%%%%%%%%%

Mounted at the Nasmyth focus of Keck II Telescope at WMKO, FOBOS (Figure
\ref{fig:layout}) will be one of the most powerful spectroscopic
facilities deployed in the next decade. FOBOS includes a compensating
lateral atmospheric dispersion corrector (CLADC, not pictured) to ensure
that target light from all wavelengths falls on allocated fibers while
also correcting image aberrations at the edges of the 17~arcmin diameter
Keck field. Each of the CLADC lenses is $\sim$700~mm in diameter, the
first two are closely spaced with lateral relative motions of element
one supplied by a single axis of motion acting along a curve equal to
the radius of curvature of the first lens surface (1028~mm). Total
offset of this lens is rather small at $\sim1$00~mm.  The final CLADC
lens translates by $\sim$50~mm and tilts slightly to track focal plane
shift. This last element acts to correct the telecentricy error into the
fiber system and acts as the drive surface for the StarBug positioning
system \comment{ref}. Starbugs patrol a large on-sky area
($\sim$1~arcmin), enabling flexible and dynamic targeting configurations
with adjacent fibers as close as 10~arcsec.  

% Reni- Can you check my numbers on the CLADC motion? - NKM
% How much do we go into risks?  How about this - NKM

Starbugs, first proposed in 2004 \citep{2004SPIE.5495..600M}, and
later perfected by AAO for use on TAIPAN \citep{2016SPIE.9912E..1WS}
are a truly remarkable fiber positioning system. They move by {\it
walking} on the focal plane using a pair of piezo tube actuators. A
light vacuums adheres the Starbugs to the surface of the field plate
and provides the frictional normal forces needed to allow for the
walking action of the piezo tubes. Positional feed back is provided
by way of a camera imaging back illuminated fibers on the focal
plane. This system allows for a highly configurable focal plane both
in terms of target densities and configuration of the fibers within
an individual actuator. The Starbugs may be used with a single fiber
or with a bundle of fiber making up an IFU. The TIPAN instrument,
currently on sky conducting a large galaxy survey, is the proof test
for the reediness of this technology. It is worth noting that the
well Starbugs are the baseline and preferred positioning technology
for FOBOS no aspect of the current front end design precludes using a
zonal system similar to those used for MOONS, PFS, or DESI. The last
element of the CLADC can be eliminated and replaced with a zonal
actuator bed which conforms to the focal plane shape. Telecentricy
can be maintained by alignment of the actuator axis to the incoming
beam as is currently being designed for the SDSS-V robotic
focal-plane system.

A total of 1800 fibers with 150-$\mu$m core diameter are deployed at the
curved focal plane. Microlens fore-optics convert the f/15 Keck input
beam to a faster f/3 focal ratio, which both demagnifies the entrance
aperture and allows for better coupling to the fiber numerical aperture
which minimizes losses from focal ratio degradation.  The focal-plane
plate rotates and translates to follow image positions as the telescope
tracks across the sky. The fiber run is kept as short as possible, less
than 10~m, to maintain high throughput at UV wavelengths (a 10~m
Polymicro Silica fiber transmits $\sim$70\% and $\sim$85\% of light at
310~nm and 350~nm, respectively). Special care is given to stress-relief
cabling to minimize instabilities (e.g., variable focal ratio
degradation) over the fiber run. In order to maintain the highest
possible transition efficiency there are no connectors used within the
fiber run.  When FOBOS is not in use the focal plane unit detaches from
the front end module that houses the ADC and associated robotics.  The
focal plane unit is then stored with the spectrographs on the Nasmyth
platform allowing the front end module to be transferred to any of the
instrument park positions.  All other Keck-II instruments can still be
used without modification.

%%%%%%%%%%%%%%%%%%%%%%%%%%%%%%%%%%%%%%%%%%%%%%%%%%%%%%%%%%%%%%%%%%%%%%%%
\begin{figure}[h!]
\vskip -0.1in
\includegraphics[width=0.96\textwidth]{figs/FOBOS_spec_optical-CAD.png}
\caption{\small Optical design (left) and mechancial rendering (right) of a 4-channel FOBOS spectrograph employing catadioptric cameras.
Light from a 600-fiber pseudo-slit strikes a collimating mirror and then
passes back through subsequent dichroics before entering each grating-camera unit.}
\label{fig:spectrograph}
\end{figure}
%%%%%%%%%%%%%%%%%%%%%%%%%%%%%%%%%%%%%%%%%%%%%%%%%%%%%%%%%%%%%%%%%%%%%%%%

Sets of 600 fibers each, feed three identical spectrographs (Fig
\ref{fig:layout} and \ref{fig:spectrograph}). Each spectrograph uses
a series of dichroics to divide the 259~mm collimated beam into four
wavelength channels, providing an instantaneous broad-band coverage
from 0.31--1 $\mu$m. Fused-silica etched (FSE) gratings provide
mid-channel spectral resolutions of $R\sim3500$ at high diffraction
efficiency in each channel. The dispersed light is focused by an
f/1.1 catadioptric camera\footnote{Based on the camera design for the
Multi-Object Optical and Near-infrared Spectrograph (MOONS) on the
Very Large Telescope (VLT).} and recorded by an on-axis 4k$\times$4k
CCD mounted at the center of the first camera lens element.
Spectrographs are mounted in a permanent temperature-controlled
housing on the Nasmyth deck, whereas the focal-plane system can be
unmounted and stowed alongside existing Keck instruments. The
end-to-end instrument throughput peaks at 60\% and is greater than
30\% at {\it all} wavelengths.

FOBOS will include observatory level systems for precise instrument
calibration using dome-interior screen illumination, a metrology system
for accurate fiber positioning, and guide cameras for field acquisition
and guiding. \comment{make the following consistent with what we say in
the summary}. Initial deployment of the focal-plane will focus on a
single-fiber format, with a secondary deployment of multi-format fiber
bundles to follow. Additional instrument upgrades include integration of
fibers that feed additional spectrographs --- these spectrographs could
provide increased multiplex capacity, higher spectral resolution, and/or
observe different spectral regions --- and additional front-end sensing
equipment that fully support and benefit from image corrections with
Ground-Layer Adaptive Optics.

