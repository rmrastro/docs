%%%%
% -- Galaxies Science Cases
% --     FOBOS Keck White Paper 2019
%%%%

\subsection{Globular cluster tracers}

Globular clusters and (GCs) and planetary nebulae (PNe) are valuable chemodynamical tracers of the outskirts of nearby galaxies but are difficult targets for current instruments because they are sparsely distributed.  FOBOS
will be a game changer in this area.  For dynamical studies of nearby galaxies with $\mathcal{M_\ast/M_\odot}
\lesssim 10^{11}$, which typically host $\lesssim1500$ GCs
\citep{2013ApJ...772...82H}, FOBOS makes it possible to acquire spectra for nearly all GCs located within $\sim$50 kpc
from the host galaxy in a single night.  These data will allow us to map the chemodynamics of massive galaxy halos and
infer orbital families as a function of stellar-population properties. Additionally, these data will inform models of
GC formation in the context of the larger galaxy population.

\subsection{Cluster galaxy populations}

Presently, it is very expensive to conduct systematic spectroscopic
studies of the various galaxy types in rich galaxy clusters, like
Coma, due to their angular spread on the sky. With FOBOS's flexible
fiber-positioning system and 17-arcminute FOV, it will be possible to
simultaneously (and efficiently) build up an unprecedented library of
spectroscopic redshifts and stellar-population parameters of galaxies
in clusters towards intermediate redshift. Follow-up FOBOS
observations using its deployable mini-IFUs will allow us to
simultaneously obtain resolved spectroscopy for 10s of these cluster
galaxies, enabling us to associate internal structures/properties of
the galaxies with their host cluster.

\subsection{Internal structure of high-$z$ galaxies}

MaNGA \citep{bundy15} and other large IFU surveys are defining the $z=0$ benchmark for how internal structure is organized across the galaxy population.  To understand and test ideas for how this internal structure emerged, we require spatially-resolved observations at $z = 1$--2, just after the peak formation epoch.  Indeed, Keck has pioneered such observations \citep[e.g.,][]{erb04, miller11,law09}, but samples have been limited to a few hundred sources.  FOBOS in multiplex IFU-mode will obtain resolved spectroscopy for thousands of galaxies.  Bright optical emission line tracers will reveal gas-phase structure and kinematics across unprecedented numbers of early galaxies.  Stacking restframe $\lambda \approx 4500$ spectra will enable radial stellar population analyses to constrain how stellar components formed and assembled.  While initially limited to coarse spatial scales, ground-layer adaptive optics (GLAO) would be transformative for this science.  A corrected FWHM of 0.2-0.3 arcsec would enable fine-sampling IFUs to probe smaller galaxies and study sub-structure on 1--2 kpc scales.


\subsection{Role of environment at $z=1$--$2$ }

Global galaxy properties and their internal structure are influenced by their environment.  The vast increase in multiplex and high sampling density will allow FOBOS to map out galaxy environments at the group scale ($\mathcal{M_\ast/M_\odot}
\lesssim 10^{13}$), and with sufficient exposure time, for tens of thousands of satellites down to sub-L$^*$
luminosities.  Taking this a step further thanks to deep, wide-field imaging surveys (e.g., LSST), we will study the
feasibility of a 1M-object environmental survey at $z=1$--$2$ .  The goal is to use improved photo-$z$s, strong priors
on spectral types, and new machine-learning techniques to deliver {\it spectroscopic} redshifts (with $\lesssim$300
km/s s accuracy) at the lowest signal-to-noise possible. Reductions by factors of 4--5 in exposure time would enable
FOBOS to complete a 1M galaxy environment survey at $z=1$--$2$ in just 20-30 nights.

%, 2011AJ....142...72E, 2017AJ....154...28B}

\noindent\comment{Cooper, further comments?}

\begin{figure}[h!]
%
\vskip -0.1in
%
\includegraphics[width=\textwidth]{figs/qso_LightEcho_v1.pdf}
%
\caption{{\it Top}: Quasar ``Light Echos'' revealed in a simulated tomographic IGM map in the immediate environs of a quasar (gold star) with several sightlines indicated \citep[from][]{2018arXiv181005156S}.  {\it Botton}: The ionizing flux within the echo's extent enhances transmission of Ly$\alpha$ photons impinging on absorbers along the line-of-sight.}
%
\label{fig:LightEcho}
%
\end{figure}


\subsection{The proto-galaxy ecosystem at $z$$\sim$2}
\label{sec:z2galaxies}

With surveys like MOSDEF \citep{kriek15} and KBSS \citep[e.g.,][]{steidel14}, MOSFIRE has provided powerful new
insights into galaxies at the $z \sim 2$ peak formation epoch.  A deeper understanding of this important period
requires a statistical characterization of the entire galaxy ``ecosystem,'' including not only the galaxies themselves
but their gas-filled environments.  The goal is to build a comprehensive picture of the physical processes that fuel
proto-galaxy growth, shape their properties, and influence their environment.

Using Ly$\alpha$ absorption in background galaxies, a tomographic map of the intergalactic medium (IGM) surveyed by MOSDEF and KBSS is a key first step.  Its promise and application was demonstrated at Keck by \citet{lee14} which motivates FOBOS's UV sensitivity, target flexibility and multiplex.  \citet{2018arXiv181005156S} take IGM tomography one step further, demonstrating with simulated observations that quasar ``light echos,'' spatial signatures of the expanding ionization front of a newly activated quasar, can be detected and used to infer opening angles and deconstruct the quasar's accretion history (see Fig \ref{fig:LightEcho}).  

The volume density and chemistry of gas in between galaxies
\noindent\comment{Hennai, KG, Prochaska, Burchett: comments? further material to add?}


\subsection{Ly$\alpha$ morphology and kinematics of lensed, magnified
galaxies at $z$$\sim$2--3}

\noindent\comment{Siana}

\subsection{The budget of ionizing photons at $z$$\gtrsim$2.5}

\noindent\comment{Shapley, Siana}


% From George:
% - fill out case for probing both galaxies and their “gas-filled
%   environments”
%    - make it more explicit that getting large numbers of redshifts
%      would make it possible to trace out large-scale structure in
%      detail
%    - enables studies of galaxy properties as a function of environment
%
% - also mention targeting galaxies along QSO lines of sight
%    - much higher target density than with LRIS, DEIMOS over larger FOV.
%
% - Worth discussing Lyman-alpha or metal-line tomography?  
%
% - More quantitative comparisons with existing data sets?
%    - What key science questions can FOBOS address that many years of
%      LRIS and DEIMOS observations have not been able to?  Surely some
%      level of the spectral tagging and photo-z training can be done
%      (and surely is being done) with existing data.  Is FOBOS going to
%      be a huge leap, or will it mainly be cleaning up neglected corners
%      of parameter space?
%
% - More excited to hear about how the FOBOS spectra will be used for
%   science directly, instead of support for LSST

%-----------------------------------------------------------------------
