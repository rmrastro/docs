%%%%
% -- Galaxies Science Cases
% --     FOBOS Keck White Paper 2019
%%%%

\subsection{A Comprehensive Picture of the Proto-galaxy Ecosystem}
\label{sec:galaxies}

% From George:
% - fill out case for probing both galaxies and their “gas-filled
%   environments”
%    - make it more explicit that getting large numbers of redshifts
%      would make it possible to trace out large-scale structure in
%      detail
%    - enables studies of galaxy properties as a function of environment
%
% - also mention targeting galaxies along QSO lines of sight
%    - much higher target density than with LRIS, DEIMOS over larger FOV.
%
% - Worth discussing Lyman-alpha or metal-line tomography?  
%
% - More quantitative comparisons with existing data sets?
%    - What key science questions can FOBOS address that many years of
%      LRIS and DEIMOS observations have not been able to?  Surely some
%      level of the spectral tagging and photo-z training can be done
%      (and surely is being done) with existing data.  Is FOBOS going to
%      be a huge leap, or will it mainly be cleaning up neglected corners
%      of parameter space?
%
% - More excited to hear about how the FOBOS spectra will be used for
%   science directly, instead of support for LSST

IGM tomography: Understand the $z \sim 2$ galaxy ``ecosystem,'' including not only the galaxies themselves but their gas-filled environments.
The goal is to build a comprehensive picture of the physical processes that fuel proto-galaxy growth, shape their
internal structure, and influence their environment.

\noindent\comment{Cooper?} Build SDSS-like statistics for galaxies at this key cosmic epoch.  Exploit
short spectroscopic exposures in combination with photometry to provide
environmental diagnostics for 1M galaxies at $z$=1--2.  Photometric
redshifts, while acceptable in large cosmological analyses, wash out
information about the local position of galaxies with respect to one
another.  To characterize a galaxy's local environment and identify its
neighbors requires (observationally expensive) spectroscopic redshifts.  However, with improved photo-$z$s available
from Challenge \ref{photozs} and strong priors on spectral types
(Challenge \ref{phot}), the challenge here is to push machine-learning techniques to deliver
\emph{spectroscopic} redshifts (with 300 km s$^{-1}$ accuracy) at the lowest signal-to-noise possible.  Reductions by
factors of 4--5 in exposure time would enable FOBOS to complete a 1M galaxy environment survey at $z=1$--$2$ in just
20-30 nights.

\noindent\comment{Siana -- Clusters?}

\noindent\comment{Westfall/Bundy -- Resolved spectroscopy}

