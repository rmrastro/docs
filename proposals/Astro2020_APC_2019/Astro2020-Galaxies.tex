%%%%
% -- Galaxies Science Cases
% --     FOBOS Keck White Paper 2019
%%%%

\subsection{The dark and luminous content of nearby galaxies}

Using globular cluster and planetary nebulae as tracers, FOBOS will
dramatically advance dynamical studies of nearby galaxies with
$\mathcal{M_\ast/M_\odot} \lesssim 10^{11}$, capturing the majority
of the $\sim$1000 GCs located within $\sim$50 kpc of typical hosts
\citep[see][]{2013ApJ...772...82H} and tightly constraining their
dark matter halos. FOBOS's multi-IFU mode will additionally provide
powerful insight on the origin of dwarf galaxies, both compact and
ultra-diffuse (UDGs), in the field and in nearby clusters like Coma
and Virgo.

% which typically host $\lesssim1500$ GCs
% \citep{2013ApJ...772...82H}, FOBOS makes it possible to acquire
% spectra for nearly all GCs located  from the host
% galaxy in a single night. These data will allow us to map the
% chemodynamics of massive galaxy halos and infer orbital families as a
% function of stellar-population properties. Additionally, these data
% will inform models of GC formation in the context of the larger
% galaxy population.

% \subsection{Cluster galaxy populations}

% Presently, it is very expensive to conduct systematic spectroscopic
% studies of the various galaxy types in rich galaxy clusters, like
% Coma, due to their angular spread on the sky. With FOBOS's flexible
% fiber-positioning system and 17-arcminute FOV, it will be possible to
% simultaneously (and efficiently) build up an unprecedented library of
% spectroscopic redshifts and stellar-population parameters of galaxies
% in clusters towards intermediate redshift. Follow-up FOBOS
% observations using its deployable mini-IFUs will allow us to
% simultaneously obtain resolved spectroscopy for 10s of these cluster
% galaxies, enabling us to associate internal structures/properties of
% the galaxies with their host cluster.

\subsection{Internal structure of galaxies at intermediate redshift}

MaNGA \citep{bundy15} and other large IFU surveys are defining the
$z=0$ benchmark for how internal structure is organized across the
galaxy population. To understand and test ideas for how this internal
structure emerged, we require spatially-resolved observations at $z =
1$--2, just after the peak formation epoch. Indeed, Keck has
pioneered such observations \citep[e.g.,][]{erb04, miller11,law09},
but samples have been limited to a few hundred sources. FOBOS in
multiplex IFU-mode will obtain resolved spectroscopy for thousands of
galaxies. Bright optical emission line tracers will reveal gas-phase
structure and kinematics across unprecedented numbers of early
galaxies. Stacking restframe $\lambda \approx 4500$ spectra will
enable radial stellar population analyses to constrain how stellar
components formed and assembled. While initially limited to coarse
spatial scales, ground-layer adaptive optics (GLAO) in combination with FOBOS would be
transformative for this science. A corrected FWHM of 0.2-0.3 arcsec
would enable fine-sampling IFUs to probe smaller galaxies and study
sub-structure on 1--2 kpc scales.

\subsection{Role of environment at $z=1$--$2$ }

Its increased multiplex and high sampling density will allow FOBOS to
map out environmental effects on galaxy evolution at the group scale
($\mathcal{M_\ast/M_\odot} \lesssim 10^{13}$), and with sufficient
exposure time, for tens of thousands of satellites down to sub-L$^*$
luminosities. Thanks to deep, wide-field imaging surveys, like LSST,
a 1M-object environmental survey at $z=1$--$2$ may then be possible
using improved photo-$z$s, strong priors on spectral types, and new
machine-learning techniques to deliver {\it spectroscopic} redshifts
(with $\lesssim$300 km/s accuracy) at the lowest signal-to-noise
possible (exposure times reduced by factors of 4--5).

%, 2011AJ....142...72E, 2017AJ....154...28B}

% \noindent\comment{Cooper, further comments?}

% TODO: May be too small...
\begin{wrapfigure}{l}{0.6\textwidth}
%
\includegraphics[width=0.6\textwidth]{figs/qso_LightEcho_v1.pdf}
%
\caption{{\it Top}: Quasar ``Light Echos'' revealed in a simulated
tomographic IGM map in the immediate environs of a quasar (gold star)
with several sightlines indicated
\citep[from][]{2018arXiv181005156S}. {\it Bottom}: The ionizing flux
within the echo's extent enhances transmission of Ly$\alpha$ photons
impinging on absorbers along the line-of-sight.}
\label{fig:LightEcho}
\end{wrapfigure}

%\begin{figure}[h!]
%%
%\vskip -0.1in
%%
%\includegraphics[width=\textwidth]{figs/qso_LightEcho_v1.pdf}
%%
%\caption{{\it Top}: Quasar ``Light Echos'' revealed in a simulated tomographic IGM map in the immediate environs of a quasar (gold star) with several sightlines indicated \citep[from][]{2018arXiv181005156S}.  {\it Bottom}: The ionizing flux within the echo's extent enhances transmission of Ly$\alpha$ photons impinging on absorbers along the line-of-sight.}
%%
%\label{fig:LightEcho}
%%
%\end{figure}

\subsection{The $z$$\sim$2 galaxy ecosystem}
\label{sec:z2galaxies}

With surveys like MOSDEF \citep{kriek15} and KBSS
\citep{steidel14}, MOSFIRE has provided powerful new
insights into early galaxies at the $z$$\sim$2 peak-formation epoch.
However, a complete picture of the galaxy ``ecosystem'' at this key
epoch must also consider the gas-filled environments. Using
Ly$\alpha$ absorption in background galaxies, a tomographic map of
the intergalactic medium (IGM) in regions surveyed by MOSDEF and KBSS
is a key first step. The promise of this approach, demonstrated at
Keck by \citet{lee14}, motivates FOBOS's UV sensitivity, target
flexibility, and multiplex for tomographic mapping of large-scale
structure, including protoclusters \citep{lee16}, voids
\citep{krolewski18}, and filaments \citep{horowitz19}.
\citet{2018arXiv181005156S} take IGM tomography in a new direction,
demonstrating with simulated observations that quasar ``light echos''
--- spatial signatures of the expanding ionization front of a newly
activated quasar --- can be detected and used to infer opening angles
and deconstruct the quasar's accretion history (see Fig
\ref{fig:LightEcho}). The required FOBOS spectra can simultaneously
constrain the CIV mass density (via $\lambda\lambda$1548,1550 \AA)
and patterns of CIV enrichment on both IGM and circumgalactic scales,
revealing the imprint of galaxy fueling and feedback processes
\citep[e.g.,][]{tumlinson17}.

%The volume density and chemistry of gas in between galaxies
%\noindent\comment{Hennawi, KG, Prochaska, Burchett: comments? further material to add?}

% \subsection{Ly$\alpha$ morphology and kinematics of lensed, magnified galaxies at $z$$\sim$2--3}

% \noindent\comment{Siana}

% \subsection{The budget of ionizing photons at $z$$\gtrsim$2.5}

% \noindent\comment{Shapley, Siana}


% From George:
% - fill out case for probing both galaxies and their “gas-filled
%   environments”
%    - make it more explicit that getting large numbers of redshifts
%      would make it possible to trace out large-scale structure in
%      detail
%    - enables studies of galaxy properties as a function of environment
%
% - also mention targeting galaxies along QSO lines of sight
%    - much higher target density than with LRIS, DEIMOS over larger FOV.
%
% - Worth discussing Lyman-alpha or metal-line tomography?  
%
% - More quantitative comparisons with existing data sets?
%    - What key science questions can FOBOS address that many years of
%      LRIS and DEIMOS observations have not been able to?  Surely some
%      level of the spectral tagging and photo-z training can be done
%      (and surely is being done) with existing data.  Is FOBOS going to
%      be a huge leap, or will it mainly be cleaning up neglected corners
%      of parameter space?
%
% - More excited to hear about how the FOBOS spectra will be used for
%   science directly, instead of support for LSST

%-----------------------------------------------------------------------
