%%%%
% -- Overview Material
% --     FOBOS Keck White Paper 2019
%%%%

\vspace{-0.5cm}
%\setcounter{secnumdepth}{0}
\section{Scientific Motivation}
%\setcounter{secnumdepth}{1}

\subsection{Community Need} The need for spectroscopic follow-up in
the LSST era was made clear in the National Research Council's 2015
report, ``Optimizing the U.S. Ground-Based Optical and Infrared
Astronomy System'' \citep{NAP21722}:
%
\noindent\begin{center}\mbox{\parbox{0.9\linewidth}{
%
The National Science Foundation should support the development of a
wide-field, highly multiplexed spectroscopic capability on a medium- or
large-aperture telescope in the Southern Hemisphere to enable a wide
variety of science, including follow-up spectroscopy of Large Synoptic
Survey Telescope targets. Examples of enabled science are studies of
cosmology, galaxy evolution, quasars, and the Milky Way.
%
}}\end{center}

Workshops organized by the National Optical Astronomy Observatory
(NOAO) in 2013 and 2016 reported specific spectroscopic needs for
LSST follow-up in all science areas. In particular, the 2016 report
notes that a critical resource in need of prompt development is to
``Develop or obtain access to a highly multiplexed, wide-field
optical multi-object spectroscopic capability on an 8m-class
telescope.''

FOBOS takes a critical first step in addressing these needs using an
existing telescope to achieve a final cost $\approx$20 times less
than wide-area spectroscopic telescopes of the future, such as the
Mauna Kea Spectroscopic Explorer \citep[MSE,][]{mse2018} and SpecTel
(see Astro2020 APC White Paper). Compared to the Prime Focus
Spectrograph (PFS) on Japan's Subaru Telescope, FOBOS would be
1.7$\times$ faster, provide unique UV sensitivity (0.31--1 $\mu$m
compared to 0.38--1.25 $\mu$m with PFS), and offer higher-density,
more flexible target sampling over ``deep-drilling'' fields. Unlike
PFS, FOBOS would be operated on a U.S.\ telescope with dedicated
U.S.\ access and a commitment to supporting U.S.-led imaging
facilities.

% FOBOS is a modular instrument composed of three major components (see
% schematic above): (1) an atmospheric dispersion corrector (ADC), (2)
% a flexible focal-plane system that deploys as many as 1800
% free-roaming Starbug positioners that sample a 17-arcminute diameter
% field, and (3) a bank of three temperature-controlled bench
% spectrographs that provide $R \sim 3500$ spectroscopy over an
% instantaneous bandpass of 0.31-1$\mu$m. The spectrographs are fed
% light from the focal plane by a short fiber run ($<$10 m) through a
% stress-relief cabling system to minimize throughput losses.

% The focal-plane system allows for flexible targeting and provides
% multiple sampling formats. In single-fiber mode, each Starbug carries
% a single optical fiber with a 150 $\mu$m diameter core fed by
% demagnifying fore-optics yielding a 0.9-arcsec diameter on-sky
% aperture. In multi-IFU mode, FOBOS deploys a different suite of
% Starbugs carrying fiber bundles coupled to lenslet arrays with finer
% spatial sampling. Its flexible focal plane allows efficient observing
% strategies that combine multiple programs and can dynamically respond
% to changing conditions and targets of opportunity.

\subsection{Key Science Goals}

With its high multiplex and Keck's large aperture, FOBOS will enable significant progress in multiple science areas by
providing much-needed large and deep spectroscopic samples.  These unprecedented data sets will be scientific gold
mines for the U.S.~community in their own right, but when combined with novel observations from forthcoming facilities,
transformational advances are possible.  These include 1) charting the assembly history of the Milky Way, M31, and
Local Group dwarf galaxies by combining deep FOBOS spectroscopy with wide spectroscopic campaigns (e.g., DESI
Bright-Time Survey and SDSS-V Milky Way Mapper), GAIA data, and panoramic imaging from LSST, Euclid, and WFIRST; 2)
mapping the baryonic ecosystem at $z \sim 2$--3 and linking it to evolving populations at lower redshifts by training
photometric diagnostics that transfer detailed spectroscopic knowledge to billion-plus galaxy samples provided by
future all sky surveys; 3) dramatically enhancing cosmological probes using panoramic deep imaging and
cross-correlation techniques with Stage-IV CMB observations thanks to precise calibration of photometric redshifts and
redshift distributions.

FOBOS addresses these goals by achieving high multiplex while optimizing for sensitivity over area.  Future imaging
data will routinely reach $i_{\rm AB} = 25$, yielding target densities of 42 arcmin$^{-2}$.  FOBOS achieves an average
on-sky sampling density of 6 arcmin$^{-2}$, and close packing would allow a maximum target density of $\sim$30
arcmin$^{-2}$.  These capabilities allow FOBOS to collect large samples of very faint sources with highly efficient
observing strategies.

Meanwhile, innovations in astrostatistics and machine learning in particular promise powerful synergies between these
information-rich but volume-limited FOBOS samples and the vast cosmic volumes that will be surveyed by LSST, Euclid,
and WFIRST.  As we discuss below, these synergies underlie all three of FOBOS's broad science goals.  Given this and
following past examples at Keck \citep[e.g., DEEP2][]{newman13}, FOBOS will play a leading role in obtaining and
publicly delivering critical, enabling data sets that leverage major project investments by the U.S.~community.  These
data sets will include raw data, fully-reduced, calibrated, and processed spectra, and high-level derived data
products, including redshifts, measurements of spectral features, and inferred physical information (e.g., stellar
parameters, galaxy star formation histories, environmental catalogs).  In addition to the 20\% of Keck observing time
already open to the public, additional FOBOS access may be possible through key programs allocated a specific fraction
of FOBOS fiber-hours and combined with other programs into multi-cycle observing campaigns.  The FOBOS team will engage in a broad campaign of community engagement in the next year to further define science goals and community access models.


