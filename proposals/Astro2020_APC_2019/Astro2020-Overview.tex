%%%%
% -- Overview Material
% --     FOBOS Keck White Paper 2019
%%%%

\vspace{-0.5cm}
%\setcounter{secnumdepth}{0}
\section{Scientific Motivation}
%\setcounter{secnumdepth}{1}

\subsection{Community Need} The need for spectroscopic follow-up in the LSST era was made clear in
the National Research Council's 2015 report, ``Optimizing the U.S.
Ground-Based Optical and Infrared Astronomy System'' \citep{NAP21722}:
%
\noindent\begin{center}\mbox{\parbox{0.95\linewidth}{
%
The National Science Foundation should support the development of a
wide-field, highly multiplexed spectroscopic capability on a medium- or
large-aperture telescope in the Southern Hemisphere to enable a wide
variety of science, including follow-up spectroscopy of Large Synoptic
Survey Telescope targets. Examples of enabled science are studies of
cosmology, galaxy evolution, quasars, and the Milky Way.
%
}}\end{center}

Workshops organized by the National Optical Astronomy Observatory (NOAO)
in 2013 and 2016 reported specific
spectroscopic needs for LSST follow-up in all science areas.  In
particular, the 2016 report notes that a critical resource in need of
prompt development is to ``Develop or obtain access to a highly
multiplexed, wide-field optical multi-object spectroscopic capability on
an 8m-class telescope.''  

FOBOS takes a critical first step in addressing these needs using an existing telescope to achieve a final cost
$\approx$20 times less than wide area spectroscopic telescopes of the future, such as the Mauna Kea Spectroscopic
Explorer (MSE, ref) and SpecTel (see Astro2020 APC White Paper).  Compared to the Prime Focus Spectrograph (PFS) on
Japan's Subaru Telescope, FOBOS would be 1.7$\times$ faster, provide unique UV sensitivity (0.31--1 $\mu$m compared to
0.38--1.25 $\mu$m with PFS), and offer higher-density, more flexible target sampling over ``deep-drilling'' fields.
Unlike PFS, FOBOS would be operated on a U.S.\ telescope with dedicated U.S.\ access and a commitment to supporting
U.S.-led imaging facilities.

% FOBOS is a modular instrument composed of three major components (see
% schematic above): (1) an atmospheric dispersion corrector (ADC), (2)
% a flexible focal-plane system that deploys as many as 1800
% free-roaming Starbug positioners that sample a 17-arcminute diameter
% field, and (3) a bank of three temperature-controlled bench
% spectrographs that provide $R \sim 3500$ spectroscopy over an
% instantaneous bandpass of 0.31-1$\mu$m. The spectrographs are fed
% light from the focal plane by a short fiber run ($<$10 m) through a
% stress-relief cabling system to minimize throughput losses.

% The focal-plane system allows for flexible targeting and provides
% multiple sampling formats. In single-fiber mode, each Starbug carries
% a single optical fiber with a 150 $\mu$m diameter core fed by
% demagnifying fore-optics yielding a 0.9-arcsec diameter on-sky
% aperture. In multi-IFU mode, FOBOS deploys a different suite of
% Starbugs carrying fiber bundles coupled to lenslet arrays with finer
% spatial sampling. Its flexible focal plane allows efficient observing
% strategies that combine multiple programs and can dynamically respond
% to changing conditions and targets of opportunity.

\subsection{Key Science Areas}

FOBOS is optimized for ``deep-drilling'' on faint targets. In the LSST era where target densities to $i_{\rm AB} = 25$
will be routine and reach 42 arcmin$^{-2}$, FOBOS yields an average, on-sky sampling density of 6 arcmin$^{-2}$,
similar to the DEEP2 Survey target density of 9 arcmin$^{-2}$. Close packing of Starbugs would allow a maximum target
density of $\sim$30 arcmin$^{-2}$ over small portions of its full FOV. The more sparse and wide-field PFS instrument
with 2400 fixed-format zonal positioners achieves 0.6 arcmin$^{-2}$. With Starbugs positioners, FOBOS will provide
single-fiber and multiplexed IFU modes plus targeting flexibility and fast reconfiguration times that will enable
multiple, simultaneous science programs. The science cases below take full advantage of these unique aspects of FOBOS.


