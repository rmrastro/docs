% \documentclass[12pt]{article}
\documentclass[oneside,12pt]{amsart}

\usepackage{times}
\usepackage{geometry}
\geometry{letterpaper, portrait, margin=1in}
\usepackage[utf8]{inputenc}
\usepackage{enumitem,amssymb}
\usepackage{ragged2e}
\usepackage{graphicx}
\usepackage{natbib,latexsym,url,enumitem,pdfpages,multicol}
\usepackage{color}
\usepackage{wrapfig}
\usepackage{caption}
\usepackage{setspace}
\newlist{thematic}{itemize}{8}
\setlist[thematic]{label=$\square$}
\usepackage{pifont}
\newcommand{\cmark}{\ding{51}}%
\newcommand{\xmark}{\ding{55}}%
\newcommand{\done}{\rlap{$\square$}{\raisebox{2pt}{\large\hspace{1pt}\cmark}}%
\hspace{-2.5pt}}
\newcommand{\wontfix}{\rlap{$\square$}{\large\hspace{1pt}\xmark}}
%\usepackage[pagestyles]{titlesec}
%\titlespacing{\section}{0pt}{*0}{*0}
%\titlespacing{\subsection}{0pt}{0}{0}
%\titlespacing{\subsubsection}{0pt}{0}{0}

% Some fancy commenting
\definecolor{todo}{RGB}{200,0,0}
\newcommand{\comment}[2][todo]{{\color{#1}[[{\bf #2}]]}}

% User commands
\input{journaldefs}

\setlength{\parskip}{3pt}

\begin{document}
\raggedright
\huge
Astro2020 APC White Paper \linebreak

FOBOS: A Next-Generation Spectroscopic Facility \linebreak
\normalsize

\noindent \textbf{Thematic Areas:} Project Paper
  
\textbf{Principal Author:}

Name:	Kevin Bundy
 \linebreak						
Institution:  University of California, Observatories
 \linebreak
Email:  kbundy@ucolick.org
 \linebreak
Phone:  831-459-3539
 \linebreak
 
\textbf{Co-authors:} {\footnotesize K.~Bundy (PI, UCO/UCSC), K.~Westfall (Project
Scientist, UCO), N.~MacDonald (Lead Engineer, UCO), R.~Kupke
(Instrument Scientist, UCO), M.~Savage (Project Manager, UCO),
C.~Poppett (Fiber System Lead, UCB/SSL), A.~Alabi (UCSC), G.~Becker
(UCR), J.~Burchett (UCSC), P.~Capak (Caltech), A.~Coil (UCSD),
M.~Cooper (UCI), D.~Cowley (UCO), W.~Deich (UCO), D.~Dillon (UCO),
J.~Edelstein (LBNL), P.~Guhathakurta (UCSC), J.~Hennawi (UCSB),
K.-G.~Lee (IPMU), D.~Masters (JPL), T.~Miller (UCB/SSL), J.~Newman
(Pitt), J.~X.~Prochaska (UCSC), J.~Rhodes (JPL), R.~M.~Rich (UCLA),
C.~Rockosi (UCSC), A.~Romanowsky (SJSU/UCSC), D.~Schlegel (LBNL),
A.~Shapley (UCLA), B.~Siana (UCR), Y.-S.~Ting (IAS), D.~Weisz
(UCB), M.~White (UCB/LBNL), B.~Williams (UW), G.~Wilson (UCR),
M.~Wilson (LBNL), \& R.~Yan (UK)}
  \linebreak

\textbf{Abstract:} 

High-multiplex and deep spectroscopic follow-up of upcoming panoramic
deep-imaging surveys like LSST is a widely recognized and
increasingly urgent necessity. No current or planned facility at a
U.S.~observatory meets the sensitivity, multiplex, and rapid-response
time needed to exploit these future datasets. FOBOS, the Fiber-Optic
Broadband Optical Spectrograph, is a near-term fiber-based facility
that addresses these spectroscopic needs by optimizing depth over
area and exploiting the aperture advantage of the existing 10m Keck
II Telescope. The result is a uniquely blue-sensitive, high-multiplex
instrument with order-of-magnitude greater sampling density and a
factor 1.7 greater survey speed than Subaru's Prime Focus
Spectrograph (PFS). In the LSST era, FOBOS will excel at building the
deep, spectroscopic reference data sets needed to interpret vast
imaging data. At the same time, its flexible focal plane, including
deployable integral-field units (IFUs), enables an expansive range of
scientific investigations. Its key programmatic areas include, (1)
dramatic enhancements in cosmological constraints thanks to precise
photometric redshifts and determined redshift distributions, (2) a
complete description of the gaseous and galaxy environments leading
up to the peak era of galaxy formation ($z \approx 2$--5), (3) nested
stellar-parameter training sets that enable studies of the Milky Way
and M31 halo sub-structure, as well as local group dwarf galaxies,
and (4) rapid transient follow-up with a fixed-format IFU that, in
combination with Keck I instrumentation, provides instant access to
medium-resolution spectroscopy with full coverage from the UV to the
K-band.

\pagebreak


%% Executive Summary and Overview
%%%%
% -- Overview Material
% --     FOBOS Keck White Paper 2019
%%%%

\vspace{-0.5cm}
%\setcounter{secnumdepth}{0}
\section{Scientific Motivation}
%\setcounter{secnumdepth}{1}

\subsection{Community Need} The need for spectroscopic follow-up in the LSST era was made clear in
the National Research Council's 2015 report, ``Optimizing the U.S.
Ground-Based Optical and Infrared Astronomy System'' \citep{NAP21722}:
%
\noindent\begin{center}\mbox{\parbox{0.95\linewidth}{
%
The National Science Foundation should support the development of a
wide-field, highly multiplexed spectroscopic capability on a medium- or
large-aperture telescope in the Southern Hemisphere to enable a wide
variety of science, including follow-up spectroscopy of Large Synoptic
Survey Telescope targets. Examples of enabled science are studies of
cosmology, galaxy evolution, quasars, and the Milky Way.
%
}}\end{center}

Workshops organized by the National Optical Astronomy Observatory (NOAO)
in 2013 and 2016 reported specific
spectroscopic needs for LSST follow-up in all science areas.  In
particular, the 2016 report notes that a critical resource in need of
prompt development is to ``Develop or obtain access to a highly
multiplexed, wide-field optical multi-object spectroscopic capability on
an 8m-class telescope.''  

FOBOS takes a critical first step in addressing these needs using an existing telescope to achieve a final cost
$\approx$20 times less than wide area spectroscopic telescopes of the future, such as the Mauna Kea Spectroscopic
Explorer (MSE, ref) and SpecTel (see Astro2020 APC White Paper).  Compared to the Prime Focus Spectrograph (PFS) on
Japan's Subaru Telescope, FOBOS would be 1.7$\times$ faster, provide unique UV sensitivity (0.31--1 $\mu$m compared to
0.38--1.25 $\mu$m with PFS), and offer higher-density, more flexible target sampling over ``deep-drilling'' fields.
Unlike PFS, FOBOS would be operated on a U.S.\ telescope with dedicated U.S.\ access and a commitment to supporting
U.S.-led imaging facilities.

% FOBOS is a modular instrument composed of three major components (see
% schematic above): (1) an atmospheric dispersion corrector (ADC), (2)
% a flexible focal-plane system that deploys as many as 1800
% free-roaming Starbug positioners that sample a 17-arcminute diameter
% field, and (3) a bank of three temperature-controlled bench
% spectrographs that provide $R \sim 3500$ spectroscopy over an
% instantaneous bandpass of 0.31-1$\mu$m. The spectrographs are fed
% light from the focal plane by a short fiber run ($<$10 m) through a
% stress-relief cabling system to minimize throughput losses.

% The focal-plane system allows for flexible targeting and provides
% multiple sampling formats. In single-fiber mode, each Starbug carries
% a single optical fiber with a 150 $\mu$m diameter core fed by
% demagnifying fore-optics yielding a 0.9-arcsec diameter on-sky
% aperture. In multi-IFU mode, FOBOS deploys a different suite of
% Starbugs carrying fiber bundles coupled to lenslet arrays with finer
% spatial sampling. Its flexible focal plane allows efficient observing
% strategies that combine multiple programs and can dynamically respond
% to changing conditions and targets of opportunity.

\subsection{Key Science Areas}

FOBOS is optimized for ``deep-drilling'' on faint targets. In the LSST era where target densities to $i_{\rm AB} = 25$
will be routine and reach 42 arcmin$^{-2}$, FOBOS yields an average, on-sky sampling density of 6 arcmin$^{-2}$,
similar to the DEEP2 Survey target density of 9 arcmin$^{-2}$. Close packing of Starbugs would allow a maximum target
density of $\sim$30 arcmin$^{-2}$ over small portions of its full FOV. The more sparse and wide-field PFS instrument
with 2400 fixed-format zonal positioners achieves 0.6 arcmin$^{-2}$. With Starbugs positioners, FOBOS will provide
single-fiber and multiplexed IFU modes plus targeting flexibility and fast reconfiguration times that will enable
multiple, simultaneous science programs. The science cases below take full advantage of these unique aspects of FOBOS.




%%%%
% -- Local Group Science Cases
% --     FOBOS Keck White Paper 2019
%%%%

\subsection{Assembly History of the Local Group}
%Unraveling the Formation History of our Local Group of Galaxies}
\label{sec:localgroup}

Studies of individual stars in the Milky Way (MW), Magellanic Clouds, Andromeda (M31), Triangulum galaxy (M33), and
numerous dwarf satellites provide an exquisitely detailed look at specific examples of galaxy assembly and evolution.
While Gaia provides on-sky motions and photometry for 1.7 billion stars in the MW, fewer than 10\%, 0.3\%, and 0.1\% of
stars will have a full complement of astrometrics and kinematics, basic stellar parameters, and chemical abundances,
respectively.  Moreover, Gaia distance errors increase quadratically with distance.  Spectroscopy with APOGEE, the
Milky Way Mapper, and WEAVE provide supporting wide-field data sets but accounting for fainter stars requires
FOBOS-like sensitivity \citep[see][]{dey19,sanderson19}.  By carefully exploiting the overlap in these data sets, FOBOS
can link high-resolution and robust stellar information from brighter targets to stars that can only be characterized
by photometry alone.  This would enable data-driven models capable of providing photometric estimates of stellar
parameters (temperature, surface gravity, metallicity, and alpha-element abundance) for {\it all} stars in the Gaia
dataset  \citep[see][]{2015ApJ...808...16N, 2018arXiv180401530T, 2018arXiv180803278T}.

Of particular interest is the ability of future imaging surveys to increase the census of stellar streams and other
substructure by a hundredfold.  The stars in these structures are faint, however, and easily confused with background
galaxies in ground-based photometry.  With spectrocopic reference samples from FOBOS, the goal is to photometrically
reconstruct the star-formation histories of disrupted satellites and compare them with dynamical models to constrain
assembly histories and enclosed mass constraints \citep[e.g.,][]{2017ApJ...836..234S}.


Performing a similar analysis on the M31 halo is highly desireable but more challenging because individual
main-sequence stars at the distance of M31 are too faint for 10m telescopes.  Thus spectroscopic training efforts must
focus on giant stars in the M31 halo and be calibrated with hydrodynamical simulations that account for M31's differing
formation history  \citep[e.g.][]{2005MNRAS.356.1071R,li19}.



%  Complementing faint {\it Gaia} targets, FOBOS will
% enable machine-learning for photometric stellar parameter recovery
% in service of direct probes
% of radial migration, disk heating, and other assembly processes in
% the MW and M31.


% FOBOS spectroscopy of
% these structures will, e.g., constrain stream orbits and the total
% mass they enclose . These data will also
% provide age and chemical composition measurements either directly or
% via targeted machine-learning applications . FOBOS is well-suited to these observations by
% dramatically improving the survey speed over DEIMOS (by an order of
% mangitude or more) and will remain sensitive enough to push toward
% the main-sequence turn-off of MW substructures. Compared to the MW,
% studies of M31's disk and halo are more suited for dedicated FOBOS
% programs. Indeed, the dramatic improvement in survey speed will,
% e.g., yield a direct measurements of secular processes active in
% M31's disk, such as radial migration and vertical disk heating
% \citep{2013ApJ...779..103D, 2015ApJ...803...24D,
% 2019ApJ...871...11Q}. Even so, {\it Gaia} targets, particularly at
% larger distances and higher target densities in the bulge, bar and
% inner disk, will prove uniquely suitable for FOBOS spectroscopy.

%\comment{Guhathakurta, Rockosi, Weisz}

% \chal{mwhalo} 
% %
% \item[] {\textsf {\large  Data-Science Challenge \ref{mwhalo}: The
% chemical evolution and assembly history of the MW stellar halo.}}  Using current MW halo models, we will simulate FOBOS
% stellar spectroscopy of main-sequence turn-off
% and red-giant stars in these substructures within the MW that also
% leverages existing data from, e.g., APOGEE and H3.  We will build
% data-driven models based on these data to measure stellar parameters
% (temperature, surface gravity, metallicity, and alpha-element abundance)
% for all halo stars with LSST+2MASS+WISE+WFIRST multi-band photometry,
% allowing us to reconstruct the star-formation history of each disrupted
% satellite. These will be combined with dynamical data and compared with
% cosmological simulations to build a generative model for the assembly
% history of the MW stellar halo.

% \chal{m31} 
% %
% \item[] {\textsf {\large Data-Science Challenge \ref{m31}: The
% differential chemical evolution of M31 and MW.}}  A natural extension of
% Data-Science Challenge \ref{mwhalo} is to perform the same analysis for the
% halo of M31.  However, we cannot expect to obtain high-quality spectra
% of individual main-sequence stars at the distance of M31 with FOBOS.
% Moreover, training a chemical evolution model using spectra of Milky Way
% stars may lead to systematic errors:  The Milky Way and Andromeda have
% distinct evolutionary histories \citep[e.g.][]{2005MNRAS.356.1071R},
% despite being relatively similar in many other respects.  We will
% therefore obtain deep observations of giant stars in the M31 halo to
% drive a machine-learning algorithm that combines a model of the MW halo
% with results from cosmological hydrodynamical simulations to constrain
% the differential history of the MW and M31 stellar halos.

% \chal{gaia} 
% %
% \item[] {\textsf {\large Data-Science Challenge \ref{gaia}: Stellar
% parameter determinations for a billion stellar spectra.}} While
% providing on-sky motions and photometry for 1.7 billion stars in the MW,
% fewer than 10\%, 0.3\%, and 0.1\% of stars will have a full complement
% of astrometrics and kinematics, basic stellar parameters, and chemical
% abundances, respectively.  Moreover, Gaia distance errors increase
% quadratically with distance.  To realize Gaia's full potential, we will
% design FOBOS training sets that, when combined with high-resolution
% datasets from, e.g., APOGEE, WEAVE, will allow us to build data-driven
% models of the absolute magnitude (yielding distance modulus),
% temperature, surface-gravity, and stellar abundance for {\it all} stars
% in the Gaia dataset.  These data will allow us to isolate coeval
% populations in the Galactic disk that can be combined with very
% high-resolution simulations of the Milky Way to provide a detailed
% evolutionary history of our Galactic home.


%%%%
% -- Galaxies Science Cases
% --     FOBOS Keck White Paper 2019
%%%%

\subsection{The dark and luminous content of nearby galaxies}

Using globular cluster and planetary nebulae as tracers, FOBOS will
dramatically advance dynamical studies of nearby galaxies with
$\mathcal{M_\ast/M_\odot} \lesssim 10^{11}$, capturing the majority
of the $\sim$1000 GCs located within $\sim$50 kpc of typical hosts
\citep[see][]{2013ApJ...772...82H} and tightly constraining their
dark matter halos. FOBOS's multi-IFU mode will additionally provide
powerful insight on the origin of dwarf galaxies, both compact and
ultra-diffuse (UDGs), in the field and in nearby clusters like Coma
and Virgo.

% which typically host $\lesssim1500$ GCs
% \citep{2013ApJ...772...82H}, FOBOS makes it possible to acquire
% spectra for nearly all GCs located  from the host
% galaxy in a single night. These data will allow us to map the
% chemodynamics of massive galaxy halos and infer orbital families as a
% function of stellar-population properties. Additionally, these data
% will inform models of GC formation in the context of the larger
% galaxy population.

% \subsection{Cluster galaxy populations}

% Presently, it is very expensive to conduct systematic spectroscopic
% studies of the various galaxy types in rich galaxy clusters, like
% Coma, due to their angular spread on the sky. With FOBOS's flexible
% fiber-positioning system and 17-arcminute FOV, it will be possible to
% simultaneously (and efficiently) build up an unprecedented library of
% spectroscopic redshifts and stellar-population parameters of galaxies
% in clusters towards intermediate redshift. Follow-up FOBOS
% observations using its deployable mini-IFUs will allow us to
% simultaneously obtain resolved spectroscopy for 10s of these cluster
% galaxies, enabling us to associate internal structures/properties of
% the galaxies with their host cluster.

\subsection{Internal structure of galaxies at intermediate redshift}

MaNGA \citep{bundy15} and other large IFU surveys are defining the
$z=0$ benchmark for how internal structure is organized across the
galaxy population. To understand and test ideas for how this internal
structure emerged, we require spatially-resolved observations at $z =
1$--2, just after the peak formation epoch. Indeed, Keck has
pioneered such observations \citep[e.g.,][]{erb04, miller11,law09},
but samples have been limited to a few hundred sources. FOBOS in
multiplex IFU-mode will obtain resolved spectroscopy for thousands of
galaxies. Bright optical emission line tracers will reveal gas-phase
structure and kinematics across unprecedented numbers of early
galaxies. Stacking restframe $\lambda \approx 4500$ spectra will
enable radial stellar population analyses to constrain how stellar
components formed and assembled. While initially limited to coarse
spatial scales, ground-layer adaptive optics (GLAO) in combination with FOBOS would be
transformative for this science. A corrected FWHM of 0.2-0.3 arcsec
would enable fine-sampling IFUs to probe smaller galaxies and study
sub-structure on 1--2 kpc scales.

\subsection{Role of environment at $z=1$--$2$ }

Its increased multiplex and high sampling density will allow FOBOS to
map out environmental effects on galaxy evolution at the group scale
($\mathcal{M_\ast/M_\odot} \lesssim 10^{13}$), and with sufficient
exposure time, for tens of thousands of satellites down to sub-L$^*$
luminosities. Thanks to deep, wide-field imaging surveys, like LSST,
a 1M-object environmental survey at $z=1$--$2$ may then be possible
using improved photo-$z$s, strong priors on spectral types, and new
machine-learning techniques to deliver {\it spectroscopic} redshifts
(with $\lesssim$300 km/s accuracy) at the lowest signal-to-noise
possible (exposure times reduced by factors of 4--5).

%, 2011AJ....142...72E, 2017AJ....154...28B}

% \noindent\comment{Cooper, further comments?}

% TODO: May be too small...
\begin{wrapfigure}{l}{0.6\textwidth}
%
\includegraphics[width=0.6\textwidth]{figs/qso_LightEcho_v1.pdf}
%
\caption{{\it Top}: Quasar ``Light Echos'' revealed in a simulated
tomographic IGM map in the immediate environs of a quasar (gold star)
with several sightlines indicated
\citep[from][]{2018arXiv181005156S}. {\it Bottom}: The ionizing flux
within the echo's extent enhances transmission of Ly$\alpha$ photons
impinging on absorbers along the line-of-sight.}
\label{fig:LightEcho}
\end{wrapfigure}

%\begin{figure}[h!]
%%
%\vskip -0.1in
%%
%\includegraphics[width=\textwidth]{figs/qso_LightEcho_v1.pdf}
%%
%\caption{{\it Top}: Quasar ``Light Echos'' revealed in a simulated tomographic IGM map in the immediate environs of a quasar (gold star) with several sightlines indicated \citep[from][]{2018arXiv181005156S}.  {\it Bottom}: The ionizing flux within the echo's extent enhances transmission of Ly$\alpha$ photons impinging on absorbers along the line-of-sight.}
%%
%\label{fig:LightEcho}
%%
%\end{figure}

\subsection{The $z$$\sim$2 galaxy ecosystem}
\label{sec:z2galaxies}

With surveys like MOSDEF \citep{kriek15} and KBSS
\citep{steidel14}, MOSFIRE has provided powerful new
insights into early galaxies at the $z$$\sim$2 peak-formation epoch.
However, a complete picture of the galaxy ``ecosystem'' at this key
epoch must also consider the gas-filled environments. Using
Ly$\alpha$ absorption in background galaxies, a tomographic map of
the intergalactic medium (IGM) in regions surveyed by MOSDEF and KBSS
is a key first step. The promise of this approach, demonstrated at
Keck by \citet{lee14}, motivates FOBOS's UV sensitivity, target
flexibility, and multiplex for tomographic mapping of large-scale
structure, including protoclusters \citep{lee16}, voids
\citep{krolewski18}, and filaments \citep{horowitz19}.
\citet{2018arXiv181005156S} take IGM tomography in a new direction,
demonstrating with simulated observations that quasar ``light echos''
--- spatial signatures of the expanding ionization front of a newly
activated quasar --- can be detected and used to infer opening angles
and deconstruct the quasar's accretion history (see Fig
\ref{fig:LightEcho}). The required FOBOS spectra can simultaneously
constrain the CIV mass density (via $\lambda\lambda$1548,1550 \AA)
and patterns of CIV enrichment on both IGM and circumgalactic scales,
revealing the imprint of galaxy fueling and feedback processes
\citep[e.g.,][]{tumlinson17}.

%The volume density and chemistry of gas in between galaxies
%\noindent\comment{Hennawi, KG, Prochaska, Burchett: comments? further material to add?}

% \subsection{Ly$\alpha$ morphology and kinematics of lensed, magnified galaxies at $z$$\sim$2--3}

% \noindent\comment{Siana}

% \subsection{The budget of ionizing photons at $z$$\gtrsim$2.5}

% \noindent\comment{Shapley, Siana}


% From George:
% - fill out case for probing both galaxies and their “gas-filled
%   environments”
%    - make it more explicit that getting large numbers of redshifts
%      would make it possible to trace out large-scale structure in
%      detail
%    - enables studies of galaxy properties as a function of environment
%
% - also mention targeting galaxies along QSO lines of sight
%    - much higher target density than with LRIS, DEIMOS over larger FOV.
%
% - Worth discussing Lyman-alpha or metal-line tomography?  
%
% - More quantitative comparisons with existing data sets?
%    - What key science questions can FOBOS address that many years of
%      LRIS and DEIMOS observations have not been able to?  Surely some
%      level of the spectral tagging and photo-z training can be done
%      (and surely is being done) with existing data.  Is FOBOS going to
%      be a huge leap, or will it mainly be cleaning up neglected corners
%      of parameter space?
%
% - More excited to hear about how the FOBOS spectra will be used for
%   science directly, instead of support for LSST

%-----------------------------------------------------------------------

\input{Astro2020-Cosmology}
\input{Astro2020-DataScience}


%%%%
% -- Instrument Description
% --     FOBOS Keck White Paper 2019
%%%%

%%%%%%%%%%%%%%%%%%%%%%%%%%%%%%%%%%%%%%%%%%%%%%%%%%%%%%%%%%%%%%%%%%%%%%%%
\begin{figure}[h!]
%\vskip -0.1in
%\includegraphics[width=\textwidth]{figs/FOBOS_FocalPlane.pdf}
\includegraphics[width=0.8\textwidth]{figs/FOBOS_FocalPlane_v2.pdf}
\caption{\small {\it
Left}: Rendering of FOBOS focal plane system deployed at the Keck II
Nasmyth port. {\it Right}: Rendering of the ADC and focal surface with
Starbugs mounted (red cylinders).}
\label{fig:focalplane}
\end{figure}
%%%%%%%%%%%%%%%%%%%%%%%%%%%%%%%%%%%%%%%%%%%%%%%%%%%%%%%%%%%%%%%%%%%%%%%%

\section{Technical Description}
\label{sec:concept}
% \noindent \comment{1 page}

% Here's an alternative way to put in figures if we want captions on the side (to save space)
% Could introduce a new ``counter'' to count and label figures appropriately
%\centerline{\hbox{\includegraphics[width=0.6\textwidth, angle=0]{figs/FOBOSatKeck_v1.pdf}
%    \hspace{0.1cm} \vspace{2in}
%    \parbox[b]{0.3\textwidth}{\small {\bf Figure ??:} Rendering of FOBOS instrument systems deployed at the Keck II Nasmyth port.  By mounting the FOBOS spectrographs under the Nasmyth platform, other instruments like DEIMOS can maintain access to the telescope. \vspace{2cm}}}}

Mounted at the Nasmyth focus of Keck II Telescope at WMKO, FOBOS will
be one of the most powerful spectroscopic facilities deployed in the
next decade. FOBOS includes a compensating lateral atmospheric
dispersion corrector (CLADC; Fig.~\ref{fig:focalplane}) to ensure
that target light from all wavelengths falls on allocated fibers
while also correcting image aberrations at the edges of the 17~arcmin
diameter Keck field. Each of the CLADC lenses is $\sim$700~mm in
diameter, the first two are closely spaced with lateral relative
motions of element one supplied by a single axis of motion acting
along a curve equal to the radius of curvature of the first lens
surface (1028~mm). The total offset of this lens is rather small at
$\sim1$00~mm. The final CLADC lens translates by $\sim$50~mm and
tilts slightly to track the focal-plane shift. This last element acts to
correct the telecentricity error into the fiber system and acts as the
drive surface for the Starbug positioning system.
Starbugs patrol a large on-sky area ($\sim$1~arcmin), enabling
flexible and dynamic targeting configurations with adjacent fibers as
close as 10~arcsec.

% Reni- Can you check my numbers on the CLADC motion? - NKM
% How much do we go into risks?  How about this - NKM

Starbugs, first proposed in 2004 \citep{2004SPIE.5495..600M}, and
later perfected by AAO for use on TAIPAN \citep{2016SPIE.9912E..1WS}
are a truly remarkable fiber-positioning system. They move by {\it
walking} on the focal plane using a pair of piezo tube actuators. A
weak vacuum adheres the Starbugs to the surface of the field plate
and provides the frictional normal forces needed to allow for the
walking action of the piezo tubes. Positional feedback is provided by
way of a camera imaging back-illuminated fibers on the focal plane.
This system allows for a highly configurable focal plane both in
terms of target densities and configuration of the fibers within an
individual actuator. The Starbugs may be used with a single fiber or
with a bundle of fiber making up an IFU. The TAIPAN instrument,
currently on sky conducting a large galaxy survey, is the proof test
for the readiness of this technology. It is worth noting that,
although Starbugs are our preferred and baseline positioning
technology, no aspect of FOBOS's current front-end design precludes
using a zonal system, such as those used for MOONS, PFS, or DESI. The
last element of the CLADC can be eliminated and replaced with a zonal
actuator bed that conforms to the focal plane shape. Telecentricity
can be maintained by alignment of the actuator axis to the incoming
beam as is currently being designed for the SDSS-V robotic
focal-plane system.

A total of 1800 fibers with 150-$\mu$m core diameter are deployed at
the curved focal plane. Microlens fore-optics convert the f/15 Keck
input beam to a faster f/3.2 focal ratio, which both demagnifies the
entrance aperture and allows for a better coupling to the fiber
numerical aperture that minimizes losses from focal ratio degradation
(FRD). The focal-plane plate rotates and translates to follow image
positions as the telescope tracks across the sky. The fiber run is
kept as short as possible to maintain high throughput at UV
wavelengths (a 10~m Polymicro Silica fiber transmits $\sim$70\% and
$\sim$85\% of light at 310~nm and 350~nm, respectively). Special care
is given to stress-relief cabling to minimize instabilities (e.g.,
variable FRD) over the fiber run. To maintain the highest possible
transition efficiency there are no connectors used within the fiber
run. When FOBOS is not in use, the focal-plane unit detaches from the
front-end, ADC module and its associated robotics, and it is stored
with the spectrographs on the Nasmyth platform. This allows the ADC
module to be transferred to any of the instrument park positions. All
other Keck-II instruments can still be used without modification.

%%%%%%%%%%%%%%%%%%%%%%%%%%%%%%%%%%%%%%%%%%%%%%%%%%%%%%%%%%%%%%%%%%%%%%%%
\begin{figure}[h!]
\vskip -0.1in
\includegraphics[width=0.96\textwidth]{figs/FOBOS_spec_optical-CAD.png}
\caption{\small Optical design (left) and mechancial rendering (right) of a 4-channel FOBOS spectrograph employing catadioptric cameras.
Light from a 600-fiber pseudo-slit strikes a collimating mirror and then
passes back through subsequent dichroics before entering each grating-camera unit.}
\label{fig:spectrograph}
\end{figure}
%%%%%%%%%%%%%%%%%%%%%%%%%%%%%%%%%%%%%%%%%%%%%%%%%%%%%%%%%%%%%%%%%%%%%%%%

FOBOS's three identical spectrographs (Fig.~\ref{fig:spectrograph})
are each fed by a pseudoslit of 600 fibers. Each FOBOS spectrograph
uses a series of dichroics to divide the 259~mm collimated beam into
four wavelength channels, providing an instantaneous broad-band
coverage from 0.31--1 $\mu$m. Fused-silica etched (FSE) gratings
provide mid-channel spectral resolutions of $R\sim3500$ at high
diffraction efficiency in each channel. The dispersed light is
focused by an f/1.1 catadioptric camera\footnote{Based on the camera
design for the Multi-Object Optical and Near-infrared Spectrograph
(MOONS) on the Very Large Telescope (VLT).} and recorded by an
on-axis 4k$\times$4k CCD mounted at the center of the first camera
lens element. Unlike the mountable ADC module, the spectrographs are
housed in a permanent temperature-controlled structure on the Nasmyth
deck. The end-to-end instrument throughput peaks at 60\% and is
greater than 30\% at {\it all} wavelengths.

FOBOS will include observatory level systems for precise instrument
calibration using dome-interior screen illumination, a metrology system
for accurate fiber positioning, and guide cameras for field acquisition
and guiding.  Initial deployment of the focal-plane will focus on a
single-fiber format, with a secondary deployment of multi-format fiber
bundles.  Beyond FOBOS and outside this proposal, future instruments could share the focal plane, integrating their
fiber feeds to separate spectrographs optimized for higher spectral resolution, and/or
different wavelengths (e.g., the near-IR).  FOBOS is also ideal for taking full advantage of a future
Ground-Layer Adaptive Optics capability.



%%%%
% -- Proposed Work and Budget
% --     FOBOS Keck White Paper 2019
%%%%

% ========================================================================
% - leverage experience from Shane prime focus ADC; Harland design; Matt
%   Radovan may know; Dave Cowley probably best person to talk to

\subsection{Technology Drivers}
\label{sec:design}

FOBOS will provide key capabilities in the near-term thanks to
deployment at the existing Keck II telescope. It both carries a
relatively modest cost compared to other proposed large-scale
spectroscopic facilities (e.g., MSE, SpecTel) and helps lay the
groundwork for their realization. Thus, while FOBOS will prove to be
a valuable long-term investment for the W.~M.~Keck Observatory, it
can also provide for invaluable technological development leading to
efficiency and cost-cutting strategies for these larger facilities.

\subsubsection{Starbugs fiber positioners} Starbugs are a positioning
technology developed and deployed by Australian Astronomical Optics (AAO), which has partnered with our team to
generate a conceptual design for use of Starbugs by FOBOS. The Starbugs positioning systems is highly attractive
because of its flexibility both in terms of configuring a given set of fibers, as well as the prospect of exchanging
different groups of Starbugs with different payloads and/or those that feed different spectrographs (e.g., high vs.\
low resolution). With such a flexible focal plane deployment, FOBOS can serve as a platform for cost-cutting technology
development, which is not possible with fixed-format instruments like PFS and DESI. Starbugs are currently being tested
on-sky with the TAIPAN instrument at the UK Schmidt Telescope and published results on their performance are expected
in summer 2019.

% \subsubsection{Spectrograph Cost} \comment{to edit} Here we make the
% case that FOBOS is a platform for figuring out how to build future
% dedicated spectroscopic telescopes like SpecTel cheaper.

\subsubsection{Data Systems} A key to FOBOS's success with the
development of robust data-reduction and data-analysis pipelines,
building on the heritage of efforts within SDSS, DESI, and MaNGA. In
particular, the FOBOS data-analysis pipeline (DAP) will take
advantage of the fixed spectral format and common target classes to
provide high-level data products, including Doppler shifts,
emission-line strengths, and template continuum fits (cf., Westfall
et al.; SDSS-IV MaNGA DAP). Planning will include development of
user-friendly platforms built on the Keck Observatory Archive for
serving raw data, reduced spectra, and DAP science products.

% \comment{to edit/remove} We will develop the requirements and initial
% concept for the FOBOS data simulator following instrument
% forward-modeling techniques developed by DESI. The simulator will
% enable tests of potential FOBOS science cases (e.g., exposure time
% estimates, redshifting success). For the MSIP proposal, we will
% construct a detailed plan for evolving the data simulator into
% data-reduction and data-analysis pipelines.

\subsection{Current Status} FOBOS is currently in its conceptual design
phase, building from a down-selection process as one of the designs for
the Wide-Field Optical Spectrograph for TMT. Recently, FOBOS has been
awarded Phase-A funding from WMKO Observatory, receiving a full
design-phase endorsement from the Keck Science Steering Committee. These
funds are devoted to completing the conceptual design in preparation for
future funding proposals, particularly the NSF MSIP and MsRI calls.

\subsection{Cost Estimates and Schedule}

Cost estimates for FOBOS reflect its current development phase and are reported in Table \ref{tab:cost} where
available.  Our conceptual-design-phase estimates have higher fidelity for the near-term phases of Preliminary Design
and the beginning of Final Design.  Where possible, costing efforts are based on quotes, and labor efforts are based on
experience with similar systems developed by the institution responsible for a given sub-system.  Our cost projections
in combination with experience from other instruments place FOBOS in the category of a medium-scale ground-based
instrument program suitable for MsRI-2 construction funding. % at \$37.5M.

Nearly all costs prior to Final Design 2 are allocated to design
efforts; however, a small amount is devoted to prototyping needed to
mitigate risk in key systems.  Our project execution plan divides Final
Design into two phases to allow for a gate for long-lead-time contracts
after a review in June of 2024, such as procurement and construction of
the ADC optics.  The Integration phase also overlaps with the
construction phase to allow for the facility system build-out at Keck
Observatory and to allow for a phased deployment of the multiplexed
focal-plane system.  A full project review is scheduled at the end of
each design phase.  A pre-ship review will be held during integration
prior to delivery of the first spectrograph.   Smaller sub-system
reviews will be held as required.

\begin{table}[h!]
\centering
\footnotesize
\caption{Nominal Schedule and Cost Estimates for FOBOS}
\label{tab:cost}
\vspace*{-10pt}
\begin{tabular}{l | l l r l }
\hline
Phase              &  Start  &     End &         Cost & Fidelity \\
                   &         &         &  (2019 USD) &  \\
\hline
\hline
Conceptual Design  & Q2 2018 & Q1 2021 &   \$730k & Resource-loaded schedule with progress tracking \\
Preliminary Design & Q2 2021 & Q2 2023 &  \$2.6M & Resource-loaded schedule \\
Final Design One   & Q3 2023 & Q2 2024 &  \$1.5M & High-level tasks with cost/effort estimates \\
Final Design Two   & Q3 2024 & Q2 2025 &  \$1.5M & High-level tasks with cost/effort estimates \\
Construction       & Q3 2025 & Q1 2027 & TBD & Block schedule with low-fidelity cost/effort estimates \\
Integration        & Q2 2026 & Q3 2027 & TBD & Block Schedule with low-fidelity cost/effort estimates \\
Commissioning      & Q4 2027 & Q1 2028 &   TBD & Block Schedule with low-fidelity cost/effort estimates \\
\hline
\end{tabular}
\end{table}


% Conceptual Design  & Q2 2018 & Q1 2021 &   730 & Resource-loaded schedule with progress tracking \\
% Preliminary Design & Q2 2021 & Q2 2023 &  2580 & Resource-loaded schedule \\
% Final Design One   & Q3 2023 & Q2 2024 &  1500 & High-level tasks with cost/effort estimates \\
% Final Design Two   & Q3 2024 & Q2 2025 &  8000 & High-level tasks with cost/effort estimates \\
% Construction       & Q3 2025 & Q1 2027 & 14000 & Block schedule with low-fidelity cost/effort estimates \\
% Integration        & Q2 2026 & Q3 2027 & 10000 & Block Schedule with low-fidelity cost/effort estimates \\
% Commissioning      & Q4 2027 & Q1 2028 &   700 & Block Schedule with low-fidelity cost/effort estimates \\







% \noindent \textbf{Operations:} Powered by Starbugs fiber positioners,
% FOBOS will enable fast ($<$2 minute), dynamic reallocation of fibers. We will
% develop an initial target allocation simulator to determine
% efficiencies for various science programs and explore options for
% program combination and optimization under different observing
% scenarios. This work requires planning interfaces with the fiber
% positioning control software and the Keck user.


% To efficiently
% determine the best options given a wide range of possible targets and
% desired observing outcomes, we will develop a conceptual design for
% MAISTRO,\footnote{MAISTRO: Modular Artificial Intelligence System for
% Target Reallocation and Observing.} an ``artificial intelligence''
% (AI) targeting system that will learn optimization strategies for
% assigning targets from a database of overlapping observing programs
% with pre-defined priorities. The AI package will aggregate data
% quality using a quick-look reduction package, science-driven
% performance metrics, {\it and real-time assessments of the observing
% conditions} to make dynamic targeting recommendations. For example,
% if conditions are slightly less than optimal, MAISTRO would
% reconfigure Starbugs to brighter objects in a field or implement a
% different program prioritization. MAISTRO will incorporate updated
% target lists and priorities from the active observer and could easily
% be over-ridden at any time. Fractions of the full FOBOS multiplex
% might also be reserved ``manual targeting'' as required by the
% program PI.

%   - maintains a database with observational progress on individual
%     targets in the survey and
%   - dynamically reallocates fibers based on real-time assessments of
%     the aggregate S/N of each target to meet the specific need of each
%     science case.

% This requires significant design and testing of a combined software
% package and hardware interface.  Specific considerations involve (1)
% fast and robust reduction procedures (cf. MaNGA DOS) that can assess
% the aggregate data and (2) a responsive database with a schema
% optimized for real-time decision making to select targets for
% (re)acquisition while accounting for collision limitations.  Provided
% enough design effort, this lends itself to a machine-learning
% application.




% \pagebreak
\clearpage
\begin{multicols}{2}
\scriptsize
\bibliographystyle{apj}
\bibliography{references}
\end{multicols}

% \textbf{References}



\end{document}
